\documentclass[12pt]{report}

% Idioma y codificación
\usepackage[spanish]{babel}
\usepackage[utf8]{inputenc}

% Comillas tipográficas con \enquote
\usepackage[autostyle=true]{csquotes}


% Márgenes y tamaño de página
% Superior: 3 cm, Inferior: 3 cm, Izquierdo: 3 cm, Derecho: 2 cm
\usepackage[letterpaper,top=3cm,bottom=3cm,left=3cm,right=2cm]{geometry}

% Fuente tipo Arial (aprox. Helvetica) en todo el documento
\usepackage{helvet}
\renewcommand{\familydefault}{\sfdefault}

% Paquetes útiles
\usepackage{natbib}
\usepackage{graphicx}
\usepackage{svg}        %svg files
%\usepackage[inkscapeversion=1,inkscapeexe={C:/Program Files/Inkscape/bin/inkscape.exe}]{svg}
\usepackage{float}
\usepackage{xcolor}
\usepackage{amsmath}
\usepackage[colorlinks=true, allcolors=black]{hyperref}
\usepackage{setspace}
\usepackage{tocloft}    % Para personalizar tabla de contenidos y listas
\usepackage[bottom]{footmisc}
%\usepackage{makeidx}    % Para índice analítico (ÍNDICE)
\usepackage{titlesec}   % Para formatear títulos y subtítulos
\usepackage{ragged2e}
\usepackage{tabularx}
\usepackage{booktabs}
\usepackage{subcaption} % to caption of subfigures

\usepackage{tikz}
\usetikzlibrary{arrows.meta,positioning,shapes.geometric,fit,calc}


\usepackage{tikz}
\usepackage[spanish]{babel}
\usetikzlibrary{babel}
\usetikzlibrary{arrows.meta,shapes.geometric,positioning}

\tikzset{
  block/.style={
    rectangle, draw, rounded corners,
    text width=4cm,        % ancho máximo del texto
    align=center,
    minimum height=8mm
  },
  decision/.style={
    diamond, draw, aspect=2,
    text width=4cm,        % rombos más altos y menos anchos
    align=center,
    inner sep=1pt
  },
  startstop/.style={
    ellipse, draw,
    text width=3cm,
    align=center
  },
  line/.style={draw, -{Latex[length=3mm]}}
}






\doublespacing
%\makeindex

% Cambiar título de las referencias (dejamos como lo tenías)
\AtBeginDocument{
  \renewcommand{\refname}{REFERENCIAS}
}

% ===========================
% Formato de títulos y subtítulos
% ===========================

% Secciones (títulos principales) centrados
\titleformat{\section}
  {\normalfont\bfseries\large\centering} % formato
  {\thesection}                          % número de sección
  {1em}                                  % separación número-título
  {}                                     % código antes del título

% Subsecciones (subtítulos) alineados a la izquierda
\titleformat{\subsection}
  {\normalfont\bfseries\large} % formato
  {\thesubsection}
  {1em}
  {}

% Subsubsecciones también a la izquierda
\titleformat{\subsubsection}
  {\normalfont\bfseries} % formato
  {\thesubsubsection}
  {1em}
  {}

% ===========================
% "pág." en Contenido, Lista de Tablas y Lista de Figuras
% ===========================

% Después del título CONTENIDO
\renewcommand{\cftaftertoctitle}{%
  \par\vspace{0.5em}%
  \noindent\hfill\textbf{pág.}\par\vspace{1em}%
}

% Después del título LISTA DE TABLAS
\renewcommand{\cftafterlottitle}{%
  \par\vspace{0.5em}%
  \noindent\hfill\textbf{pág.}\par\vspace{1em}%
}

% Después del título LISTA DE FIGURAS
\renewcommand{\cftafterloftitle}{%
  \par\vspace{0.5em}%
  \noindent\hfill\textbf{pág.}\par\vspace{1em}%
}

%% Recomendaciones del comite
% los títulos correspondientes al material complementario (DESDE LA BIBLIOGRAFIA) se deben escribir con mayúscula sostenida y se indica la página donde están ubicados. No se deben anteceder por numerales.

%Lista alfabética de términos y sus definiciones o explicaciones necesarios para la comprensión del documento. El GLOSARIO tiene carácter opcional y su existencia no justifica la omisión de una explicación la primera vez que aparece un término. Los términos se deben escribir con mayúscula sostenida seguidos de dos puntos y en orden alfabético. La definición correspondiente se coloca después de los dos puntos, se deja un espacio y se inicia con minúscula. Entre termino y termino se deja una interlinea. 

% El resumen debe ser de máximo 500 palabras, presentando el documento de forma abreviada y precisa, sin interpretación del contenido. La palabra resumen se escribe centrada, a 3cm del borde superior, en mayúscula sostenida. El texto debe estar separado por una línea en blanco. Al final del resumen deben aparecer las palabras claves tomadas del texto. La misma información debe repetirse en inglés, en la misma página. Para mejor comprensión de cómo hacer un resumen véase la norma ISO 214: 1976. No debe usarse más de una página para el resumen.

% Presentación del documento. Debe decirse brevemente por qué este es importante, antecedentes, objetivos, metodología y aplicación en el área de conocimiento. No debe confundirse con el resumen, ni contener un recuento detallado de la teoría, tampoco anticipar conclusiones y recomendaciones. 

% En ningún caso debe ser una repetición del Anteproyecto. Mientras el anteproyecto se escribe en futuro (En este proyecto se pretende desarrollar…), la introducción debe redactarse en pasado (En este proyecto se desarrolló…).

% El número correspondiente al primer nivel debe llevar punto final. Los títulos de primer nivel de los capítulos se escriben con mayúscula sostenida, centrados, al borde del margen y precedido por el numeral correspondiente. El TITULO, no lleva punto final y el texto y/o contenido del capítulo correspondiente será separado por dos interlineas y/o espacios.


% Los títulos de segundo nivel (Subcapítulos) se deben escribir con mayúscula al margen izquierdo; no deben llevar punto final y se deben presentar a dos espacios del numeral, separados del texto o contenido por dos interlineas y/o espacios. Entre los números que designan la subdivisión debe escribirse un punto, sin embargo, después del número que designa el ultimo nivel NO SE ESCRIBE PUNTO. 

% Del tercer nivel en adelante, los títulos se deben escribir con mayúscula inicial y punto seguido. El texto debe continuar en el mismo renglón, dejando un espacio, después del punto seguido. 


% ACLARATORIAS: 

% ¡NO SE DEBEN DEJAR TITULOS A FINAL DE LA PÁGINA, SIN TEXTO!

% De la quinta subdivisión en adelante, cada nueva división o ítem puede ser señalada con viñetas, conservando el mismo estilo de ésta, a lo largo de todo el documento. Las subdivisiones, las viñetas y sus textos acompañantes deben presentarse sin sangría y justificados.

% Sobre las ilustraciones (tablas, cuadros, figuras y otros…)

% NO emplear la abreviatura de número ni el signo #. 

% La fuente documental se debe escribir al pie de la ilustración y no como pie de página. Si es elaboración propia entonces escribir: “Fuente: elaboración propia” O “Fuente: elaboración propia, con base en (cite la fuente)” 

% El nombre de cada ilustración debe escribirse en la parte superior y al margen izquierdo de la misma, después de la palabra “Tabla 1.” O “Cuadro 2.” O etcétera. Se deben utilizar números arábigos y en orden consecutivo a lo largo del texto. Deben llevar un título breve sobre su contenido.

% Si la figura ocupa más de una página, se debe repetir su identificación numérica seguida por la palabra: “Continuación”, con mayúscula inicial y entre paréntesis. Del mismo modo, en este caso, los encabezados de las columnas se deben repetir en todas las páginas después de la primera.

% Conclusiones: Se presentan de forma lógica, los resultados del trabajo. Las conclusiones deben ser la respuesta a los objetivos o propósitos planteados. El autor debe sintetizar todo lo expuesto. 
% Este capítulo independiente debe titularse con la palabra: CONCLUSIONES, escrita en mayúscula sostenida, centrada, al borde del margen superior, precedida por el numeral correspondiente y separada del contenido por dos espacios y/o interlineas. Cuando se requiera diferenciar cada una de las conclusiones, se recomienda usar viñetas. 

% RECOMENDACIONES: De carácter opcional, se debe titular con la palabra: RECOMENDACIONES, en mayúscula sostenida, centrada, al borde superior del margen, precedida por el numeral correspondiente y separada del texto por dos interlineas. 
% En este capítulo, se escriben las sugerencias, proyecciones o alternativas que se presentan para modificar, cambiar o incidir sobre una situación específica o una problemática encontrada. 

\begin{document}

%%%%%%%%%%%%%%%%%%%%%%%%%%%%%%%%%%%%%%%%%%%%%%%%%%%%%
%% PORTADA
%%%%%%%%%%%%%%%%%%%%%%%%%%%%%%%%%%%%%%%%%%%%%%%%%%%%%

\begin{titlepage}
	\centering
    \linespread{1.2}
	{\large \textbf{APLICACIÓN PARA LA MONITORIZACIÓN DE LA PRUEBA TIMED UP AND GO, CON UN TELÉFONO INTELIGENTE, GESTIÓN DE INFORMACIÓN EN LA NUBE Y NIVEL DE MADUREZ TECNOLÓGICA 6} \par}
	\vfill
	{\large John Sebastian Chamorro Narváez \par}
	\vfill
	{\large Universidad del Valle\par}
	{\large Facultad de Ingeniería \par}
	{\large Escuela de Ingeniería Eléctrica y Electrónica \par}
	{\large Santiago de Cali \par}
	{\large 2025 \par}
\end{titlepage}

%%%%%%%%%%%%%%%%%%%%%%%%%%%%%%%%%%%%%%%%%%%%%%%%%%%%%
%% CONTRA PORTADA
%%%%%%%%%%%%%%%%%%%%%%%%%%%%%%%%%%%%%%%%%%%%%%%%%%%%%

\newpage
\thispagestyle{empty}
\begin{center}
    \linespread{1.2}
	{\large \textbf{APLICACIÓN PARA LA MONITORIZACIÓN DE LA PRUEBA TIMED UP AND GO, CON UN TELÉFONO INTELIGENTE, GESTIÓN DE INFORMACIÓN EN LA NUBE Y NIVEL DE MADUREZ TECNOLÓGICA 6} \par}
	\vfill
	{\large John Sebastian Chamorro Narváez \par}
	\vfill
	{\large Trabajo de grado para optar por el título de: \par}
	{\large Ingeniero Electrónico \par}
	\vfill
	{\large Directores: \par}
	{\large Dr.-Ing. Esteban Rosero \par}
	{\large José Miguel Ramírez Scarpetta, Ph.D. \par}
	\vfill
	{\large Grupo de Investigación en Control Industrial (GICI) \par}
	\vfill
	{\large Universidad del Valle \par}
	{\large Facultad de Ingeniería \par}
	{\large Escuela de Ingeniería Eléctrica y Electrónica \par}
	{\large Santiago de Cali \par}
	{\large 2025 \par}
\end{center}

%%%%%%%%%%%%%%%%%%%%%%%%%%%%%%%%%%%%%%%%%%%%%%%%%%%%%
%% NOTA DE ACEPTACIÓN
%%%%%%%%%%%%%%%%%%%%%%%%%%%%%%%%%%%%%%%%%%%%%%%%%%%%%

\newpage
\thispagestyle{empty}

\noindent\textbf{Nota de aceptación:} 

\vspace{3cm}

\rule{\textwidth}{0.4pt}

\vspace{0.8cm}

\rule{\textwidth}{0.4pt}

\vspace{0.8cm}

\rule{\textwidth}{0.4pt}

\vspace{0.8cm}

\rule{\textwidth}{0.4pt}

\vspace{0.8cm}

\rule{\textwidth}{0.4pt}

\vspace{3cm}

\noindent\begin{minipage}[t]{0.45\textwidth}
\centering
\rule{6cm}{0.4pt}\\
Firma del director del trabajo
\end{minipage}
\hfill
\begin{minipage}[t]{0.45\textwidth}
\centering
\rule{6cm}{0.4pt}\\
Firma del evaluador
\end{minipage}

\vspace{3cm}

\noindent\centering
\rule{6cm}{0.4pt}\\
Firma del evaluador

\vfill

\noindent Santiago de Cali, \rule{3cm}{0.4pt}.

%%%%%%%%%%%%%%%%%%%%%%%%%%%%%%%%%%%%%%%%%%%%%%%%%%%%%
%% DEDICATORIA (opcional)
%%%%%%%%%%%%%%%%%%%%%%%%%%%%%%%%%%%%%%%%%%%%%%%%%%%%%

\begin{justify}
\newpage
\thispagestyle{empty}
\begin{center}
\textbf{DEDICATORIA}
\end{center}

\vspace{1cm}

Este trabajo está dedicado a mis tres hermanos, mi padre y mi madre, quienes han estado presentes de manera constante a lo largo de todas las etapas de mi vida, brindándome apoyo, comprensión y fortaleza, incluso en esta fase adulta, para poder culminar este proceso académico.\\

De manera muy especial, dedico este trabajo a mi compañera de vida, por su paciencia, aliento y acompañamiento incondicional, y por ofrecerme apoyo y serenidad en aquellos momentos en los que el camino académico y personal se hizo más exigente.\\

A todos ellos, mi más sincero agradecimiento y cariño.

%%%%%%%%%%%%%%%%%%%%%%%%%%%%%%%%%%%%%%%%%%%%%%%%%%%%%
%% AGRADECIMIENTOS (opcional)
%%%%%%%%%%%%%%%%%%%%%%%%%%%%%%%%%%%%%%%%%%%%%%%%%%%%%

\newpage
\thispagestyle{empty}
\begin{center}
\textbf{AGRADECIMIENTOS}
\end{center}

\vspace{1cm}

Agradezco de manera especial a mis directores de trabajo de grado, el Dr.-Ing. Esteban Rosero y José Miguel Ramírez Scarpetta, Ph.D., por su acompañamiento académico, orientación técnica y disposición permanente durante el desarrollo de este proyecto. Sus aportes y exigencia intelectual fueron fundamentales para la consolidación y calidad del presente trabajo.\\

Expreso igualmente mi agradecimiento a la Universidad del Valle, mi alma máter, por brindarme una formación académica sólida y un entorno propicio para el desarrollo profesional y personal. De manera particular, agradezco a la Escuela de Ingeniería Eléctrica y Electrónica por el apoyo institucional y académico recibido a lo largo de mi proceso formativo.\\

Agradezco al Grupo de Investigación GICI por proporcionar el espacio académico, los recursos y el acompañamiento necesarios para el desarrollo de este trabajo, así como por fomentar un ambiente de investigación colaborativo y riguroso.\\

Finalmente, agradezco profundamente a mi familia por su apoyo constante, comprensión y motivación a lo largo de todo este proceso, siendo un pilar fundamental para alcanzar este logro académico.

\end{justify}

%%%%%%%%%%%%%%%%%%%%%%%%%%%%%%%%%%%%%%%%%%%%%%%%%%%%%
%% CONTENIDO (TABLA DE CONTENIDOS)
%%%%%%%%%%%%%%%%%%%%%%%%%%%%%%%%%%%%%%%%%%%%%%%%%%%%%

\clearpage
\pagenumbering{arabic} % numeración arábiga desde aquí

\renewcommand{\contentsname}{CONTENIDO}


\tableofcontents

%%%%%%%%%%%%%%%%%%%%%%%%%%%%%%%%%%%%%%%%%%%%%%%%%%%%%
%% LISTA DE TABLAS
%%%%%%%%%%%%%%%%%%%%%%%%%%%%%%%%%%%%%%%%%%%%%%%%%%%%%

\clearpage
\renewcommand{\listtablename}{LISTA DE TABLAS}
%\addcontentsline{toc}{section}{LISTA DE TABLAS}
\listoftables

%%%%%%%%%%%%%%%%%%%%%%%%%%%%%%%%%%%%%%%%%%%%%%%%%%%%%
%% LISTA DE FIGURAS
%%%%%%%%%%%%%%%%%%%%%%%%%%%%%%%%%%%%%%%%%%%%%%%%%%%%%

\clearpage
\renewcommand{\listfigurename}{LISTA DE FIGURAS}
%\addcontentsline{toc}{section}{LISTA DE FIGURAS}
\listoffigures

%%%%%%%%%%%%%%%%%%%%%%%%%%%%%%%%%%%%%%%%%%%%%%%%%%%%%
%% GLOSARIO
%%%%%%%%%%%%%%%%%%%%%%%%%%%%%%%%%%%%%%%%%%%%%%%%%%%%%

\begin{justify}

\clearpage
\section*{GLOSARIO}
%\addcontentsline{toc}{section}{GLOSARIO}

\noindent \textbf{ACELERÓMETRO:} Sensor que mide la aceleración lineal de un cuerpo en uno o varios ejes, incluyendo tanto aceleraciones debidas al movimiento como a la gravedad.  
\vspace{0.3cm}

\noindent \textbf{API (Application Programming Interface):} Conjunto de definiciones, protocolos y funciones que permite la comunicación e intercambio de datos entre diferentes componentes o servicios de software.  
\vspace{0.3cm}

\noindent \textbf{BASE DE DATOS RELACIONAL:} Sistema de gestión de datos que organiza la información en tablas relacionadas entre sí mediante claves primarias y foráneas, facilitando la integridad y consistencia de los datos.  
\vspace{0.3cm}

\noindent \textbf{BLAND--ALTMAN:} Método estadístico utilizado para evaluar el grado de acuerdo entre dos métodos de medición cuantitativos, basado en el análisis del sesgo y los límites de concordancia.  
\vspace{0.3cm}

\noindent \textbf{COEFICIENTE DE CORRELACIÓN INTRACLASE (ICC):} Estadístico utilizado para cuantificar la confiabilidad o concordancia entre mediciones realizadas por diferentes métodos u observadores, considerando tanto la variabilidad entre sujetos como el error de medición.  
\vspace{0.3cm}

\noindent \textbf{GIROSCOPIO:} Sensor que mide la velocidad angular de rotación de un cuerpo alrededor de uno o varios ejes.  
\vspace{0.3cm}

\noindent \textbf{IMU (Inertial Measurement Unit):} Sensor inercial que integra acelerómetros y giroscopios, y en algunos casos magnetómetros, para estimar movimiento, orientación y cambios posturales.  
\vspace{0.3cm}

\noindent \textbf{MDC (Minimal Detectable Change):} Cambio mínimo en una medición que puede interpretarse como un cambio real, por encima del error de medición, con un nivel de confianza estadística predefinido.  
\vspace{0.3cm}

\noindent \textbf{MICROSERVICIO:} Arquitectura de software en la cual una aplicación se compone de servicios independientes, especializados y desacoplados, que se comunican entre sí mediante interfaces bien definidas.  
\vspace{0.3cm}

\noindent \textbf{PLATAFORMA CLIENTE--SERVIDOR:} Modelo de arquitectura de software donde los clientes realizan solicitudes y un servidor centralizado gestiona la lógica, el procesamiento y el almacenamiento de los datos.  
\vspace{0.3cm}

\noindent \textbf{SENSOR INERCIAL:} Dispositivo electrónico utilizado para medir aceleraciones lineales, velocidades angulares y orientación espacial, comúnmente empleado en aplicaciones de análisis del movimiento humano.  
\vspace{0.3cm}

\noindent \textbf{TELEMETRÍA:} Técnica que permite la adquisición, transmisión y análisis remoto de datos medidos por sensores, utilizando medios electrónicos y redes de comunicación.  
\vspace{0.3cm}

\noindent \textbf{TIMED UP AND GO (TUG):} Prueba clínica funcional utilizada para evaluar movilidad, equilibrio y riesgo de caídas, basada en el tiempo que tarda un individuo en levantarse de una silla, caminar una distancia fija, girar, regresar y sentarse nuevamente.  
\vspace{0.3cm}

\noindent \textbf{TRL (Technology Readiness Level):} Escala que clasifica el nivel de madurez tecnológica de un sistema, desde conceptos básicos (TRL 1) hasta sistemas completamente operativos en entornos reales y comerciales (TRL 9).  
\vspace{0.3cm}



%%%%%%%%%%%%%%%%%%%%%%%%%%%%%%%%%%%%%%%%%%%%%%%%%%%%%
%% RESUMEN Y SUMMARY
%%%%%%%%%%%%%%%%%%%%%%%%%%%%%%%%%%%%%%%%%%%%%%%%%%%%%

\clearpage
\section*{RESUMEN}
%\addcontentsline{toc}{section}{RESUMEN}

En este trabajo se presenta el desarrollo y validación de una aplicación móvil Android para la instrumentación de la prueba funcional Timed Up and Go (TUG), orientada a la evaluación de la movilidad mediante el uso de sensores inerciales integrados en teléfonos inteligentes. El sistema desarrollado incluye una aplicación móvil capaz de operar en modo local y en línea, un servidor con mecanismos de autenticación y persistencia de datos, y un microservicio especializado para el procesamiento automático de señales crudas y el cálculo de variables de interés asociadas al TUG, alcanzando un nivel de madurez tecnológica cercano a TRL~6.\\

La validación del sistema se realizó mediante pruebas experimentales en el laboratorio del Servicio de Rehabilitación Humana de la Universidad del Valle, comparando las mediciones de tiempo obtenidas por la aplicación con un sensor inercial comercial de referencia (BTS GSensor). Se analizaron el tiempo total del TUG y las subfases de la prueba utilizando métricas de acuerdo (Bland--Altman), error absoluto y relativo, y el coeficiente de correlación intraclase (ICC). Los resultados evidencian un alto grado de concordancia y una confiabilidad excelente para el tiempo total del TUG, con diferencias sistemáticas pequeñas y límites de concordancia estrechos, situados por debajo de los valores de \textit{minimal detectable change} reportados en la literatura, lo que respalda su relevancia clínica.\\

En contraste, el análisis por subfases mostró una confiabilidad más limitada, atribuible principalmente a la sensibilidad de la segmentación temporal automática y a la corta duración de estos eventos. En conjunto, los resultados indican que la aplicación es una alternativa válida, de bajo costo y fácil implementación para la medición del tiempo total del TUG, mientras que el análisis detallado por subfases debe considerarse exploratorio y sujeto a futuras mejoras algorítmicas y validaciones clínicas ampliadas.\\

\vspace{0.5cm}
\noindent\textbf{Palabras clave:} Timed Up and Go, teléfonos inteligentes, telemetría, TRL 6, aplicaciones móviles.

\vspace{1cm}

\section*{ABSTRACT}
%\addcontentsline{toc}{section}{SUMMARY}

This work presents the development and validation of an Android mobile application for the instrumentation of the Timed Up and Go (TUG) functional test, aimed at mobility assessment using inertial sensors embedded in smartphones. The proposed system comprises a mobile application capable of operating in both offline and online modes, a backend server with authentication and data persistence mechanisms, and a dedicated microservice for automated processing of raw sensor signals and extraction of clinically relevant TUG variables, achieving a technological readiness level close to TRL~6.\\

System validation was conducted through experimental testing at the Human Rehabilitation Service laboratory of Universidad del Valle, comparing time measurements obtained by the mobile application against a commercial inertial reference sensor (BTS GSensor). Both total TUG time and individual subphases were analyzed using agreement metrics (Bland--Altman analysis), absolute and relative error measures, and the intraclass correlation coefficient (ICC). Results demonstrated a high level of agreement and excellent reliability for total TUG time, with small systematic differences and narrow limits of agreement, well below the \textit{minimal detectable change} values reported in the literature, supporting its clinical relevance.\\

In contrast, subphase analysis showed lower reliability, mainly due to the sensitivity of automatic temporal segmentation and the short duration of these events. Overall, the findings indicate that the proposed mobile application represents a valid, low-cost, and easily deployable alternative for measuring total TUG time, while subphase-level analysis should be considered exploratory and subject to further algorithmic refinement and extended clinical validation.\\

\vspace{0.5cm}
\noindent\textbf{Keywords:} Timed Up and Go, smartphones, telemetry, TRL 6, mobile applications.


%%%%%%%%%%%%%%%%%%%%%%%%%%%%%%%%%%%%%%%%%%%%%%%%%%%%%
%% CUERPO DEL TRABAJO
%%%%%%%%%%%%%%%%%%%%%%%%%%%%%%%%%%%%%%%%%%%%%%%%%%%%%

\clearpage

\chapter{INTRODUCCIÓN}

Las caídas constituyen uno de los principales problemas de salud en la población adulta mayor y se consideran un síndrome geriátrico de alta relevancia clínica, debido a su asociación con deterioro funcional, hospitalización y aumento de la mortalidad \cite{falls-in-elderly}. En respuesta a esta problemática, se han desarrollado diversas pruebas clínicas orientadas a la evaluación de la movilidad y el riesgo de caídas.\\

Entre estas pruebas, el \textit{Timed Up and Go} (TUG) se ha consolidado como una de las herramientas más utilizadas en la práctica clínica por su simplicidad, rapidez de aplicación y bajo requerimiento de recursos. La prueba permite evaluar de manera global el equilibrio y la movilidad funcional mediante la medición del tiempo requerido para levantarse de una silla, caminar una distancia corta, girar, regresar y sentarse nuevamente \cite{TimedUpAndGoTUG}. Diversos estudios han demostrado que el TUG presenta una correlación moderada con el riesgo de caídas y una alta confiabilidad interevaluador e intraevaluador en diferentes poblaciones, con valores de coeficiente de correlación intraclase (ICC) reportados entre 0.80 y 0.99 \cite{Sensibilidad_TUG}.\\

No obstante, la versión tradicional del TUG presenta una limitación importante, ya que la medición suele realizarse mediante un cronómetro, registrando únicamente el tiempo total de la prueba. Este enfoque impide el análisis detallado de las subfases del movimiento, tales como las transiciones posturales, la marcha y los giros, lo que dificulta la detección temprana de alteraciones sutiles en el patrón motor. En este contexto, diversos trabajos han demostrado que la instrumentación del TUG mediante sensores inerciales permite obtener información cinemática adicional con alto valor diagnóstico y pronóstico \cite{Convergent-Validity-wearable-sensors}.\\

El avance de los teléfonos inteligentes, que incorporan sensores inerciales y capacidades de procesamiento cada vez más potentes, ha permitido el desarrollo de soluciones accesibles para la evaluación funcional. Estudios recientes han reportado que aplicaciones móviles pueden alcanzar niveles de confiabilidad cercanos a los de sistemas de referencia de mayor costo, como plataformas de captura de movimiento o sensores inerciales dedicados, obteniendo valores de ICC cercanos a 0.9 para la medición del tiempo total del TUG \cite{Reliability_Accuracy_Falls_Risk}.\\

En este Trabajo de Grado se desarrolló una aplicación móvil Android para la instrumentación de la prueba TUG, como continuación y mejora de un trabajo de maestría previo \cite{Sistema-portable-TUG-AKuleshov}. La aplicación permite la gestión de pacientes, la ejecución de pruebas en modo local u online, el almacenamiento estructurado de los datos y el envío seguro de la información a un servidor central. Adicionalmente, se participó en el desarrollo de una plataforma web y microservicios asociados, en colaboración con el grupo de investigación GICI, para el procesamiento automático de los datos crudos provenientes de los sensores y la extracción de variables temporales y cinemáticas relevantes.\\

Finalmente, se realizó un proceso de validación comparando las mediciones obtenidas por la aplicación móvil con las de un sensor inercial de referencia (BTS GSensor), con el objetivo de evaluar la confiabilidad y el grado de acuerdo entre ambos métodos. Los resultados obtenidos permiten establecer el alcance, las fortalezas y las limitaciones del sistema desarrollado, así como su potencial aplicación en contextos clínicos y de investigación.\\


%%%%% HASTA AQUÍ (OK) Excepto por lo del "prototipo final"





%% PLANTEAMIENTO DEL PROBLEMA
\section{Planteamiento del problema}

Las caídas se definen como sucesos involuntarios que provocan la pérdida del equilibrio y ocasionan que el cuerpo impacte el suelo u otra superficie firme \cite{OMS-falls}. De acuerdo con la Organización Mundial de la Salud (OMS), las caídas constituyen una de las principales causas de traumatismos involuntarios y representan una problemática de salud pública con impacto significativo, especialmente en población mayor de 60 años y en países de ingresos medianos y bajos \cite{OMS-falls}. Además de la mortalidad asociada, una proporción considerable de caídas deriva en hospitalización y reducción de la movilidad, lo cual incrementa el riesgo de dependencia y deterioro funcional \cite{OMS-falls}.\\

En la práctica clínica, una estrategia frecuente para estimar el riesgo de caída y el estado funcional del paciente es la aplicación de pruebas de movilidad, como el \textit{Timed Up and Go} (TUG). Esta prueba mide el desempeño del paciente al levantarse de una silla, caminar tres metros, realizar un giro de 180$^\circ$, regresar y sentarse nuevamente. Aunque el tiempo total del TUG aporta información relevante, el análisis por subfases y variables de marcha (p.\ ej., velocidad, estabilidad o características del patrón motor) puede complementar la interpretación clínica y apoyar la toma de decisiones \cite{fragilidad-indicador}.\\

La instrumentación de la prueba mediante sensores permite capturar señales cinemáticas de interés y estimar automáticamente los tiempos de cada subfase. Sin embargo, las soluciones basadas en equipamiento especializado pueden resultar costosas o poco accesibles, lo que limita su adopción en entornos con recursos restringidos. En este contexto, el uso de teléfonos inteligentes como plataforma de adquisición representa una alternativa viable por su disponibilidad, portabilidad e integración de sensores inerciales.\\

Como antecedente directo, el trabajo de maestría de Pérez Kuleshova desarrolló un prototipo basado en Android para capturar señales y estimar los tiempos por subfases del TUG, complementado con un aplicativo de escritorio para el procesamiento posterior \cite{Sistema-portable-TUG-AKuleshov}. Si bien dicho prototipo cumplió con la adquisición y el análisis básico, presentaba limitaciones para su uso operativo: dependencia de transferencia manual de datos, ausencia de persistencia estructurada por paciente, baja autonomía del flujo clínico y falta de integración con una plataforma centralizada.\\

Por lo anterior, se identifica la necesidad de evolucionar la solución hacia un sistema autónomo, usable y conectado, que permita gestionar pacientes y pruebas, almacenar información de forma estructurada, y centralizar el procesamiento y la consulta de resultados. En consecuencia, la pregunta de investigación que orienta este Trabajo de Grado es:\\

\textit{¿Cómo escalar una solución móvil para instrumentar la prueba Timed Up and Go hacia un nivel de madurez tecnológica que permita su operación en condiciones relevantes, con gestión de pacientes y almacenamiento centralizado, y con validación frente a un sensor de referencia?}\\


%% JUSTIFICACIÓN




\section{Justificación}

Las caídas constituyen un problema de salud pública con consecuencias importantes en términos de lesiones, discapacidad y demanda de servicios de salud. En el contexto colombiano, reportes como Forensis evidencian que las caídas representan una causa frecuente de lesión en adultos mayores, asociándose con factores como deterioro neuromuscular, menor movilidad y otras condiciones que incrementan el riesgo \cite{Forensis_2020}. En este escenario, contar con herramientas de evaluación funcional accesibles y sistemáticas puede apoyar la detección oportuna de deterioro y contribuir al seguimiento de la movilidad en población vulnerable.\\

Desde una perspectiva de desarrollo tecnológico, el marco de Niveles de Madurez Tecnológica (TRL) proporciona una guía para evolucionar prototipos desde pruebas controladas hasta validación en entornos relevantes \cite{TRL-ayming}. El antecedente directo de este proyecto alcanzó un nivel de integración funcional en laboratorio, pero mantenía limitaciones operativas para su adopción: baja autonomía del flujo de trabajo (procesamiento posterior en computador), carencia de una base de datos organizada por paciente y ausencia de conectividad para centralizar información \cite{Sistema-portable-TUG-AKuleshov}.\\

En consecuencia, este Trabajo de Grado se justifica por la necesidad de:
\begin{itemize}
    \item Desarrollar una aplicación móvil con mayor usabilidad y autonomía, capaz de gestionar pacientes y almacenar pruebas localmente.
    \item Habilitar un modo en línea con autenticación para el envío seguro de resultados hacia una plataforma central, facilitando la gestión y el análisis institucional de la información.
    \item Implementar procesamiento en servidor mediante servicios especializados para extraer variables temporales y cinemáticas de interés a partir de datos crudos.
    \item Realizar validación experimental comparando las mediciones obtenidas por la aplicación con un sensor inercial de referencia, fortaleciendo la confiabilidad del sistema en un entorno relevante.
\end{itemize}

De este modo, el proyecto contribuye tanto a la práctica académica y clínica (medición estructurada y trazable del TUG) como al ecosistema de investigación, al centralizar datos que pueden utilizarse en análisis posteriores y en el desarrollo de mejoras algorítmicas orientadas a poblaciones reales.\\


%% DEFINICIÓN DE LOS OBJETIVOS

\section{Definición de los objetivos}

\subsection{Objetivo general}

Escalar la aplicación Timed Up and Go a un TRL 6 en la escala tecnológica, con gestión de la información en la nube.

\subsection{Objetivos específicos}

\begin{itemize}
    \item Especificar los requerimientos a nivel de TRL 6, para el sistema de monitorización de la prueba \textit{Timed Up and Go}.
    \item Desarrollar la aplicación móvil \textit{Timed Up and Go}.
    \item Desarrollar la aplicación web con almacenamiento en la nube.
    \item Validar la correcta funcionalidad de la aplicación \textit{Timed Up and Go} y probarla en un ambiente cercano al real.
\end{itemize}



\chapter{MARCO DE REFERENCIA}

\section{antecedentes}

El uso de teléfonos inteligentes en evaluación de marcha ha aumentado debido a su disponibilidad, portabilidad e incorporación de sensores inerciales. En una revisión exploratoria, Strongman et al. reportan evidencia de validez y confiabilidad en el uso de acelerómetros de \textit{smartphones} para capturar información cinemática de la marcha, destacando su potencial como alternativa de bajo costo frente a tecnologías especializadas \cite{reliabilty-smartphone-kinematic}. Adicionalmente, se han reportado resultados comparables para variables cinemáticas entre dispositivos móviles y sistemas de captura de movimiento, lo cual respalda su aplicabilidad en contextos clínicos y de investigación \cite{subtasks}.\\

En particular, la instrumentación de la prueba \textit{Timed Up and Go} (iTUG) ha mostrado un incremento del valor informativo frente al TUG tradicional, ya que permite extraer características temporales y cinemáticas asociadas a subfases del movimiento. Una revisión sistemática enfocada en iTUG destaca que la mayoría de propuestas emplean sensores inerciales y que la ubicación en la región lumbar baja es frecuente por su cercanía al centro de masa, facilitando la implementación clínica \cite{iTUG}.\\

\subsection{Trabajo previo y limitaciones}

Como antecedente directo, el trabajo de maestría de Pérez Kuleshova desarrolló un prototipo basado en \textit{smartphone} para capturar señales inerciales y estimar los tiempos de subfases del TUG, complementado con un aplicativo de escritorio para el procesamiento posterior \cite{Sistema-portable-TUG-AKuleshov}. Aunque se logró la adquisición y segmentación básica de subetapas, el flujo presentaba limitaciones para su adopción operativa: dependencia de transferencia manual de datos al computador, ausencia de persistencia estructurada por paciente y falta de integración con una plataforma centralizada. Asimismo, el trabajo plantea como líneas de mejora la optimización del procesamiento, una interfaz más usable y mecanismos de almacenamiento en la nube.\\

Este Trabajo de Grado retoma dicho antecedente y lo extiende hacia una solución autónoma y conectada: (i) aplicación móvil con gestión de pacientes y pruebas en modo local y online; (ii) envío autenticado al servidor; (iii) procesamiento en servidor mediante microservicio para extracción de variables temporales y cinemáticas.\\

\section{Prueba \textit{Timed Up and Go} e instrumentación}

El TUG es una prueba funcional ampliamente utilizada para evaluar movilidad básica mediante el tiempo requerido para levantarse de una silla, caminar tres metros, girar, regresar y sentarse \cite{Convergent-Validity-wearable-sensors}. Aunque el tiempo total es el indicador más empleado, el análisis por subfases puede aportar información adicional al discriminar en qué componente del movimiento se concentra el desempeño del paciente (por ejemplo, transición sentado--de pie, giros o marcha) \cite{Association-Between-Performance-on-TUG}.\\

La instrumentación del TUG mediante sensores inerciales permite estimar automáticamente el tiempo total y los tiempos por subfase, así como variables asociadas a aceleración y rotación, incrementando la riqueza del análisis clínico y de investigación \cite{iTUG}.\\

% Si esta figura es importante para tu documento, déjala aquí (sí aporta al marco de referencia).
\begin{figure}[H]
    \centering
    \includegraphics[width=0.9\textwidth]{Images/Esquema_susFases_TUG.png}
    \caption{Esquema de las subetapas de la prueba \textit{Timed Up and Go} \cite{Reproducibilidad-TUG}.}
    \label{subEtapasTug}
\end{figure}

\section{Sensores inerciales en teléfonos inteligentes}

Los teléfonos inteligentes integran acelerómetros y giroscopios útiles para capturar señales de movimiento. En la literatura se resalta la importancia de la frecuencia de muestreo para representar adecuadamente la dinámica de la marcha; sin embargo, frecuencias excesivamente altas pueden incrementar el ruido y el costo computacional, por lo que suele recomendarse justificar experimentalmente el valor seleccionado \cite{reliabilty-smartphone-kinematic}. Estas consideraciones son relevantes para aplicaciones móviles orientadas a instrumentación clínica.\\

\section{Variables de interés en iTUG}

La instrumentación del TUG permite extraer variables temporales y cinemáticas que complementan el tiempo total. En este Trabajo de Grado, el procesamiento incluye: tiempo total, tiempos por subfase, aceleración en ejes antero--posterior, medio--lateral y vertical, variables de rotación (picos y promedios) y métricas asociadas a flexión--extensión durante transiciones posturales. La literatura respalda que la segmentación por subfases y la extracción de características pueden aumentar la capacidad descriptiva del TUG instrumentado, especialmente para identificar componentes del movimiento que presentan mayor deterioro \cite{iTUG}.\\

% Se eliminaron listas largas de variables cinéticas/cinemáticas sin fuente explícita.
% Si deseas conservarlas, se recomienda citarlas directamente del trabajo previo o de un artículo específico.

\section{Niveles de madurez tecnológica (TRL)}

Los niveles de madurez tecnológica (\textit{Technology Readiness Levels}, TRL) fueron propuestos inicialmente por la NASA y adoptados posteriormente por múltiples organizaciones como un marco para describir el avance de una tecnología desde su concepción hasta su despliegue \cite{TRL-EuroFunding}. En este trabajo se busca alcanzar TRL 6, entendido como la validación de un prototipo funcional en un entorno relevante, con pruebas en condiciones cercanas a la operación real \cite{TRL-ayming}.\\

% Se recomienda resumir la descripción de TRL en lugar de listar los 9 niveles completos.
% Si el comité exige la lista completa, mantenla en anexos.

\section{Plataforma, nube y consideraciones de seguridad}

Dado que el alcance incluye envío de datos y almacenamiento centralizado, se consideran conceptos de telemetría (captura, transmisión y almacenamiento de señales) \cite{Telemetría}, así como prácticas básicas de seguridad para autenticación y protección de la información en servicios web \cite{Cyber-security-protocols}. Además, se contemplan principios de arquitectura de software como soporte para escalabilidad, mantenibilidad y separación de responsabilidades (aplicación, servidor y microservicios) \cite{Enterprise_architecture-frameworks,Software-architecture}.\\

\section{Confiabilidad, validez y acuerdo en estudios de medición}

La \textbf{confiabilidad} describe el grado en que una medición es consistente y reproducible bajo condiciones comparables. En estudios de instrumentos y comparación entre métodos, una métrica ampliamente utilizada es el \textbf{coeficiente de correlación intraclase} (ICC), que evalúa la proporción de variabilidad atribuible a diferencias reales frente al error de medición \cite{ShroutFleiss1979,McGrawWong1996}.\\

La \textbf{validez} en comparación entre métodos requiere cuantificar \textbf{acuerdo} y no únicamente asociación. Para ello, el método de \textbf{Bland--Altman} evalúa el sesgo promedio entre métodos y define límites de concordancia al 95\% que describen el rango esperado de discrepancia \cite{BlandAltman1986}.\\

\subsection{ICC(2,1) y criterios de interpretación}

El ICC(2,1) corresponde a un modelo de dos vías con efectos aleatorios y medición única, apropiado cuando se desea \textbf{acuerdo absoluto} entre métodos y se asume que los evaluadores/métodos son representativos de una población mayor \cite{ShroutFleiss1979,McGrawWong1996}. Para interpretación práctica, se emplean guías de clasificación (pobre, moderada, buena, excelente) \cite{KooLi2016}.\\

\subsection{Bland--Altman: sesgo y límites de concordancia}

Sea $d_i = M_{app,i} - M_{ref,i}$ la diferencia entre métodos para la observación $i$. El sesgo es $\overline{d}$ y la desviación estándar de las diferencias es:
\[
SD_d = \sqrt{\frac{1}{n-1}\sum_{i=1}^n (d_i - \overline{d})^2}.
\]
Los límites de concordancia al 95\% se definen como:
\[
LoA_{low} = \overline{d} - 1.96\cdot SD_d,\qquad
LoA_{high} = \overline{d} + 1.96\cdot SD_d,
\]
y permiten juzgar intercambiabilidad en función de si el rango es clínicamente aceptable \cite{BlandAltman1986}.\\

\subsection{Métricas de error (MAE, RMSE, MAPE)}

Para complementar Bland--Altman, se reportan:
\[
MAE = \frac{1}{n}\sum_{i=1}^n |d_i|,\qquad
RMSE = \sqrt{\frac{1}{n}\sum_{i=1}^n d_i^2},\qquad
MAPE = \frac{100}{n}\sum_{i=1}^n \left|\frac{d_i}{M_{ref,i}}\right|.
\]
Estas métricas resumen la magnitud del error; el MAPE debe interpretarse con cautela en subfases cortas, donde el denominador puede amplificar el porcentaje.\\

\subsection{Error de medición: SEM y MDC}

El \textbf{error estándar de medición} (SEM) cuantifica la imprecisión típica:
\[
SEM = SD \cdot \sqrt{1-ICC}.
\]
A partir del SEM se define el \textbf{cambio mínimo detectable} (MDC), como umbral para distinguir cambio real de variación por error:
\[
MDC_{95} = 1.96 \cdot \sqrt{2}\cdot SEM,
\]
conceptos ampliamente utilizados en medición clínica y rehabilitación \cite{Weir2005}.
\clearpage




%%%%%%%%%%%%%%%%%%%%%%%%%%%%%%%%%%%%%%%%%%%%%%%%%%%%%
%% DESARROLLO DE LA APLICACIÓN
%%%%%%%%%%%%%%%%%%%%%%%%%%%%%%%%%%%%%%%%%%%%%%%%%%%%%


\chapter{DISEÑO E IMPLEMENTACIÓN DEL SISTEMA}
\label{ch:diseno-implementacion}

Este capítulo describe el proceso de diseño e implementación del sistema desarrollado para la instrumentación de la prueba \textit{Timed Up and Go} (TUG). Se presenta el levantamiento de requerimientos orientado al cumplimiento de un nivel de madurez tecnológica TRL 6, la definición de población objetivo y escenarios de uso, y el desarrollo de la aplicación móvil Android. Finalmente, se detalla el funcionamiento del sistema en modo \textit{offline} y \textit{online}, incluyendo el flujo de ejecución de la prueba, la gestión de usuarios, el almacenamiento local y el envío autenticado de datos al servidor.

\section{Levantamiento de requerimientos orientado a TRL 6}
\label{sec:req-trl6}

Una de las causas recurrentes de problemas en proyectos tecnológicos es la falta de planificación y de un proceso de ingeniería que permita especificar, controlar y validar requerimientos, lo que incrementa riesgos como sobrecostos, baja mantenibilidad y fallas durante el desarrollo \cite{desarrollo-RUP}. En este Trabajo de Grado se adoptó la metodología RUP como guía para el levantamiento y refinamiento de requerimientos, priorizando aquellos que permiten escalar el prototipo hacia condiciones de operación relevantes.

Los niveles de madurez tecnológica (\textit{Technology Readiness Levels}, TRL) se utilizan para describir el avance de una tecnología desde su concepción hasta su despliegue. En particular, TRL 6 establece que el sistema debe contar con un prototipo piloto capaz de ejecutar las funciones necesarias, y haber sido probado de manera satisfactoria en un entorno relevante, bajo condiciones cercanas a la operación real \cite{TRL-EuroFunding}. Para el presente trabajo, esta definición implica tres compromisos principales:

\begin{enumerate}
    \item \textbf{Prototipo funcional completo:} el sistema debe integrar adquisición de señales, gestión de pacientes y pruebas, almacenamiento y envío de información, y generación de resultados a partir de datos crudos.
    \item \textbf{Pruebas de factibilidad en condiciones relevantes:} el sistema debe ser probado en un entorno de laboratorio con voluntarios, siguiendo un protocolo de uso reproducible y comparando resultados frente a un sensor de referencia (BTS GSensor) disponible en el laboratorio SERH.
    \item \textbf{Operación en condiciones reales de uso:} la solución debe permitir ejecución en escenarios con conectividad limitada, por lo cual debe operar en modo \textit{offline} con almacenamiento local y sincronización posterior, y en modo \textit{online} con autenticación y persistencia centralizada.
\end{enumerate}

A partir de estos lineamientos, se definieron requerimientos funcionales y no funcionales del sistema, tomando como referencia (i) las necesidades del laboratorio SERH, (ii) la operación esperada en escenarios de campo (con o sin conectividad) y (iii) las variables reportadas por sistemas comerciales como el BTS GSensor, con el objetivo de facilitar el contraste y la validación posterior.

\subsection{Definición de población objetivo y escenarios de prueba}
\label{subsec:poblacion-escenarios}

Para contextualizar el diseño del sistema, se realizó una entrevista al personal de fisioterapia del Servicio de Rehabilitación Humana (SERH) de la Universidad del Valle. Las principales conclusiones para el diseño del sistema fueron:

\begin{itemize}
    \item La prueba TUG se aplica de forma frecuente en población mayor, en particular a partir de los 55 años, con predominio de pacientes mujeres, coherente con el contexto clínico de evaluación de movilidad.
    \item En el SERH, las pruebas relacionadas con marcha se realizan en un ambiente controlado: superficie plana, silla estable sin reposabrazos y señalización del recorrido.
    \item En general no se realizan salidas de campo como parte del servicio regular, aunque existen contextos institucionales donde la evaluación podría extenderse fuera del laboratorio, lo cual respalda el interés en soluciones portátiles.
\end{itemize}

En este Trabajo de Grado, la validación técnica se realizó con adultos sanos mayores de edad, debido a que el objetivo principal no es el diagnóstico clínico, sino la verificación de funcionamiento y la evaluación de acuerdo de medición frente a un sensor de referencia. En consecuencia, el sistema desarrollado no pretende reemplazar al BTS GSensor, sino complementarlo como alternativa portátil para escenarios donde no se dispone de equipamiento especializado o donde se requiere captura descentralizada con posterior análisis centralizado.

\subsection{Requerimientos funcionales del sistema}
\label{subsec:req-funcionales}

Los requerimientos funcionales se definieron para cubrir la operación completa del sistema (aplicación móvil + servidor), asegurando autonomía, trazabilidad y posibilidad de validación. Se resumen a continuación:

\begin{itemize}
    \item \textbf{Gestión de pacientes y sesiones:} registrar pacientes en el dispositivo, iniciar y cerrar sesión por paciente, y asociar pruebas a un paciente específico.
    \item \textbf{Ejecución guiada del TUG:} ofrecer un flujo que incluya verificación de sensores, calibración, indicación de inicio y finalización automática de la prueba.
    \item \textbf{Adquisición de datos crudos:} capturar señales de acelerómetro y giroscopio durante la prueba, almacenarlas en formato estructurado y asociarlas a metadatos del paciente.
    \item \textbf{Almacenamiento local:} guardar pruebas por paciente en el dispositivo cuando no exista conexión, incluyendo historial y estado de sincronización.
    \item \textbf{Modo online y autenticación:} permitir inicio de sesión con credenciales, obtención de token y envío autenticado de pruebas al servidor.
    \item \textbf{Confirmación de envío y manejo de errores:} reportar al usuario el estado de envío y conservar datos localmente ante fallos de red, \textit{timeout} o errores de procesamiento.
\end{itemize}

\subsection{Requerimientos no funcionales}
\label{subsec:req-no-funcionales}

Los requerimientos no funcionales se orientaron a garantizar que el prototipo sea usable, mantenible y adecuado para TRL 6:

\begin{itemize}
    \item \textbf{Usabilidad:} interfaz amigable e intuitiva, siguiendo buenas prácticas UX/UI para minimizar errores operativos.
    \item \textbf{Operación sin conectividad:} funcionamiento \textit{offline} con almacenamiento local y sincronización posterior.
    \item \textbf{Seguridad:} autenticación para el envío de datos al servidor y protección de información sensible.
    \item \textbf{Compatibilidad:} ejecución en Android 10 o superior, asegurando soporte para sensores y librerías actuales.
    \item \textbf{Escalabilidad y mantenibilidad:} diseño modular que permita evolucionar funcionalidades y facilitar mantenimiento.
\end{itemize}

\section{Desarrollo de la aplicación móvil}
\label{sec:desarrollo-movil}

A partir de los requerimientos definidos, se implementó una aplicación Android orientada a la captura de datos del TUG y a la gestión de pacientes y pruebas. La aplicación se diseñó para operar en dos modalidades: (i) \textit{offline}, donde el registro y almacenamiento se realiza localmente; y (ii) \textit{online}, donde se habilita autenticación y sincronización con el servidor.

\subsection{Requisitos de software y compatibilidad}
\label{subsec:requisitos-software}

El prototipo previo \cite{Sistema-portable-TUG-AKuleshov} proponía compatibilidad desde Android 5.0, sin embargo, versiones antiguas son actualmente minoritarias y presentan limitaciones en seguridad y soporte. Reportes de distribución de versiones muestran una adopción mayoritaria en versiones modernas, lo cual respalda la selección de Android 10 como versión mínima para este desarrollo, buscando un equilibrio entre cobertura de dispositivos y compatibilidad con componentes actuales \cite{Android-versions-2023}.

\subsection{Diagrama funcional y navegación principal}
\label{subsec:diagrama-funcional}

Con el fin de formalizar el flujo de usuario y la operación interna, se elaboró un diagrama funcional de la aplicación móvil (Figura~\ref{fig:diagramaFuncionalMovil}). Este diagrama resume los módulos principales: inicio/registro de paciente, tutorial de uso, calibración, ejecución de prueba, almacenamiento local, historial y sincronización con servidor.

\begin{figure}[h!]
    \centering
    \includesvg[width=1.0\textwidth]{Images/movil_functional_diagram_V1.svg}
    \caption{Diagrama funcional de la aplicación móvil.}
    \label{fig:diagramaFuncionalMovil}
\end{figure}

\section{Funcionamiento de la aplicación}
\label{sec:funcionamiento-app}

\subsection{Gestión de datos en local y persistencia por paciente}
\label{subsec:gestion-local}

Un requerimiento esencial para operación en escenarios con conectividad limitada es la persistencia local de pruebas y el acceso a historial por paciente. Para ello, la aplicación implementa un mecanismo de identificación de paciente y almacenamiento local de pruebas, permitiendo que cada registro quede asociado al paciente correspondiente. Este diseño evita mezclar pruebas entre pacientes y facilita la posterior sincronización selectiva hacia el servidor.

Cuando el usuario se registra en modo \textit{offline}, se solicita información mínima para identificar al paciente y asociar las pruebas (p.\ ej., número de cédula y datos demográficos). Si el paciente ya existe en la base local, la aplicación evita duplicados y muestra directamente el historial y opciones disponibles para ese paciente.

\subsection{Modo \textit{offline}: ejecución sin conexión}
\label{subsec:offline}

Al iniciar la aplicación se muestra una pantalla de presentación (Figura~\ref{fig:SplashAutoTug}). Luego, al presionar ``Comenzar'', el sistema verifica la disponibilidad de acelerómetro y giroscopio. Si los sensores no están disponibles o presentan fallas, se notifica al usuario y se impide continuar, dado que estos sensores son indispensables para la adquisición.\\

Posteriormente, la aplicación verifica conectividad. En ausencia de conexión, se activa el modo \textit{offline}, en el cual el paciente registra sus datos y accede a la pantalla principal (Figura~\ref{fig:HomeAutoTug}). Desde allí, puede: (i) ejecutar una nueva prueba, (ii) consultar el tutorial y (iii) gestionar pruebas pendientes de envío. Adicionalmente, el usuario puede acceder al historial y configuraciones (Figuras~\ref{fig:HistoryAutoTug} y~\ref{fig:SettingsAutoTug}).

\begin{figure}[h!]
    \centering
    \begin{subfigure}{0.45\textwidth}
        \centering
        \includegraphics[width=0.75\linewidth]{Images/SplashAutoTug.png}
        \caption{Pantalla de bienvenida.}
        \label{fig:SplashAutoTug}
    \end{subfigure}
    \hspace{1cm}
    \begin{subfigure}{0.45\textwidth}
        \centering
        \includegraphics[width=0.75\linewidth]{Images/HomeViewAutoTug.png}
        \caption{Pantalla principal.}
        \label{fig:HomeAutoTug}
    \end{subfigure}
    \caption{Pantallas iniciales de la aplicación.}
    \label{fig:SplashHomeAutoTug}
\end{figure}

\begin{figure}[h!]
    \centering
    \begin{subfigure}{0.45\textwidth}
        \centering
        \includegraphics[width=0.8\linewidth]{Images/HistorialPruebasAutoTug.png}
        \caption{Historial de pruebas.}
        \label{fig:HistoryAutoTug}
    \end{subfigure}
    \hspace{1cm}
    \begin{subfigure}{0.45\textwidth}
        \centering
        \includegraphics[width=0.8\linewidth]{Images/SettingsAutoTug.png}
        \caption{Menú de configuración.}
        \label{fig:SettingsAutoTug}
    \end{subfigure}
    \caption{Historial y configuración del sistema.}
    \label{fig:HistorySettingsAutoTug}
\end{figure}

\subsection{Modo \textit{online}: autenticación y sincronización con servidor}
\label{subsec:online}

Cuando existe conectividad, la aplicación habilita el inicio de sesión \textit{online} (Figura~\ref{fig:LogInAutoTug}). Para enviar datos al servidor es obligatorio autenticarse mediante usuario y contraseña. Si las credenciales son válidas, el servidor retorna un código HTTP 200 y un token, el cual se utiliza para autorizar el envío de pruebas.\\

Tras autenticación, si el paciente no existe en la base local, la aplicación solicita el diligenciamiento de información clínica/demográfica requerida para asociar adecuadamente las pruebas (Figura~\ref{fig:FormAutoTug}). Esta decisión se justifica porque la autenticación en la plataforma central puede contener únicamente credenciales y rol, mientras que el procesamiento clínico requiere metadatos adicionales del paciente.

En caso de que el servidor no responda (p.\ ej., \textit{timeout}) o ocurra un error durante el procesamiento, la aplicación notifica al usuario y conserva los datos localmente para reintento posterior. Esta estrategia permite mantener la continuidad operativa sin pérdida de información.

\begin{figure}[h!]
    \centering
    \begin{subfigure}{0.45\textwidth}
        \centering
        \includegraphics[width=1\linewidth]{Images/InicioSesionAutoTug.png}
        \caption{Inicio de sesión \textit{online}.}
        \label{fig:LogInAutoTug}
    \end{subfigure}
    \hspace{1cm}
    \begin{subfigure}{0.45\textwidth}
        \centering
        \includegraphics[width=1\linewidth]{Images/InicioSesionOfflineAutoTug.png}
        \caption{Registro de paciente en modo local.}
        \label{fig:FormAutoTug}
    \end{subfigure}
    \caption{Pantallas de autenticación y registro.}
    \label{fig:OnlineOfflineLogInAutoTug}
\end{figure}

% IMPORTANTE: Aquí debes incluir capturas de:
% - Respuesta satisfactoria con resumen de prueba
% - Error por timeout / servidor caído
% - Error por procesamiento inválido
% Recomendación: 1 figura con 3 subfigures.

\section{Ejecución de la prueba TUG}
\label{sec:ejecucion-prueba}

Con el fin de reducir errores operativos y facilitar el uso por primera vez, la aplicación incluye un tutorial accesible desde la pantalla principal. La ejecución de la prueba se estructura en tres etapas: ubicación del dispositivo, calibración e inicio/fin automáticos.

\subsection{Ubicación del dispositivo}

El teléfono debe ubicarse en la región lumbar baja del paciente, aproximadamente a nivel de L2. Esta ubicación se seleccionó por su cercanía al centro de masa y por coherencia con la recomendación de posicionamiento del BTS GSensor, lo cual facilita el contraste posterior de mediciones.

Para asegurar fijación y reproducibilidad, se recomienda el uso de un sujetador horizontal (Figura~\ref{fig:SujetadorCelularComercial}). En este trabajo se confeccionó un sujetador alternativo a partir de un brazalete convencional, adaptado con correa para fijación lumbar (Figura~\ref{fig:SujetadorConfeccionado}).

\begin{figure}[h!]
    \centering
    \begin{subfigure}{0.7\textwidth}
        \centering
        \includegraphics[width=0.8\linewidth]{Images/sujetador_celular_horizontal.jpg}
        \caption{Riñonera deportiva horizontal.}
        \label{fig:SujetadorCelularComercial}
    \end{subfigure}
    \vspace{0.8cm}
    \begin{subfigure}{0.7\textwidth}
        \centering
        \includegraphics[width=0.8\linewidth]{Images/Cellphone_holder_sewn.jpeg}
        \caption{Sujetador confeccionado.}
        \label{fig:SujetadorConfeccionado}
    \end{subfigure}
    \caption{Sujetadores utilizados para fijación del dispositivo.}
    \label{fig:SujetadoresCelular}
\end{figure}

\subsection{Calibración}

Una vez ubicado el dispositivo, el supervisor presiona el botón de calibración. Se concede un intervalo inicial para que el paciente adopte postura correcta (sentado con espalda recta y apoyada). Posteriormente, el paciente permanece inmóvil durante un periodo fijo, durante el cual se estiman referencias de señal y se valida la alineación del dispositivo. Si la orientación es inadecuada, la aplicación solicita corrección.

\subsection{Inicio y finalización automática}

Finalizada la calibración, la aplicación emite una señal sonora que indica el inicio. El paciente ejecuta el protocolo estándar del TUG (levantarse, caminar 3 m, girar, regresar y sentarse). El final de la prueba se detecta automáticamente mediante un criterio de inactividad: si las señales del giroscopio permanecen por debajo de un umbral durante un intervalo de 5 s, la aplicación finaliza el registro y emite una señal sonora de cierre. Este mecanismo reduce dependencia de intervención manual y mejora reproducibilidad.

\section{Resumen del capítulo}

En este capítulo se describió el proceso de levantamiento de requerimientos y diseño del sistema para alcanzar TRL 6, así como la implementación de la aplicación móvil orientada a la adquisición de datos del TUG. Se detalló su funcionamiento en modo \textit{offline} y \textit{online}, la gestión de pacientes, el almacenamiento local y la sincronización con servidor mediante autenticación. Finalmente, se explicó el protocolo de ejecución de la prueba y los mecanismos de calibración y detección automática de finalización.




%%%
%%END OF DEVELOPMENT
%%%%%%%%%%%%%%%%%%%%%%%%%%%%%%%%%%%%%%%%%%%%%%%%%%%%%

\chapter{PROCESAMIENTO DE DATOS Y ARQUITECTURA DEL SERVIDOR}
\label{ch:server-processing}

Este capítulo describe el procesamiento de datos realizado en el servidor, la arquitectura de la plataforma de la marcha humana y las modificaciones introducidas en los algoritmos de segmentación temporal de la prueba \textit{Timed Up and Go} (TUG). Se detallan los cambios conceptuales y numéricos que permitieron mejorar la robustez del sistema, así como los métodos empleados para el cálculo de variables biomecánicas adicionales a partir de los datos crudos capturados por el dispositivo móvil.


\section{Adaptación de la lógica de segmentación temporal}
\label{sec:segmentation-changes}

El trabajo previo desarrollado por Pérez Kuleshova \cite{Sistema-portable-TUG-AKuleshov} estableció una base funcional para la detección de las subfases del TUG. No obstante, durante las pruebas preliminares realizadas en este Trabajo de Grado se identificaron limitaciones asociadas a la sensibilidad a la orientación del dispositivo, dependencia excesiva de umbrales fijos y uso intensivo de banderas lógicas intermedias.

Con el objetivo de alcanzar un nivel de madurez tecnológica TRL 6, fue necesario reformular la lógica de segmentación, manteniendo la estructura general del algoritmo pero introduciendo cambios sustanciales en los criterios de detección, en los umbrales numéricos y en el uso combinado de aceleración, desplazamiento integrado y velocidad angular.

Las Figuras~\ref{fig:TestConCono} y~\ref{fig:TestSinCono} ilustran ejemplos representativos de señales utilizadas para evaluar el comportamiento del algoritmo durante pruebas con y sin giro marcado.

\begin{figure}[h!]
    \centering
    \includesvg[width=1\textwidth]{Images/SignalsXYTest1}
    \caption{Señales de aceleración y desplazamiento durante una prueba sin cono.}
    \label{fig:TestConCono}
\end{figure}

\begin{figure}[h!]
    \centering
    \includesvg[width=1\linewidth]{Images/SignalsXYTest2}
    \caption{Señales de aceleración y desplazamiento durante una prueba con giro marcado.}
    \label{fig:TestSinCono}
\end{figure}

\subsection{Cambios generales de preprocesamiento}

Un primer cambio relevante fue la incorporación de una verificación automática de la orientación del dispositivo. En la versión original se asumía una colocación correcta del teléfono, lo cual generaba errores cuando el dispositivo era invertido o rotado. En la versión actual se compara la magnitud del desplazamiento integrado en los ejes horizontales; si el desplazamiento lateral resulta significativamente menor que el antero–posterior, se interpreta una inversión de ejes y se realiza un intercambio automático de componentes antes del procesamiento posterior.\\

Adicionalmente, se introdujo una separación explícita entre el final del segundo giro y el inicio de la fase parado–sentado. En la versión anterior ambas transiciones se encontraban fusionadas, lo que provocaba una sobreestimación de la duración del segundo giro. La nueva lógica define el final del giro a partir de la velocidad angular, permitiendo una segmentación temporal más precisa y clínicamente interpretable.

\subsection{Detección de la transición sentado--parado}

En la versión original, el inicio de la fase sentado–parado se detectaba mediante un umbral fijo sobre la aceleración. En la versión actual, este criterio se refuerza exigiendo simultáneamente:

\begin{itemize}
    \item Una aceleración antero–posterior superior a un umbral aumentado.
    \item Un desplazamiento integrado positivo mayor que un valor mínimo.
\end{itemize}

Este enfoque evita falsos positivos debidos a oscilaciones de alta frecuencia y garantiza que el evento corresponda a un movimiento real del cuerpo. Asimismo, la transición hacia la marcha inicial se redefine empleando un patrón más simple basado en un cruce negativo seguido de estabilización, eliminando la dependencia de picos positivos poco robustos.

\subsection{Detección de la marcha inicial}

La detección de la primera fase de marcha deja de depender de un ascenso posterior de la aceleración vertical. Se considera suficiente que el patrón completo de transición desde sentado–parado se haya completado, incluso si la señal no presenta un pico vertical pronunciado. Esta modificación mejora la robustez ante diferentes estilos de levantamiento.

\subsection{Detección del primer giro}

La lógica de detección del primer giro fue reformulada para reducir latencia y complejidad. Los umbrales de desplazamiento lateral se redujeron significativamente, permitiendo identificar el giro en una etapa más temprana. Además, se reemplazó la búsqueda retrospectiva extensa por un análisis limitado al intervalo comprendido entre el inicio de la marcha y el instante actual.

La dirección del giro se infiere directamente a partir del signo del desplazamiento lateral, eliminando la necesidad de banderas explícitas para giro izquierdo o derecho.

\subsection{Detección de la marcha de retorno}

A diferencia de la versión original, la transición entre giro y marcha de retorno no depende únicamente del desplazamiento lateral. La nueva versión exige que la velocidad angular se mantenga cercana a cero durante un intervalo continuo, representando de manera más fiel el fin real de la rotación corporal.

\subsection{Detección del segundo giro}

La detección del segundo giro incorpora uno de los cambios más significativos. Se abandona el uso de múltiples bandas de desplazamiento y se introduce el análisis de la derivada del desplazamiento lateral. El inicio del giro se reconoce cuando dicha derivada supera un umbral fijo, condicionado por el sentido del primer giro, lo que garantiza coherencia cinemática entre ambos giros.

\subsection{Transición final parado--sentado y fin de la prueba}

El inicio de la fase parado–sentado se detecta mediante un umbral positivo moderado en la aceleración antero–posterior, sin necesidad de búsquedas retrospectivas. El final de la prueba se define a partir de un patrón de desaceleración y retorno a valores cercanos a cero, empleando banderas dedicadas exclusivamente a esta fase.

\subsection{Resumen conceptual de la adaptación}

En conjunto, la versión actual del algoritmo deja de depender exclusivamente de desplazamientos integrados y cruces por umbrales altos, y pasa a apoyarse en una combinación de aceleración, desplazamiento, derivadas y velocidad angular. Esta reformulación reduce la sensibilidad al ruido, elimina estados intermedios innecesarios y mejora la fiabilidad de la segmentación temporal en condiciones reales de uso.

\section{Plataforma de la marcha humana}
\label{sec:gait-platform}

Como parte de este Trabajo de Grado se colaboró con el grupo de investigación GICI en el desarrollo de una plataforma web destinada al almacenamiento, procesamiento y análisis de datos relacionados con la marcha humana. La plataforma integra múltiples subsistemas provenientes de diferentes proyectos de investigación, entre ellos el subsistema TUG desarrollado en este trabajo.

\subsection{Arquitectura de la plataforma}

La plataforma adopta una arquitectura cliente–servidor basada en servicios web RESTful. Los usuarios autorizados acceden mediante un navegador web, mientras que los subsistemas envían datos al servidor de manera unidireccional. La autenticación se realiza mediante tokens JWT, garantizando la seguridad de la información transmitida.

\begin{figure}[H]
	\centering
    \includegraphics[width=0.8\textwidth]{Images/Gait_Platform_Infrastructure.jpeg}
    \caption{Arquitectura cliente–servidor de la plataforma de la marcha humana.}
    \label{fig:software-architecture-data-automation}
\end{figure}

Cada subsistema envía un archivo CSV que contiene tanto los datos del paciente como las señales crudas de los sensores. Una vez recibido, el servidor separa esta información y la procesa mediante microservicios específicos, permitiendo escalabilidad y modularidad.

\subsection{Limitaciones actuales}

La creación de usuarios se encuentra restringida a los administradores de la plataforma, lo que impide el registro autónomo desde la aplicación móvil. Además, el servidor implementa límites de carga, permitiendo un máximo de diez envíos diarios por usuario, lo cual responde a restricciones de infraestructura.

\section{Cálculo de variables biomecánicas}
\label{sec:biomechanical-variables}

A partir de los datos crudos capturados por el dispositivo móvil, el servidor calcula diversas variables de interés por fase del TUG. Para ello, se utilizan ventanas temporales definidas por los instantes de inicio y fin de cada subfase detectada.

\subsection{Transformación de aceleraciones al marco mundo}

Las aceleraciones del dispositivo se transformaron al marco mundo mediante una rotación Euler Z–Y–X (yaw–pitch–roll). Posteriormente se compensó la gravedad restando $9.81~\text{m/s}^2$ en el eje vertical del mundo. Las señales resultantes se filtraron mediante un filtro pasa–bajo de primer orden ($f_c = 9$ Hz) aplicado hacia adelante y hacia atrás para evitar desfase.

\subsection{Proyección a ejes clínicos}

Para alinear las aceleraciones con los ejes clínicos antero–posterior (AP) y medio–lateral (ML), se estimó un offset de yaw como el promedio de los primeros tres segundos de señal. Con este ángulo se aplicó una rotación plana en el plano XY, obteniendo las componentes $a_{AP}$, $a_{ML}$ y $a_{VT}$.

\subsection{Rangos de aceleración por fase}

Para cada fase de interés se calculó el rango pico–a–pico de aceleración como la diferencia entre el valor máximo y mínimo dentro de la ventana temporal correspondiente. Estas métricas cuantifican la intensidad del movimiento en cada eje.

\subsection{Velocidad angular de giro}

La velocidad de rotación se obtuvo a partir del giroscopio en el eje Y. Para cada giro se calcularon el valor máximo absoluto y el promedio absoluto de la velocidad angular, expresados en grados por segundo.

\subsection{Flexión y extensión del tronco}

Los parámetros de flexión y extensión se calcularon a partir del ángulo de pitch, corregido por un offset inicial. Para cada fase se obtuvo el pico de flexión (máximo), el pico de extensión (mínimo posterior) y los rangos correspondientes, proporcionando una medida del control postural durante las transiciones.

\section{Resumen del capítulo}

Este capítulo presentó la arquitectura del servidor y la reformulación del procesamiento de datos del TUG instrumentado. La adaptación de los algoritmos permitió mejorar la robustez de la segmentación temporal y habilitó el cálculo de variables biomecánicas adicionales, consolidando el sistema como una herramienta funcional y validable en condiciones reales.
 


\chapter{RESULTADOS Y VALIDACIÓN}
\label{ch:resultados-validacion}

En este capítulo se presentan las pruebas de validación realizadas para evaluar la validez por acuerdo y la confiabilidad inter-método de la aplicación móvil Android desarrollada para la prueba \textit{Timed Up and Go} (TUG), utilizando como referencia un sensor inercial comercial BTS GSensor.\\

Se analizaron $n=14$ mediciones correspondientes a 7 sujetos, cada uno con 2 ensayos. Adicionalmente, uno de los sujetos realizó una ejecución con marcha lenta simulando patología, con el objetivo de incrementar la variabilidad funcional de la muestra y observar el comportamiento del sistema ante una ejecución no estándar.\\

\section{Pruebas de validación en laboratorio}
\label{sec:pruebas-validacion}

Las pruebas se realizaron en el laboratorio del Servicio de Rehabilitación Humana (SERH) de la Universidad del Valle. Cada sujeto ejecutó el protocolo estándar del TUG utilizando simultáneamente el BTS GSensor y la aplicación móvil, ubicando el teléfono sobre el sensor comercial, fijado mediante el portacelular adaptado. Este montaje permitió adquirir señales bajo un mismo movimiento, reduciendo diferencias asociadas a variabilidad intra-sujeto entre repeticiones no simultáneas.\\

\subsection{Protocolo para la toma de datos}
\label{subsec:protocolo-validacion}

Debido a que tanto el BTS GSensor como la aplicación móvil requieren calibración previa, se modificó temporalmente el código fuente de la aplicación para permitir una sincronización operativa con el flujo de calibración del BTS.\\

El protocolo aplicado fue el siguiente. Primero se colocaron ambos dispositivos en la región lumbar baja (aprox. L2), asegurando una fijación estable. Luego, el investigador presionó el botón de calibración en la aplicación; tras ello, la aplicación otorgó 6 segundos para que el paciente se acomodara en la silla y, posteriormente, ejecutó la calibración. Una vez finalizada, la aplicación emitió una señal sonora, que indicó al encargado del SERH que podía iniciar la calibración del BTS desde el computador.\\

Cuando el BTS finalizaba su calibración, el encargado daba la instrucción al paciente para iniciar la prueba. Al finalizar la ejecución, el encargado detenía la prueba desde el software del BTS. Por su parte, la aplicación se configuró para finalizar automáticamente la captura al detectar 5 segundos de inactividad \textit{después de haber detectado el primer movimiento característico de inicio} (levantarse de la silla). Por ello, el paciente permaneció inmóvil aproximadamente 5 segundos adicionales tras finalizar la prueba, hasta que la aplicación emitió una segunda señal sonora, confirmando el fin del registro.\\

Este procedimiento prolonga el tiempo total almacenado por la aplicación al inicio (por espera de calibración del BTS), pero dicho segmento adicional no afecta el cálculo final, ya que el algoritmo de segmentación identifica el inicio real del TUG cuando se cumplen condiciones dinámicas características del levantamiento. En particular, se considera inicio de la prueba cuando el giroscopio en el eje correspondiente supera un umbral (p.\ ej., $>0.5$ rad/s) junto con un desplazamiento integrado absoluto mayor que un valor mínimo (p.\ ej., $>0.1$), garantizando que el análisis temporal posterior corresponda al evento real de inicio y no al periodo de espera.\\

\begin{figure}[H]
    \centering
    \includegraphics[width=0.8\textwidth]{Images/Cellphone_position_autoTug.jpeg}
    \caption{Colocación del celular sobre el BTS GSensor para la toma de datos durante la validación.}
    \label{fig:Phone_position_autoTug}
\end{figure}

La Figura~\ref{fig:Phone_position_autoTug} muestra la colocación simultánea del celular y del BTS GSensor en la región lumbar baja (aprox. L2). Si bien este montaje permite simultaneidad, introduce un desplazamiento espacial (offset) entre sensores que puede afectar principalmente magnitudes derivadas de aceleración y velocidad angular, aspecto que se considera en la discusión de resultados.\\

\subsection{Preparación de datos y fuentes de referencia}
\label{subsec:preparacion-datos}

Una vez realizadas las pruebas, los datos del BTS se exportaron en formato CSV y se obtuvo el reporte generado por el software del fabricante. Para la aplicación móvil, se exportaron los registros en formato CSV y se descargó el reporte generado por el servidor tras el envío de datos.\\

El análisis comparó directamente los tiempos de las subfases reportadas por ambos métodos: \textit{sit\_to\_stand}, \textit{gait1}, \textit{turn1}, \textit{gait2}, \textit{turn2} y \textit{stand\_to\_sit}, además del tiempo total. A partir de esta comparación se calcularon diferencias por fase y por ensayo, y se obtuvieron métricas de acuerdo y confiabilidad.\\

\section{Análisis de acuerdo entre métodos}
\label{sec:acuerdo}

El acuerdo entre la aplicación y el BTS GSensor se evaluó mediante el enfoque de Bland--Altman (sesgo y límites de concordancia al 95\%), complementado con métricas de error absoluto y relativo (MAE, RMSE y MAPE). Los resultados por fase se presentan en la Tabla~\ref{tab:agreement_metrics}.\\

\begin{table}[H]
\centering
\caption{Métricas de acuerdo entre la aplicación Android y el BTS GSensor para el TUG}
\label{tab:agreement_metrics}
\resizebox{\textwidth}{!}{
\begin{tabular}{lcccccccc}
\hline
\textbf{Fase} & \textbf{n} & \textbf{Sesgo (s)} & \textbf{SD$_{diff}$ (s)} & \textbf{MAE (s)} & \textbf{RMSE (s)} & \textbf{LoA$_{low}$ (s)} & \textbf{LoA$_{high}$ (s)} & \textbf{MAPE (\%)} \\
\hline
Sit-to-Stand   & 14 & 0.280 & 0.204 & 0.289 & 0.342 & -0.119 & 0.679 & 17.95 \\
Gait 1         & 14 & -0.219 & 0.406 & 0.366 & 0.448 & -1.015 & 0.576 & 15.46 \\
Turn 1         & 14 & 0.174 & 0.335 & 0.279 & 0.366 & -0.483 & 0.830 & 15.61 \\
Gait 2         & 14 & -0.349 & 0.487 & 0.504 & 0.585 & -1.304 & 0.607 & 19.39 \\
Turn 2         & 14 & 0.256 & 0.501 & 0.469 & 0.546 & -0.727 & 1.238 & 33.78 \\
Stand-to-Sit   & 14 & 0.410 & 0.331 & 0.441 & 0.519 & -0.239 & 1.059 & 25.01 \\
\textbf{Total} & \textbf{14} & \textbf{0.186} & \textbf{0.193} & \textbf{0.239} & \textbf{0.263} & \textbf{-0.192} & \textbf{0.565} & \textbf{2.02} \\
\hline
\end{tabular}
}
\end{table}

Para el \textbf{tiempo total} se observó un sesgo de $+0.186$ s (ligera sobreestimación de la aplicación). Los límites de concordancia al 95\% fueron estrechos $[-0.192,\ 0.565]$ s, y los errores MAE y RMSE se mantuvieron por debajo de $0.3$ s. El MAPE fue bajo ($2.02\%$), lo cual sugiere un comportamiento estable al comparar ambos métodos para el desenlace clínico más utilizado del TUG.\\

En contraste, las \textbf{subfases} presentaron mayor dispersión relativa, con límites de concordancia más amplios, particularmente en \textit{turn2} y \textit{stand-to-sit}, donde se alcanzaron rangos cercanos a $\pm 1$ s o superiores. En subfases de corta duración, errores absolutos de décimas de segundo se traducen en errores porcentuales elevados, lo cual explica los valores altos de MAPE en fases como \textit{turn2}.\\

\section{Confiabilidad inter-método mediante ICC}
\label{sec:icc}

La confiabilidad inter-método se evaluó con el coeficiente de correlación intraclase bajo un modelo de dos vías con acuerdo absoluto ICC(2,1). Se reportan dos análisis: (i) considerando cada ensayo como una unidad independiente (paciente--ensayo) y (ii) usando el promedio de ensayos por paciente, con el fin de observar la estabilidad de la medición al reducir variabilidad intra-sujeto.\\

\begin{table}[H]
\centering
\caption{Coeficiente de correlación intraclase ICC(2,1) entre la aplicación y el BTS GSensor (paciente--ensayo)}
\label{tab:icc_trials}
\begin{tabular}{lccc}
\hline
\textbf{Fase} & \textbf{ICC(2,1)} & \textbf{IC 95\%} & \textbf{Interpretación} \\
\hline
Sit-to-Stand & 0.373 & [-0.12,\;0.75] & Pobre \\
Gait 1       & 0.925 & [0.76,\;0.98]  & Excelente \\
Turn 1       & 0.604 & [0.15,\;0.85]  & Moderada \\
Gait 2       & 0.767 & [0.31,\;0.92]  & Buena \\
Turn 2       & 0.393 & [-0.08,\;0.74] & Pobre \\
Stand-to-Sit & 0.455 & [-0.11,\;0.80] & Pobre \\
\textbf{Total} & \textbf{0.994} & \textbf{[0.94,\;1.00]} & \textbf{Excelente} \\
\hline
\end{tabular}
\end{table}

\begin{table}[H]
\centering
\caption{Coeficiente de correlación intraclase ICC(2,1) usando el promedio de ensayos por paciente}
\label{tab:icc_patient_mean}
\begin{tabular}{lccc}
\hline
\textbf{Fase} & \textbf{ICC(2,1)} & \textbf{IC 95\%} & \textbf{Interpretación} \\
\hline
Sit-to-Stand & 0.341 & [-0.10,\;0.81] & Pobre \\
Gait 1       & 0.900 & [0.52,\;0.98]  & Buena--Excelente \\
Turn 1       & 0.572 & [-0.09,\;0.91] & Moderada \\
Gait 2       & 0.757 & [0.08,\;0.95]  & Buena \\
Turn 2       & 0.410 & [-0.23,\;0.85] & Pobre \\
Stand-to-Sit & 0.276 & [-0.16,\;0.77] & Pobre \\
\textbf{Total} & \textbf{0.993} & \textbf{[0.77,\;1.00]} & \textbf{Excelente} \\
\hline
\end{tabular}
\end{table}

Los resultados muestran \textbf{confiabilidad excelente} para el \textbf{tiempo total} del TUG (ICC(2,1)$\approx 0.99$ en ambos enfoques), lo cual indica que la aplicación reproduce de manera consistente la medición del desenlace global respecto al sensor de referencia.\\

Para las subfases, se observaron valores que varían entre pobres y buenos. Este comportamiento es coherente con la naturaleza del problema: las subfases dependen de la detección precisa de eventos temporales (inicio/fin de cada fase), de modo que pequeños desplazamientos en la segmentación producen variaciones relevantes en fases de corta duración. En muestras pequeñas, además, los intervalos de confianza del ICC tienden a ser amplios, por lo que la interpretación debe realizarse con cautela y preferiblemente acompañada de análisis de acuerdo.\\

\section{Interpretación clínica y discusión de aceptabilidad de LoA}
\label{sec:discusion-aceptabilidad}

Desde una perspectiva clínica, el desenlace más utilizado del TUG es el \textbf{tiempo total}, debido a su asociación con desempeño funcional y riesgo de caídas. En este trabajo, los límites de concordancia obtenidos para el tiempo total del TUG
($[-0.192,\;0.565]$ s) se encuentran muy por debajo de los valores de cambio mínimo detectable (MDC) reportados en la literatura para poblaciones clínicas, los cuales suelen situarse entre 2 y 3 segundos dependiendo de la patología y el contexto de evaluación \cite{Steffen2008_MDC_TUG_parkinsonism,Huang2011_MDC_TUG_PD,Ries2009_MDC_TUG_Alzheimer}.\\

Este rango es pequeño frente a cambios clínicamente relevantes reportados comúnmente para el TUG en la literatura (por ejemplo, cambios mínimos detectables suelen ser del orden de segundos en múltiples poblaciones), por lo que, para fines prácticos, el desacuerdo observado puede considerarse \textbf{clínicamente aceptable} para la medición del tiempo total.\\

En contraste, los LoA de subfases como \textit{turn2} y \textit{stand-to-sit} alcanzan valores superiores a 1 s, lo que puede ser clínicamente problemático si el objetivo es interpretar componentes específicos del movimiento. En consecuencia, el uso de la aplicación para análisis detallado por subfases debe considerarse \textbf{exploratorio} hasta mejorar la segmentación temporal y validar con una muestra más amplia y heterogénea.\\

\section{Consideraciones metodológicas y limitaciones}
\label{sec:consideraciones}

Existen factores que pueden influir en las diferencias observadas entre métodos. Primero, el periodo de muestreo difiere: el BTS GSensor opera a 100 Hz y el teléfono a 50 Hz. Considerando que la mayoría de movimientos relevantes del cuerpo humano se encuentran por debajo de 10 Hz, este muestreo es teóricamente suficiente; sin embargo, una menor frecuencia puede suavizar transiciones rápidas y afectar la detección de eventos de corta duración.\\

Segundo, la colocación simultánea introduce un \textit{offset} espacial, ya que el celular se ubicó sobre el BTS. Esto puede afectar aceleraciones y velocidades angulares y, por ende, las reglas de segmentación basadas en umbrales. Además, los dispositivos pueden no estar perfectamente alineados en el plano frontal, lo que también introduce diferencias en señales por eje.\\

Tercero, el tamaño muestral es reducido y la muestra es relativamente homogénea (adultos sanos). Esto limita la estabilidad estadística del ICC en subfases y sugiere que futuros estudios deben incorporar participantes con alteraciones reales de marcha y un mayor número de ensayos.\\

Finalmente, debido a medidas repetidas por sujeto, futuros trabajos pueden complementar Bland--Altman con aproximaciones para datos repetidos (considerando correlación intra-sujeto) y reportar intervalos de confianza de sesgo y LoA, fortaleciendo la inferencia estadística.\\

\section{Conclusiones de la validación}
\label{sec:conclusiones-validacion}

Los resultados sustentan que la aplicación Android desarrollada presenta un \textbf{alto acuerdo} y \textbf{confiabilidad excelente} frente al BTS GSensor para la medición del \textbf{tiempo total} del TUG. El sesgo fue bajo ($+0.186$ s), los límites de concordancia fueron estrechos y el ICC(2,1) se aproximó a 1, lo que respalda su uso como alternativa portátil para la medición del desenlace global.\\

Para el análisis por subfases, se observaron mayores discrepancias y confiabilidad limitada en algunas fases, lo cual sugiere que la segmentación temporal automática requiere optimización adicional antes de emplearse como herramienta clínica para interpretación fina del movimiento. En consecuencia, se recomienda reportar subfases como métricas complementarias en escenarios exploratorios y continuar el ajuste de umbrales, reglas de decisión y validación en muestras más diversas.\\



\chapter{CONCLUSIONES Y TRABAJOS FUTUROS}
\label{ch:conclusiones-futuros}

Los resultados obtenidos permiten concluir que la aplicación móvil desarrollada es una herramienta \textbf{válida por acuerdo} y \textbf{confiable} para la medición del \textbf{tiempo total} de la prueba Timed Up and Go (TUG), al mostrar un alto grado de concordancia con el sensor de referencia BTS GSensor y una confiabilidad excelente según el coeficiente de correlación intraclase (ICC).\\

En particular, el análisis de Bland--Altman para el tiempo total evidenció un sesgo pequeño y límites de concordancia estrechos, del orden de décimas de segundo. Al contrastar la magnitud de este desacuerdo con los valores de \textit{minimal detectable change} (MDC) reportados en la literatura para el TUG en diferentes poblaciones clínicas —típicamente superiores a 2 s—, se sustenta que las diferencias observadas entre la aplicación y el BTS GSensor se encuentran \textbf{por debajo del umbral de cambio clínicamente detectable} \cite{Steffen2008_MDC_TUG_parkinsonism,Huang2011_MDC_TUG_PD,Ries2009_MDC_TUG_Alzheimer}. En consecuencia, bajo el protocolo experimental empleado, dichas diferencias pueden considerarse \textbf{clínicamente poco relevantes} para escenarios habituales de uso del TUG, como la evaluación funcional transversal o el seguimiento longitudinal del desempeño global, siempre que el desenlace de interés sea el tiempo total y se mantengan condiciones comparables de colocación y ejecución.\\

No obstante, el análisis por \textbf{subfases} (sentado--parado, marchas, giros y parado--sentado) mostró una confiabilidad más limitada y un acuerdo menos estrecho, con errores relativos mayores. Este comportamiento es coherente con dos factores principales: (i) las subfases presentan duraciones más cortas, por lo que errores absolutos pequeños se traducen en errores porcentuales elevados; y (ii) la segmentación temporal automática depende de la detección precisa de eventos de inicio y fin en señales inerciales, las cuales son sensibles a variaciones intersujeto, diferencias en la ejecución motora y discrepancias entre dispositivos. En este contexto, y dado que no existen valores de MDC ampliamente establecidos para las subfases individuales del TUG, la evidencia obtenida no permite aún afirmar equivalencia clínica a nivel de componentes. Por tanto, el uso de la aplicación para interpretación clínica fina por subfases debe considerarse \textbf{exploratorio}, condicionado a mejoras adicionales en los algoritmos de segmentación y a su validación en muestras más amplias, diversas y clínicamente representativas.\\

Respecto a las variables adicionales de interés (aceleraciones por ejes, métricas de rotación y parámetros de flexión/extensión), este trabajo logró su cálculo sistemático a partir de las señales crudas y su integración dentro del microservicio del subsistema TUG, así como su almacenamiento estructurado en la plataforma. Sin embargo, dado que el BTS GSensor es un sistema propietario y no se dispone de una descripción pública de sus algoritmos de procesamiento, no es metodológicamente apropiado exigir coincidencia numérica directa variable-a-variable entre ambos sistemas sin establecer previamente equivalencias claras en cuanto a definición de ejes, filtrado, compensación de gravedad, convenciones de signo y criterios de segmentación temporal. Por esta razón, la validación realizada se centró de forma adecuada en el desenlace primario del TUG (tiempo total), utilizando el análisis por subfases y variables adicionales como información complementaria y exploratoria.\\

Finalmente, este trabajo demuestra que es factible alcanzar un nivel de madurez tecnológica cercano a TRL 6 para un sistema móvil de instrumentación del TUG, al integrar de manera coherente: (i) una aplicación Android operativa en condiciones cercanas al uso real, (ii) un backend con mecanismos de autenticación, persistencia y control de errores, y (iii) un microservicio capaz de procesar señales inerciales crudas, segmentar la prueba, calcular variables de interés clínico y devolver respuestas estructuradas. En conjunto, estos elementos sientan bases técnicas y metodológicas sólidas para escalar el sistema hacia estudios clínicos con población real y para ampliar el ecosistema de subsistemas dentro de la plataforma de análisis de la marcha humana.\\


\clearpage
\section{Trabajos futuros}

El presente Trabajo de Grado desarrolló integralmente la aplicación móvil, el flujo de autenticación, el almacenamiento local y el envío de pruebas al servidor, además del microservicio del subsistema TUG para procesar datos crudos y extraer variables de interés. A partir de los resultados y de las limitaciones identificadas, se proponen las siguientes líneas de trabajo futuro.\\

\textbf{(1) Optimización de segmentación por subfases.}
Aunque se robustecieron los algoritmos heredados del trabajo previo, los resultados muestran que los principales retos se concentran en eventos cortos (giros y transiciones). Como continuación, se propone: (i) incorporar estrategias de detección de eventos más resistentes al ruido y a variaciones inter-sujeto (p.\ ej., validación por múltiples condiciones y ventanas adaptativas), (ii) complementar reglas determinísticas con enfoques supervisados entrenados con anotaciones clínicas, y (iii) evaluar algoritmos específicos para giros basados en velocidad angular y consistencia temporal.\\

\textbf{(2) Validación clínica con población patológica real.}
La inclusión de una marcha patológica simulada incrementó la variabilidad funcional, pero no reemplaza la validación con pacientes reales. Se recomienda diseñar estudios con adultos mayores y poblaciones con alteraciones de marcha (Parkinson, ECV, fragilidad, etc.), con tamaños muestrales suficientes para estimar intervalos de confianza estables en ICC y Bland--Altman y para analizar desempeño por subgrupos.\\

\textbf{(3) Estandarización y comparabilidad de variables adicionales.}
Para poder comparar aceleraciones, rotación y flexión/extensión con sistemas comerciales, es necesario definir un marco de referencia común: convenciones de ejes, filtros, compensación de gravedad, segmentación y definiciones exactas de cada métrica. Un trabajo futuro relevante es construir un protocolo de equivalencia de variables y, si es posible, contrastar contra múltiples IMUs comerciales o contra un sistema de referencia biomecánico.\\

\textbf{(4) Evolución de la plataforma y análisis longitudinal.}
La plataforma ya permite persistencia y consulta; como evolución natural se propone implementar visualización avanzada, reportes longitudinales por paciente, auditoría de envíos, control de roles más granular y herramientas de exploración de datos para investigación clínica (tendencias, cohortes, exportación y trazabilidad).\\

\textbf{(5) Escalabilidad e interoperabilidad.}
A nivel de despliegue, se recomienda migrar hacia infraestructura escalable (contenedores, balanceo, colas de procesamiento y almacenamiento elástico), además de considerar interoperabilidad con estándares de salud digital cuando el contexto institucional lo permita. Estas mejoras facilitarían transitar hacia TRL 7--8 con pilotos institucionales controlados y operación sostenida.\\
\clearpage






\end{justify}

%%%%%%%%%%%%%%%%%%%%%%%%%%%%%%%%%%%%%%%%%%%%%%%%%%%%%
%% BIBLIOGRAFÍA PRINCIPAL
%%%%%%%%%%%%%%%%%%%%%%%%%%%%%%%%%%%%%%%%%%%%%%%%%%%%%

\addcontentsline{toc}{section}{REFERENCIAS}
\bibliographystyle{plain}
\bibliography{bibliography}

%%%%%%%%%%%%%%%%%%%%%%%%%%%%%%%%%%%%%%%%%%%%%%%%%%%%%
%% BIBLIOGRAFÍA COMPLEMENTARIA (ejemplo)
%%%%%%%%%%%%%%%%%%%%%%%%%%%%%%%%%%%%%%%%%%%%%%%%%%%%%

\clearpage
\begin{justify}
\section*{BIBLIOGRAFÍA COMPLEMENTARIA}

    
\addcontentsline{toc}{section}{BIBLIOGRAFÍA COMPLEMENTARIA}

Strongman, C. (2020). \textit{Modern approaches to gait analysis using wearable sensors}. Journal of Biomechanics, 54(2), 110–125. \\

Muñoz, H. (2019). \textit{Tecnologías móviles aplicadas a la salud}. Editorial Alfaomega. \\

López, M. \& García, F. (2021). \textit{Sistemas portátiles para la evaluación clínica del movimiento humano}. IEEE Latin America Transactions, 19(8), 1402–1410. \\



%%%%%%%%%%%%%%%%%%%%%%%%%%%%%%%%%%%%%%%%%%%%%%%%%%%%%
%% ANEXOS
%%%%%%%%%%%%%%%%%%%%%%%%%%%%%%%%%%%%%%%%%%%%%%%%%%%%%

\appendix
\clearpage
\chapter{ANEXOS}
\label{ch:anexos}

\section{Arquitectura general del sistema}

Este anexo presenta la arquitectura general del sistema desarrollado en el presente trabajo de grado, el cual integra una aplicación móvil Android, un backend servidor y un microservicio especializado para el procesamiento de la prueba Timed Up and Go (TUG).

El sistema sigue una arquitectura cliente--servidor con enfoque modular, donde cada componente cumple una función específica y se comunica mediante interfaces bien definidas. La aplicación móvil actúa como cliente, encargándose de la adquisición de señales inerciales y del envío de datos al servidor. El backend gestiona la autenticación, el almacenamiento y la orquestación de los microservicios, mientras que el microservicio TUG se encarga del procesamiento de señales crudas y del cálculo de variables clínicas.

Esta separación de responsabilidades permite mejorar la mantenibilidad, escalabilidad y trazabilidad del sistema, además de facilitar la incorporación futura de nuevos subsistemas de análisis de la marcha humana.

% \begin{figure}[H]
%   \centering
%     \includegraphics[width=0.9\textwidth]{Images/Gait_Platform_Infrastructure.jpeg}
%     \caption{Arquitectura general del sistema de instrumentación del TUG.}
%     \label{fig:overall-system-architecture}
% \end{figure}

\section{Arquitectura del microservicio TUG}

El microservicio del subsistema Timed Up and Go (TUG) fue diseñado como un servicio independiente encargado exclusivamente del procesamiento de datos inerciales y del cálculo de variables de interés clínico.

El flujo de funcionamiento del microservicio se puede resumir en las siguientes etapas:

\begin{itemize}
    \item Autenticación del usuario mediante token JWT validado por el backend.
    \item Recepción de los datos enviados por la aplicación móvil.
    \item Separación de los datos recibidos en archivos CSV independientes:
    \begin{itemize}
        \item Señales crudas de sensores (acelerómetro, giroscopio y orientación).
        \item Ventanas temporales correspondientes a las fases del TUG.
    \end{itemize}
    \item Preprocesamiento de señales (rotaciones, compensación de gravedad y filtrado).
    \item Segmentación temporal automática de las fases del TUG.
    \item Cálculo de variables clínicas y biomecánicas.
    \item Validación de resultados y control de errores.
    \item Almacenamiento en la base de datos si los cálculos son consistentes.
    \item Envío de respuesta estructurada a la aplicación móvil, indicando éxito o error.
\end{itemize}

Este diseño permite aislar la lógica de procesamiento del resto de la plataforma y facilita su validación, depuración y evolución independiente.


\section{Marcos de referencia y ejes anatómicos}

Durante la adquisición de datos, las señales medidas por los sensores inerciales del teléfono móvil se encuentran expresadas inicialmente en el marco de referencia del dispositivo. Sin embargo, para un análisis biomecánico coherente con la literatura clínica, es necesario transformar dichas señales a un marco de referencia alineado con los ejes anatómicos del cuerpo humano.

\subsection{Marco del dispositivo}

El marco del dispositivo está definido por los ejes físicos del teléfono móvil:
\begin{itemize}
    \item Eje X: eje lateral del dispositivo.
    \item Eje Y: eje longitudinal del dispositivo.
    \item Eje Z: eje perpendicular a la pantalla.
\end{itemize}

Este marco depende directamente de la orientación en la que el dispositivo sea colocado sobre el cuerpo del sujeto.

\subsection{Marco mundo}

Para eliminar la dependencia de la orientación del dispositivo, las aceleraciones medidas se transformaron al marco mundo mediante una rotación basada en ángulos de Euler (yaw--pitch--roll). Posteriormente, se compensó la gravedad restando $9.81~\text{m/s}^2$ del eje vertical del marco mundo.

\subsection{Ejes anatómicos clínicos}

Una vez expresadas en el marco mundo, las señales se proyectaron sobre ejes clínicos de interés, comúnmente utilizados en el análisis de la marcha humana:

\begin{itemize}
    \item \textbf{Antero--posterior (AP):} asociado al avance y retroceso del cuerpo durante la marcha.
    \item \textbf{Medio--lateral (ML):} asociado a oscilaciones laterales y control del equilibrio.
    \item \textbf{Vertical (VT):} asociado a movimientos de elevación y descenso del centro de masa.
\end{itemize}

La alineación de los ejes AP y ML se realizó mediante la estimación de un offset inicial de yaw calculado a partir de los primeros segundos de la señal, asumiendo que el sujeto se encontraba en posición estática. Esta transformación garantiza que las variables calculadas sean independientes de la orientación inicial del teléfono y comparables entre sujetos y sesiones.

\section{Variables calculadas por el microservicio TUG}

El microservicio TUG calcula automáticamente un conjunto de variables clínicas y biomecánicas a partir de las señales inerciales y de la segmentación temporal de la prueba.

Las principales variables calculadas incluyen:

\begin{itemize}
    \item Tiempo total de la prueba TUG.
    \item Duración de cada subfase:
    \begin{itemize}
        \item Sentado--parado.
        \item Marcha 1.
        \item Giro 1.
        \item Marcha 2.
        \item Giro 2.
        \item Parado--sentado.
    \end{itemize}
    \item Rangos de aceleración pico--a--pico en los ejes AP, ML y VT por fase.
    \item Velocidad angular máxima y promedio durante los giros.
    \item Métricas de flexión y extensión del tronco obtenidas a partir del ángulo Pitch.
\end{itemize}

Todas las variables se calculan únicamente dentro de las ventanas temporales correspondientes a cada fase del TUG, garantizando consistencia temporal y evitando contaminación entre eventos.


\section{Protocolo experimental resumido}

Las pruebas de validación se realizaron en el laboratorio del Servicio de Rehabilitación Humana (SERH) de la Universidad del Valle, siguiendo un protocolo controlado que permitió la adquisición simultánea de datos con el sensor BTS GSensor y la aplicación móvil Android.

Los dispositivos se colocaron en la región lumbar baja (aproximadamente a la altura de la vértebra L2), asegurando una fijación estable para minimizar artefactos por movimiento relativo. Ambos sistemas fueron calibrados antes del inicio de la prueba y la ejecución siguió el protocolo estándar del Timed Up and Go.

Este procedimiento permitió comparar directamente los tiempos obtenidos por ambos sistemas bajo condiciones equivalentes.


\section{Código fuente del sistema}

El código fuente correspondiente a la aplicación móvil Android, al backend servidor y al microservicio del subsistema Timed Up and Go (TUG) desarrollado en este trabajo de grado no se incluye de forma íntegra en este documento debido a su extensión.

El código se encuentra alojado en un repositorio remoto privado en la plataforma GitHub. El acceso a dicho repositorio puede ser otorgado, previa autorización del autor y del grupo de investigación GICI, para fines académicos, de investigación o para la continuidad de futuros trabajos de grado relacionados con este proyecto.



\end{justify}



\end{document}
