\documentclass[12pt]{article}

% Idioma y codificación
\usepackage[spanish]{babel}
\usepackage[utf8]{inputenc}

% Comillas tipográficas con \enquote
\usepackage[autostyle=true]{csquotes}


% Márgenes y tamaño de página
% Superior: 3 cm, Inferior: 3 cm, Izquierdo: 3 cm, Derecho: 2 cm
\usepackage[letterpaper,top=3cm,bottom=3cm,left=3cm,right=2cm]{geometry}

% Fuente tipo Arial (aprox. Helvetica) en todo el documento
\usepackage{helvet}
\renewcommand{\familydefault}{\sfdefault}

% Paquetes útiles
\usepackage{natbib}
\usepackage{graphicx}
\usepackage{svg}        %svg files
%\usepackage[inkscapeversion=1,inkscapeexe={C:/Program Files/Inkscape/bin/inkscape.exe}]{svg}
\usepackage{float}
\usepackage{xcolor}
\usepackage{amsmath}
\usepackage[colorlinks=true, allcolors=black]{hyperref}
\usepackage{setspace}
\usepackage{tocloft}    % Para personalizar tabla de contenidos y listas
\usepackage[bottom]{footmisc}
%\usepackage{makeidx}    % Para índice analítico (ÍNDICE)
\usepackage{titlesec}   % Para formatear títulos y subtítulos
\usepackage{ragged2e}
\usepackage{tabularx}
\usepackage{booktabs}
\usepackage{subcaption} % to caption of subfigures


\usepackage{tikz}
\usetikzlibrary{arrows.meta,shapes.geometric,positioning}

\tikzset{
  block/.style={
    rectangle, draw, rounded corners,
    text width=4cm,        % ancho máximo del texto
    align=center,
    minimum height=8mm
  },
  decision/.style={
    diamond, draw, aspect=2,
    text width=4cm,        % rombos más altos y menos anchos
    align=center,
    inner sep=1pt
  },
  startstop/.style={
    ellipse, draw,
    text width=3cm,
    align=center
  },
  line/.style={draw, -{Latex[length=3mm]}}
}






\doublespacing
%\makeindex

% Cambiar título de las referencias (dejamos como lo tenías)
\AtBeginDocument{
  \renewcommand{\refname}{REFERENCIAS}
}

% ===========================
% Formato de títulos y subtítulos
% ===========================

% Secciones (títulos principales) centrados
\titleformat{\section}
  {\normalfont\bfseries\large\centering} % formato
  {\thesection}                          % número de sección
  {1em}                                  % separación número-título
  {}                                     % código antes del título

% Subsecciones (subtítulos) alineados a la izquierda
\titleformat{\subsection}
  {\normalfont\bfseries\large} % formato
  {\thesubsection}
  {1em}
  {}

% Subsubsecciones también a la izquierda
\titleformat{\subsubsection}
  {\normalfont\bfseries} % formato
  {\thesubsubsection}
  {1em}
  {}

% ===========================
% "pág." en Contenido, Lista de Tablas y Lista de Figuras
% ===========================

% Después del título CONTENIDO
\renewcommand{\cftaftertoctitle}{%
  \par\vspace{0.5em}%
  \noindent\hfill\textbf{pág.}\par\vspace{1em}%
}

% Después del título LISTA DE TABLAS
\renewcommand{\cftafterlottitle}{%
  \par\vspace{0.5em}%
  \noindent\hfill\textbf{pág.}\par\vspace{1em}%
}

% Después del título LISTA DE FIGURAS
\renewcommand{\cftafterloftitle}{%
  \par\vspace{0.5em}%
  \noindent\hfill\textbf{pág.}\par\vspace{1em}%
}

%% Recomendaciones del comite
% los títulos correspondientes al material complementario (DESDE LA BIBLIOGRAFIA) se deben escribir con mayúscula sostenida y se indica la página donde están ubicados. No se deben anteceder por numerales.

%Lista alfabética de términos y sus definiciones o explicaciones necesarios para la comprensión del documento. El GLOSARIO tiene carácter opcional y su existencia no justifica la omisión de una explicación la primera vez que aparece un término. Los términos se deben escribir con mayúscula sostenida seguidos de dos puntos y en orden alfabético. La definición correspondiente se coloca después de los dos puntos, se deja un espacio y se inicia con minúscula. Entre termino y termino se deja una interlinea. 

% El resumen debe ser de máximo 500 palabras, presentando el documento de forma abreviada y precisa, sin interpretación del contenido. La palabra resumen se escribe centrada, a 3cm del borde superior, en mayúscula sostenida. El texto debe estar separado por una línea en blanco. Al final del resumen deben aparecer las palabras claves tomadas del texto. La misma información debe repetirse en inglés, en la misma página. Para mejor comprensión de cómo hacer un resumen véase la norma ISO 214: 1976. No debe usarse más de una página para el resumen.

% Presentación del documento. Debe decirse brevemente por qué este es importante, antecedentes, objetivos, metodología y aplicación en el área de conocimiento. No debe confundirse con el resumen, ni contener un recuento detallado de la teoría, tampoco anticipar conclusiones y recomendaciones. 

% En ningún caso debe ser una repetición del Anteproyecto. Mientras el anteproyecto se escribe en futuro (En este proyecto se pretende desarrollar…), la introducción debe redactarse en pasado (En este proyecto se desarrolló…).

% El número correspondiente al primer nivel debe llevar punto final. Los títulos de primer nivel de los capítulos se escriben con mayúscula sostenida, centrados, al borde del margen y precedido por el numeral correspondiente. El TITULO, no lleva punto final y el texto y/o contenido del capítulo correspondiente será separado por dos interlineas y/o espacios.


% Los títulos de segundo nivel (Subcapítulos) se deben escribir con mayúscula al margen izquierdo; no deben llevar punto final y se deben presentar a dos espacios del numeral, separados del texto o contenido por dos interlineas y/o espacios. Entre los números que designan la subdivisión debe escribirse un punto, sin embargo, después del número que designa el ultimo nivel NO SE ESCRIBE PUNTO. 

% Del tercer nivel en adelante, los títulos se deben escribir con mayúscula inicial y punto seguido. El texto debe continuar en el mismo renglón, dejando un espacio, después del punto seguido. 


% ACLARATORIAS: 

% ¡NO SE DEBEN DEJAR TITULOS A FINAL DE LA PÁGINA, SIN TEXTO!

% De la quinta subdivisión en adelante, cada nueva división o ítem puede ser señalada con viñetas, conservando el mismo estilo de ésta, a lo largo de todo el documento. Las subdivisiones, las viñetas y sus textos acompañantes deben presentarse sin sangría y justificados.

% Sobre las ilustraciones (tablas, cuadros, figuras y otros…)

% NO emplear la abreviatura de número ni el signo #. 

% La fuente documental se debe escribir al pie de la ilustración y no como pie de página. Si es elaboración propia entonces escribir: “Fuente: elaboración propia” O “Fuente: elaboración propia, con base en (cite la fuente)” 

% El nombre de cada ilustración debe escribirse en la parte superior y al margen izquierdo de la misma, después de la palabra “Tabla 1.” O “Cuadro 2.” O etcétera. Se deben utilizar números arábigos y en orden consecutivo a lo largo del texto. Deben llevar un título breve sobre su contenido.

% Si la figura ocupa más de una página, se debe repetir su identificación numérica seguida por la palabra: “Continuación”, con mayúscula inicial y entre paréntesis. Del mismo modo, en este caso, los encabezados de las columnas se deben repetir en todas las páginas después de la primera.

% Conclusiones: Se presentan de forma lógica, los resultados del trabajo. Las conclusiones deben ser la respuesta a los objetivos o propósitos planteados. El autor debe sintetizar todo lo expuesto. 
% Este capítulo independiente debe titularse con la palabra: CONCLUSIONES, escrita en mayúscula sostenida, centrada, al borde del margen superior, precedida por el numeral correspondiente y separada del contenido por dos espacios y/o interlineas. Cuando se requiera diferenciar cada una de las conclusiones, se recomienda usar viñetas. 

% RECOMENDACIONES: De carácter opcional, se debe titular con la palabra: RECOMENDACIONES, en mayúscula sostenida, centrada, al borde superior del margen, precedida por el numeral correspondiente y separada del texto por dos interlineas. 
% En este capítulo, se escriben las sugerencias, proyecciones o alternativas que se presentan para modificar, cambiar o incidir sobre una situación específica o una problemática encontrada. 

\begin{document}

%%%%%%%%%%%%%%%%%%%%%%%%%%%%%%%%%%%%%%%%%%%%%%%%%%%%%
%% PORTADA
%%%%%%%%%%%%%%%%%%%%%%%%%%%%%%%%%%%%%%%%%%%%%%%%%%%%%

\begin{titlepage}
	\centering
    \linespread{1.2}
	{\large \textbf{APLICACIÓN PARA LA MONITORIZACIÓN DE LA PRUEBA TIMED UP AND GO, CON UN TELÉFONO INTELIGENTE, GESTIÓN DE INFORMACIÓN EN LA NUBE Y NIVEL DE MADUREZ TECNOLÓGICA 6} \par}
	\vfill
	{\large John Sebastian Chamorro Narváez \par}
	\vfill
	{\large Universidad del Valle\par}
	{\large Facultad de Ingeniería \par}
	{\large Escuela de Ingeniería Eléctrica y Electrónica \par}
	{\large Santiago de Cali \par}
	{\large 2025 \par}
\end{titlepage}

%%%%%%%%%%%%%%%%%%%%%%%%%%%%%%%%%%%%%%%%%%%%%%%%%%%%%
%% CONTRA PORTADA
%%%%%%%%%%%%%%%%%%%%%%%%%%%%%%%%%%%%%%%%%%%%%%%%%%%%%

\newpage
\thispagestyle{empty}
\begin{center}
    \linespread{1.2}
	{\large \textbf{APLICACIÓN PARA LA MONITORIZACIÓN DE LA PRUEBA TIMED UP AND GO, CON UN TELÉFONO INTELIGENTE, GESTIÓN DE INFORMACIÓN EN LA NUBE Y NIVEL DE MADUREZ TECNOLÓGICA 6} \par}
	\vfill
	{\large John Sebastian Chamorro Narváez \par}
	\vfill
	{\large Trabajo de grado para optar por el título de: \par}
	{\large Ingeniero Electrónico \par}
	\vfill
	{\large Directores: \par}
	{\large Dr.-Ing. Esteban Rosero \par}
	{\large José Miguel Ramírez Scarpetta, Ph.D. \par}
	\vfill
	{\large Grupo de Investigación en Control Industrial (GICI) \par}
	\vfill
	{\large Universidad del Valle \par}
	{\large Facultad de Ingeniería \par}
	{\large Escuela de Ingeniería Eléctrica y Electrónica \par}
	{\large Santiago de Cali \par}
	{\large 2025 \par}
\end{center}

%%%%%%%%%%%%%%%%%%%%%%%%%%%%%%%%%%%%%%%%%%%%%%%%%%%%%
%% NOTA DE ACEPTACIÓN
%%%%%%%%%%%%%%%%%%%%%%%%%%%%%%%%%%%%%%%%%%%%%%%%%%%%%

\newpage
\thispagestyle{empty}

\noindent\textbf{Nota de aceptación:} 

\vspace{3cm}

\rule{\textwidth}{0.4pt}

\vspace{0.8cm}

\rule{\textwidth}{0.4pt}

\vspace{0.8cm}

\rule{\textwidth}{0.4pt}

\vspace{0.8cm}

\rule{\textwidth}{0.4pt}

\vspace{0.8cm}

\rule{\textwidth}{0.4pt}

\vspace{3cm}

\noindent\begin{minipage}[t]{0.45\textwidth}
\centering
\rule{6cm}{0.4pt}\\
Firma del director del trabajo
\end{minipage}
\hfill
\begin{minipage}[t]{0.45\textwidth}
\centering
\rule{6cm}{0.4pt}\\
Firma del evaluador
\end{minipage}

\vspace{3cm}

\noindent\centering
\rule{6cm}{0.4pt}\\
Firma del evaluador

\vfill

\noindent Santiago de Cali, \rule{3cm}{0.4pt}.

%%%%%%%%%%%%%%%%%%%%%%%%%%%%%%%%%%%%%%%%%%%%%%%%%%%%%
%% DEDICATORIA (opcional)
%%%%%%%%%%%%%%%%%%%%%%%%%%%%%%%%%%%%%%%%%%%%%%%%%%%%%

\newpage
\thispagestyle{empty}
\begin{center}
\textbf{DEDICATORIA}
\end{center}

\vspace{1cm}

% Escribe aquí tu dedicatoria.

%%%%%%%%%%%%%%%%%%%%%%%%%%%%%%%%%%%%%%%%%%%%%%%%%%%%%
%% AGRADECIMIENTOS (opcional)
%%%%%%%%%%%%%%%%%%%%%%%%%%%%%%%%%%%%%%%%%%%%%%%%%%%%%

\newpage
\thispagestyle{empty}
\begin{center}
\textbf{AGRADECIMIENTOS}
\end{center}

\vspace{1cm}

% Escribe aquí tus agradecimientos.

%%%%%%%%%%%%%%%%%%%%%%%%%%%%%%%%%%%%%%%%%%%%%%%%%%%%%
%% CONTENIDO (TABLA DE CONTENIDOS)
%%%%%%%%%%%%%%%%%%%%%%%%%%%%%%%%%%%%%%%%%%%%%%%%%%%%%

\clearpage
\pagenumbering{arabic} % numeración arábiga desde aquí

\renewcommand{\contentsname}{CONTENIDO}


\tableofcontents

%%%%%%%%%%%%%%%%%%%%%%%%%%%%%%%%%%%%%%%%%%%%%%%%%%%%%
%% LISTA DE TABLAS
%%%%%%%%%%%%%%%%%%%%%%%%%%%%%%%%%%%%%%%%%%%%%%%%%%%%%

\clearpage
\renewcommand{\listtablename}{LISTA DE TABLAS}
%\addcontentsline{toc}{section}{LISTA DE TABLAS}
\listoftables

%%%%%%%%%%%%%%%%%%%%%%%%%%%%%%%%%%%%%%%%%%%%%%%%%%%%%
%% LISTA DE FIGURAS
%%%%%%%%%%%%%%%%%%%%%%%%%%%%%%%%%%%%%%%%%%%%%%%%%%%%%

\clearpage
\renewcommand{\listfigurename}{LISTA DE FIGURAS}
%\addcontentsline{toc}{section}{LISTA DE FIGURAS}
\listoffigures

%%%%%%%%%%%%%%%%%%%%%%%%%%%%%%%%%%%%%%%%%%%%%%%%%%%%%
%% LISTA DE ANEXOS (ejemplo)
%%%%%%%%%%%%%%%%%%%%%%%%%%%%%%%%%%%%%%%%%%%%%%%%%%%%%

\clearpage
\section*{LISTA DE ANEXOS}
%\addcontentsline{toc}{section}{LISTA DE ANEXOS}

\begin{tabular}{p{12cm}r}
\textbf{Anexo A.} Encuesta aplicada a los participantes ............................. & 120 \\
\textbf{Anexo B.} Resultados estadísticos complementarios ........................... & 122 \\
\textbf{Anexo C.} Fotografías del montaje experimental .............................. & 125 \\
\end{tabular}

%%%%%%%%%%%%%%%%%%%%%%%%%%%%%%%%%%%%%%%%%%%%%%%%%%%%%
%% GLOSARIO
%%%%%%%%%%%%%%%%%%%%%%%%%%%%%%%%%%%%%%%%%%%%%%%%%%%%%

\begin{justify}

\clearpage
\section*{GLOSARIO}
%\addcontentsline{toc}{section}{GLOSARIO}
\noindent \textbf{ACELERÓMETRO:} Sensor que mide la aceleración lineal en uno o varios ejes.  
\vspace{0.3cm}


\noindent \textbf{API:} Conjunto de funciones y procedimientos que permite la comunicación entre diferentes componentes de software.
\vspace{0.3cm}

\noindent \textbf{BASE DE DATOS RELACIONAL:} Sistema que almacena datos en tablas estructuradas utilizando relaciones entre ellas.  
\vspace{0.3cm}

\noindent \textbf{SENSOR INERCIAL (IMU):} Dispositivo que combina acelerómetro y giroscopio para medir orientación y movimiento.  
\vspace{0.3cm}

\noindent \textbf{TELEMETRÍA:} Técnica que permite medir y transmitir datos a distancia mediante medios electrónicos.


%%%%%%%%%%%%%%%%%%%%%%%%%%%%%%%%%%%%%%%%%%%%%%%%%%%%%
%% RESUMEN Y SUMMARY
%%%%%%%%%%%%%%%%%%%%%%%%%%%%%%%%%%%%%%%%%%%%%%%%%%%%%

\clearpage
\section*{RESUMEN}
%\addcontentsline{toc}{section}{RESUMEN}

% Escribe aquí tu resumen.

\vspace{0.5cm}
\noindent\textbf{Palabras clave:} Timed Up and Go, teléfonos inteligentes, telemetría, TRL 6, aplicaciones móviles.

\vspace{1cm}

\section*{SUMMARY}
%\addcontentsline{toc}{section}{SUMMARY}

% Escribe aquí tu summary en inglés.

\vspace{0.5cm}
\noindent\textbf{Keywords:} Timed Up and Go, smartphones, telemetry, TRL 6, mobile applications.

%%%%%%%%%%%%%%%%%%%%%%%%%%%%%%%%%%%%%%%%%%%%%%%%%%%%%
%% CUERPO DEL TRABAJO
%%%%%%%%%%%%%%%%%%%%%%%%%%%%%%%%%%%%%%%%%%%%%%%%%%%%%

\clearpage

\section{INTRODUCCIÓN}


Las caídas se consideran un síndrome geriátrico por excelencia, siendo comunes entre las personas mayores. Cada año, estas causan lesiones graves que van desde la hospitalización del paciente hasta su muerte, siendo los adultos mayores de 60 años los más afectados \cite{falls-in-elderly}. \\

%% https://libros.usc.edu.co/index.php/usc/catalog/download/351/495/7051?inline=1
Con el objetivo de anticiparse a este problema, los profesionales de la salud han desarrollado varias pruebas para determinar el riesgo potencial de sufrir una caída. Entre estas pruebas se encuentra el \textit{Timed Up and Go}, que está relacionado con el deterioro de la salud general, la discapacidad en las actividades de la vida diaria y las caídas \cite{TimedUpAndGoTUG}. \\

%% https://lafisioterapia.net/timed-up-and-go-tug/

De esta manera, la prueba Timed Up and Go (TUG) se ha consolidado como uno de los instrumentos clínicos más utilizados para valorar equilibrio, marcha, movilidad básica y riesgo de caídas en adultos y adultos mayores. Su simplicidad —levantarse de una silla, caminar tres metros, girar, regresar y sentarse— la convierte en una prueba accesible, rápida y clínicamente validada en múltiples poblaciones y que se ha demostrado su correlación como predictor de caídas, mostrando una correlación moderada con el riesgo de caídas con una confiabilidad (ICC 0,80-0,99). No obstante, su validez y sensibilidad varían entre poblaciones, condiciones socioeconómicas e incluso género, siendo las mujeres quienes tienden a sufrir más caídas que los hombres \cite{Sensibilidad_TUG}. \\. \\

A pesar de ello, la versión tradicional del TUG presenta una limitación fundamental: la medición se realiza típicamente con un cronómetro, registrando únicamente el tiempo total de la prueba. Esto impide obtener información precisa sobre cada una de las fases del movimiento (levantarse, marcha inicial, giro medio, marcha de retorno y sentarse), y dificulta la detección temprana de alteraciones en el patrón motor del paciente. Diversos estudios han demostrado que la instrumentación de la prueba mediante sensores inerciales puede revelar variaciones sutiles que no son perceptibles a simple vista, y que tienen un alto valor diagnóstico y pronóstico \cite{Convergent-Validity-wearable-sensors}. \\


% Originalmente, la prueba del \textit{Timed Up and Go}, diseñada en 1985, tenía como objetivo evaluar el equilibrio de la persona. En 1991, se introdujo una versión cronometrada para evaluar la movilidad en adultos mayores. Desde entonces, ha sido ampliamente empleada como predictor de caídas, mostrando una correlación moderada con el riesgo de caídas y una muy buena confiabilidad (ICC 0,80-0,99). No obstante, su validez y sensibilidad varían entre poblaciones, condiciones socioeconómicas e incluso género, siendo las mujeres quienes tienden a sufrir más caídas que los hombres \cite{Sensibilidad_TUG}. \\
%% https://www.scielo.cl/scielo.php?script=sci_arttext&pid=S0034-98872021000901302#:~:text=El%20TUG%20tiene%20una%20moderada,y%20sensibilidad%20var%C3%ADan%20entre%20poblaciones.


%% CORRECTED PARAGRAPH
Por otra parte, la tecnología avanza cada vez más rápido, año tras año e incluso mes a mes, debido a la feroz competencia a nivel global entre las grandes empresas tecnológicas. Esto ha posibilitado que la sociedad tenga a disposición tecnologías más potentes a precios cada vez más bajos, como es el caso de los teléfonos celulares, que han evolucionado, incorporando características que van más allá de su propósito genérico. Hoy en día, los teléfonos inteligentes son utilizados como herramientas multidisciplinarias, desde GPS hasta asistentes personales, gracias a la amplia variedad de componentes que incluyen. Esta evolución, por lo tanto, ha facilitado que los teléfonos inteligentes sean implementados en el campo de la ciencia, aprovechando su gran potencial en el procesamiento de datos.

En los últimos años los teléfonos inteligentes han sido utilizados para poder obtener datos más precisos de pruebas, como la del \textit{Timed Up and Go}, obteniendo resultados con un Coeficiente de Correlación Intraclase (ICC) de alrededor de 0.9, demostrando así, ser una alternativa económica y de fácil acceso ante tecnologías de mayor costo para la adquisición de datos biométricos como el Vicon MX o el BTS GSensor \cite{Reliability_Accuracy_Falls_Risk}.\\
%% https://www.proquest.com/healthcomplete/docview/2843128255/8C82A19D11714AFAPQ/1?accountid=174776&sourcetype=Scholarly%20Journals

%% Quizá hablar antes del trabajo del ingeniero Arturo Pérez

En este proyecto se desarrolló una aplicación móvil el cual consistió en la continuación del trabajo de maestría del ingeniero Arturo Pérez Kuleshova, titulado ``Sistema portable de telemetría haciendo uso de teléfonos inteligentes para la caracterización de la prueba \textit{Timed Up and Go}'' \cite{Sistema-portable-TUG-AKuleshov}. En dicho trabajo se logró diseñar un prototipo capaz de recoger las señales necesarias para determinar los tiempos de duración de cada sub-fase de la prueba, por medio de una aplicación móvil Android. Dicha aplicación consistía de una interfaz sencilla pero funcional con botones interactivos que permitian al usuario comenzar y detener la prueba (Ver Figura \ref{InterfazAndroidKuleshova}), así como ver los datos en bruto de las señales y exportar los datos tomados en archivo ``.txt''. \\

Una vez recopilados los datos, estos debían ingresarse manualmente a una computadora donde serían procesados por una aplicación de escritorio desarrollada en el lenguaje de programación C\#, la cual arroajaba un informe general de la prueba con los tiempos de cada sub-fase de la prueba en cuestión, también daba la posibilidad de generar dos gráficos de las señales tomadas, mostrando las aceleraciones de los giroscopios y las señales integradas(Ver Figura \ref{InterfazPCKuleshova}). \\

De esta manera, en este trabajo se desarrolló una aplicación que reemplaza la anterior la cual cuenta con una interfaz más moderna y amigable, siguiendo las metodologías de experiencia de usuario UX/UI, envío de datos al servidor del grupo de investigación de control industrial (GICI), gestión de almacenamiento de pruebas por usuario, autenticación de usuario por medio de petición http a la API del servidor, gestión de información por paciente, gestión de sesiones por paciente (Inicio y cierre de sesión), mode de uso con y sin conexión a internet, tutorial de uso de la aplicación y ejecución de la prueba.

Así mismo, se participó activamente en el desarrollo de la plataforma web de la marcha humana desarrollada y gestionada por el grupo de investigación GICI, plataforma donde se alojarán los datos de los subsistemas desarrollados como parte de trabajos de grado relacionados con el estudio de la marcha humana. De esta manera, se desarrolló un sistema de gestión de usuario dentro de la plataforma, otorgando permisos dependiendo del rol del usuario, así mismo, se desarrolló la lógica del ``backend'' que recibe los archivos CSV de los subsistemas y el servicio que procesa los datos de las pruebas enviadas por el subsistemas Timed Up and Go, rescatando gran parte de la lógica para la detección de las fases de la prueba desarrollado por el trabajo de maestría antes mencionado. Finalmente, se desarrolló la integración de un explorador de archivos por paciente y fecha, un sistema de autenticación seguro, y un módulo de generación dinámica de reportes. \\


La solución propuesta alcanza un Nivel de Madurez Tecnológica (TRL) 6, al haber sido validada en un entorno relevante con datos reales capturados por pacientes voluntarios sanos mayores de edad y validación de datos por pruebas simultáneas con la IMU TBS GSensor disponible en el laboratorio SERH de la Universidad del Valle. Esto demuestra viabilidad técnica, estabilidad operativa y pertinencia clínica, abriendo la posibilidad de continuar su evolución hacia TRL 7–8 mediante estudios piloto y escalabilidad institucional. \\

Finalmente, este proyecto no solo ofrece una herramienta útil para la práctica clínica, sino que también contribuye al ecosistema de investigación en biomecánica y salud digital, permitiendo analizar patrones de movimiento con un nivel de resolución que tradicionalmente estaba reservado a laboratorios especializados. Con este aporte la prueba del \textit{Timed Up and Go} se podría ejecutar en zonas remotas o en simultaneo desde varios lugares, enviando los datos al servidor de manera remota donde un especialista puede ver los resultados de la prueba y dar su propio análisis de manera inmediata. De esta manera, la plataforma constituye un paso significativo hacia la integración de tecnologías accesibles en la evaluación funcional de pacientes, anticipando un futuro donde la salud digital y la instrumentación biomédica sean parte fundamental del cuidado clínico cotidiano.

%%%%% HASTA AQUÍ (OK) Excepto por lo del "prototipo final"



%% PLANTEAMIENTO DEL PROBLEMA

\subsection{PLANTEAMIENTO DEL PROBLEMA}


Las caídas se describen usualmente como ``sucesos involuntarios que hacen perder el equilibrio y dar con el cuerpo en el suelo o en otra superficie firme que lo detenga'' \cite{OMS-falls}. Según la OMS, las caídas representan la segunda causa de defunción por traumatismos involuntarios, ubicándose solo por detrás de las colisiones de tránsito. Para el año 2021, más del 80\% de las caídas mortales se registraban en países de ingresos medianos y bajos, y de estas, el 60\% ocurrieron en regiones del Pacífico Occidental y Asia Oriental, afectando principalmente a personas mayores de 60 años \cite{OMS-falls}. \\

Aunque la tasa de mortalidad es bastante alta para este problema, no es la única consecuencia cuando ocurre una caída, ya que aproximadamente 37,3 millones de caídas registradas cada año requieren hospitalización; la mayoría de estas caídas reducen la movilidad de quienes las sufren, lo que en conjunto resulta, según la OMS, en una ``pérdida total de 38 millones de años de vida ajustados en función de la discapacidad (AVAD)'' y donde el 40\% de estos años de vida perdidos corresponden a niños. Sin embargo, este porcentaje no refleja con exactitud la proporción de caídas, ya que los adultos mayores, por probabilidad, tienen menos años de vida que perder \cite{OMS-falls}. \\

%% Caer una vez, duplica las probabilidades de sufrir nuevamente otra caída

%% Muchas de las personas cuando sufren una caída temen volver a caer, esto hace que las personas reduzcan su actividad lo que genera que las personas se vuelvan más débiles y esto a su vez aumenta el riesgo de caída.

%% Maybe for justifications:
%% Investigaciones han identificado varias condiciones que contribuyen a sufrir un caída, dichas condiciones son llamadas "factores de riesgo", algunos de estos factores incluyen: debilidad en la parte inferior del cuerpo, deficiencia de vitamina D, dificultades al caminar y mantener el equilibrio, entre otras.



Con esto en mente, los profesionales de la salud utilizan la prueba del \textit{Timed Up and Go}, la cual permite valorar el riesgo de caída de una persona. Esta prueba consiste en medir el tiempo que tarda una persona en levantarse de una silla, caminar una distancia de 3 metros, girar sobre su propio eje  formando un ángulo de 180°, caminar de regreso y volver a sentarse. Estas acciones se pueden dividir en ``subfases'' o ``etapas''. Los resultados de la prueba ayudan a los fisioterapeutas a determinar el riesgo de caída que pueda tener una persona. Sin embargo, durante la prueba no solo se tiene en cuenta el tiempo total de duración, ya que otros factores como la velocidad de la marcha o el largo de la zancada influyen en las conclusiones finales que determine el profesional de la salud \cite{fragilidad-indicador}. \\



Existen muchas investigaciones acerca del uso de teléfonos móviles como herramienta que ayude a medir de manera más precisa los tiempos de las subfases durante el desarrollo de la prueba. De esta manera, se tiene información más precisa que le permite a los fisioterapeutas dar un mejor diagnóstico. El uso de un teléfono inteligente durante la ejecución de la prueba es una opción fiable y barata, ya que existen otros estudios que utilizan cámaras para detectar el movimiento de los pacientes durante la prueba, sin embargo, este método requiere de dispositivos costosos y un set completo para poder realizarla. Es por esta razón, que el uso de un celular es ideal para ayudar a los profesionales de la salud a mejorar su diagnóstico, teniendo a la mano un herramienta portátil y fiable.


En el trabajo de maestría desarrollado por el ingeniero Arturo Perez Kuleshova, se utilizaron los sensores inerciales de un teléfono inteligente para identificar cada una de las etapas en una prueba de \textit{Timed Up and Go} y poder medir sus respectivos tiempos. En este trabajo, el ingeniero logró desarrollar una aplicación tanto móvil como de escritorio para la toma y posterior procesamiento de datos en las respectivas aplicaciones (Ver figuras \ref{InterfazPCKuleshova} y \ref{InterfazAndroidKuleshova}), obteniendo resultados satisfactorios en estos objetivos. \\


\begin{figure}[H]
	\begin {center}    	
    \includegraphics[width=0.8 \textwidth]{Images/InterfazAplicativoAndroid.png}
    	\caption{Kuleshova, A. (2022). Interfaz aplicativo Android.}
    	\label{InterfazAndroidKuleshova}
	\end {center}
\end{figure}


\begin{figure}[H]
	\begin {center}    	
    \includegraphics[width=0.8 \textwidth]{Images/InterfazAplicativoPC.png}
    	\caption{Kuleshova, A. (2022). Interfaz aplicativo PC.}
    	\label{InterfazPCKuleshova}
	\end {center}
\end{figure}


Igualmente, se lograron identificar correctamente las variables biomecánicas presentes en la ejecución de la prueba. El trabajo define correctamente los filtros que permiten suprimir el ruido generado en la señal y definir de manera más clara el inicio de cada fase. Asimismo, el trabajo logra formular de manera efectiva los algoritmos necesarios para la detección autónoma y efectiva de cada etapa de la prueba, teniendo en cuenta cada posible evento que se pueda producir durante la ejecución de la misma.\\

La aplicación desarrollada tenía como nombre ``TUG'', acrónimo de la prueba \textit{Timed Up and Go}. Aunque la aplicación ofrecía  resultados fiables en la medición de los tiempos de las etapas del test, era susceptible a mejoras. La aplicación contaba con una interfaz poco amigable con el usuario, careciendo de autonomía ya que los datos debían ser transferidos manualmente al computador para que la aplicación de escritorio pudiera realizar los cálculos y no contaba con una base de datos para la persistencia de información los usuarios (Ver Figuras \ref{InterfazInicioAndroidKuleshova} y \ref{InterfazInicioPCKuleshova}). \\

\begin{figure}[H]
	\begin {center}    	
    \includegraphics[width=0.8 \textwidth]{Images/InterfazInicioAplicativoAndroid.png}
    	\caption{Kuleshova, A. (2022). Interfaz aplicativo Android Inicio.}
    	\label{InterfazInicioAndroidKuleshova}
	\end {center}
\end{figure}

\begin{figure}[H]
	\begin {center}    	
    \includegraphics[width=0.8 \textwidth]{Images/InterfazInicioAplicativoPC.png}
    	\caption{Kuleshova, A. (2022). Interfaz aplicativo PC Inicio.}
    	\label{InterfazInicioPCKuleshova}
	\end {center}
\end{figure}


%En el trabajo ``Sistema portable de telemetría haciendo uso de teléfonos inteligentes para la caracterización de la prueba \textit{Timed Up and Go}'', desarrollado por el ingeniero Arturo Perez,


% Todas estas características permiten catalogar a la aplicación en un nivel tecnológico TRL 6, dado que la aplicación ha sido probada satisfactoriamente en entornos relevantes y realizando las tareas que se llevarán acabo en aplicaciones reales. Sin embargo, aún existen puntos relevantes para que la aplicación llegue a su aplicabilidad real. El uso de la prueba \textit{Timed Up and Go} es ampliamente usado por los fisioterapeutas, sin embargo, es una prueba sencilla de realizar y lo único que se necesita es el uso de un cronómetro para medir la duración total de la prueba. Por esta razón, la aplicación debe comportarse de manera autónoma, para que exista un beneficio por parte de los fisioterapeutas a la hora de usar la aplicación. Actualmente se requiere más trabajo por parte del fisioterapeuta, ya que este deberá hacer todo manualmente, incluso el paso de los datos del celular a la aplicación de escritorio, sin contar que la interfaz gráfica es poco amigable con el usuario lo que hace que su uso sea muy dispendioso. Poco sirve tener un producto que ayude al análisis de una prueba si esta no es fácil de implementar y sobre todo, que le genere más trabajo al usuario, por el contrario se busca que con el uso de la aplicación el profesional tenga solamente puntos positivos, esto se logra automatizando la aplicación, mejorando la experiencia de usuario con la interfaz gráfica para que sea fácil de usar y sobre todo, una vez realizada la prueba, los datos se carguen automáticamente al sistema, ahorrándole tiempo al fisioterapeuta y a su vez, dándole información precisa acerca de los parámetros más relevantes durante la ejecución de la misma, como los tiempos de cada subfase y nuevas mediciones como el nivel de balanceo durante la ejecución de la prueba y el largo de la zancada, mediciones que no serían posibles determinar fácilmente con el ojo humano. 



Todas estas características permitían catalogar a la aplicación en un nivel tecnológico TRL 4, dado que la aplicación había sido probada satisfactoriamente en un entorno de laboratorio con la correcta integración de todos sus componentes, pero aún no había sido probada en entornos reales de mano de los fisioterapeutas. La prueba \textit{Timed Up and Go} es ampliamente utilizada por los fisioterapeutas; no obstante, es una prueba fácil de realizar y solo requiere el uso de un cronómetro para medir la duración total de la misma. Por esta razón, la aplicación debe operar de manera autónoma para que los fisioterapeutas obtengan beneficios al usarla. Anteriormente, se necesitaba más trabajo por parte del fisioterapeuta, ya que debía realizar todo manualmente, incluso la transferencia de datos del celular a la aplicación de escritorio. Además, la interfaz gráfica era poco amigable, lo que dificultaba su uso.\\

%% Revisar esta parte, que no suene tan agresivo
Para lograr que el profesional de la salud obtenga beneficios al usar la aplicación se debía lograr la automatización de la aplicación, mejorando la experiencia del usuario con la interfaz gráfica para que sea fácil de usar y, una vez realizada la prueba, enviando automáticamente los datos a la nube donde serán gestionados por los administradores del grupo GICI. Esto ahorra tiempo al fisioterapeuta y proporciona información precisa sobre los parámetros más relevantes durante la ejecución de la prueba.\\


Es por esto que, pensando en brindar a los fisioterapeutas de la Universidad del Valle una herramienta portátil, de fácil uso y autónoma, que era necesario escalar la aplicación ``TUG'' a un TRL 6, contando con todas las funcionalidades especificadas en el documento así como haber sido probada en condiciones reales con voluntarios como pacientes. En este contexto, surge la siguiente pregunta de investigación: ¿Cómo escalar la aplicación \textit{Timed Up and Go} a un TRL 6 en la escala tecnológica, de forma que funcione en condiciones reales y gestione la información en la nube?

% Mencionar TRL 4 del anterior proyecto (CHECKED)

%%%%%% HASTA AQUÍ (OK)



%% JUSTIFICACIÓN (OK)
\subsection{JUSTIFICACIÓN}


Las caídas han demostrado ser un problema importante de salud pública a nivel mundial, donde los más afectados y con consecuencias más graves son los adultos mayores de 60 años. Según la publicación Forensis 2020, publicación anual que compila el comportamiento de las lesiones de causa externa a través de análisis descriptivos y según variables sociodemográficas, tiempo, modo y lugar en el territorio colombiano, las caídas son la causa más común de lesiones en adultos mayores, lo que se debe principalmente a ``una menor movilidad, deterioro de la función neuromuscular y uso frecuente de medicamentos y sustancias relacionadas que aumentan el riesgo de caídas'' \cite{Forensis_2020}. \\
%% https://www.medicinalegal.gov.co/documents/20143/787115/Forensis_2020.pdf

%% Colocar aquí biblio de RTL

En el caso de los adultos mayores, la caída desde su propia altura es la principal causa de muerte por muertes violentas. Este problema no es fácil de solucionar, dado que no hay forma de prever exactamente cuándo ocurrirá una caída, debido a su naturaleza accidental. De este modo, el presente trabajo dió continuidad al trabajo desarrollado por el ingeniero Arturo Pérez, teniendo en cuenta los niveles de madurez tecnológica (TRL) establecidos por la NASA a mediados de los años 70's \cite{TRL-ayming}. Se buscará elevar el proyecto de su estado actual en lo que se considera un TRL 4 a un TRL 6. \\

La aplicación actualmente se considera en un nivel TRL 4 principalmente a su falta de autonomía y a la falta de pruebas bajo entornos reales. Los datos deben transferirse manualmente a un ordenador para su procesamiento, además, la interfaz gráfica no es muy amigable, lo que afecta su usabilidad. Además, carece de una base de datos que recopile información de los usuarios y su desempeño con la prueba, desperdiciando así una valiosa fuente de datos que podrían ser utilizados para mejorar la predictibilidad de la prueba por medio de análisis estadísticos. \\

Aunque la aplicación registra el tiempo total y las sub-fases de la prueba, desaprovecha el potencial de los dispositivos móviles para capturar datos adicionales, como el largo de la zancada y el nivel de balanceo del usuario, que serían relevantes para mejorar el diagnóstico de la prueba por parte de los fisioterapeutas. Por último, si bien la aplicación fue probada en usuarios reales, no se realizaron pruebas en laboratorios de la mano de los fisioterapeutas ni haciendo pruebas de validación con IMU's comerciales. \\

Por estas razones, es necesario llevar el proyecto a un nivel de madurez tecnológica más alto, a un TRL 6. Esto implica desarrollar una aplicación autónoma y amigable, aprovechando mejor las capacidades de los teléfonos móviles y contando con una base de datos analizable por los fisioterapeutas. Además, se realizarán pruebas en laboratorio con voluntarios sanos mayores de edad como pacientes y con la participación de profesionales de la salud, para su implementación en los laboratorios de fisioterapia de la Universidad del Valle. Finalmente, se harán pruebas de validación para comparar los resultados con el BTS GSensor. La aplicación contribuirá a aumentar la predictibilidad de la prueba \textit{Timed Up and Go} gracias a su portabilidad, facilidad de uso y autonomía. De esta manera, se espera reducir la tasa de caídas en adultos mayores diagnosticados en los laboratorios de fisioterapia de la Universidad del Valle a largo plazo. \\



% La razón por la que el proyecto se considera en un TRL:6 se debe a la falta de autonomía que tiene actualmente la aplicación al tener que pasar los datos manualmente a un computador para que la aplicación de escritorio procese todos los datos recogidos. La interfaz gráfica no es muy amigable con el usuario, haciendo que sea poco apetecible el uso de la aplicación. De igual forma, no se cuenta con una base de datos que recopile toda la información de los usuarios con los resultados de sus pruebas, por lo que se desperdicia una valiosa fuente de información que serviría para realizar análisis estadísticos que permitan mejorar la predictibilidad de la prueba en el futuro. \\

% Actualmente, la aplicación solamente mide el tiempo total de la prueba y las subfases de la misma, cosa que por cierto, es lo más relevante de la prueba, sin embargo se están desperdiciando las herramientas potenciales que tienen los dispositivos móviles hoy en día, lo que permitiría procesar otras señales que no son fácilmente visibles al ojo humano y que podrían ser captadas por medio de la prueba, señales y datos como el largo de la zancada del usuario y el nivel de balanceo que éste posee, datos relevantes que ayudarían a los fisioterapeutas a mejorar su rendimiento en el diagnóstico de la prueba. Si bien la aplicación se probó en un entorno real con usuarios de diferentes edades, no se probó en un laboratorio de la mano con los fisioterapeutas y tampoco se llevó a cabo de manera masiva ni en diferentes entornos, por lo que este proyecto buscará realizar dichas pruebas para que la aplicación pueda ser considerada funcionalmente autónoma, mantenible y escalable en el tiempo.\\

% Es por esto que surge la necesidad de llevar al proyecto a un nivel de madurez tecnológico más alto, a un TRL:8, con el ánimo de obtener una aplicación autónoma, amigable con el usuario, que aproveche todas la potencialidades que ofrecen hoy en día los teléfonos móviles, que disponga de una base de datos que pueda ser analizada por los fisioterapeutas y que sea probada en el campo de la mano de los profesionales de la salud para que pueda ser implementada en los laboratorios de fisioterapia de la universidad del valle, lo que ayudaría a mediano y largo plazo a aumentar el nivel de predictibilidad de la prueba TUG y por ende, reducir la tasa de accidentalidad por caídas en los adultos mayores en la ciudad de Cali.

%% Aún falta hablar de qué implicaciones tiene esto para los fisioterapeutas, de ahora en más "fis.", y para la población objetivo, ¿Por qué debería de usar esta proueba móvil los fis.?, ¿Se puede hacer la prueba sin necesidad de la supervisión de un fis.?
%% ¿Qué implicaciones tiene que los usuarios puedan hacer esta prueba por ejemplo, de manera remota?
%% La aplicación no podrá ser desplegada de manera comercial, por lo que sólo podrá ser utilizada por ls fisioterapeutas de la universidad del valle, sin embargo, la aplicación pueda estar interconectada y ser llevada a cabo por varios fis. en toda la región del valle, por lo que una persona encargada podría realizar la prueba en un sitio remoto y los datos sería enviados a la base de datos general donde un fis. calificado pueda hacer el diagnóstico de la prueba, lo que ayudaría a que la prueba pueda ser empleada y llegar a poblaciones que de otro modo no tendrían acceso a la misma y tener el análisis de un profesional, en otras palabras, el fis. calificado podría ayudar a diagnosticar a poblaciones remotas, lo que la app contribuiría a nivel regional al diagnositco oportuno y de calidad a poblaciones objetivo a nivel regional.

% https://www.medicinalegal.gov.co/documents/20143/49508/Muertes+Accidentales.pdf



%% OBJETIVOS GENERAL Y ESPECÍFICOS

\subsection{DEFINICIÓN DE LOS OBJETIVOS}
\subsubsection{Objetivo general}

Escalar la aplicación Timed Up and Go a un TRL 6 en la escala tecnológica, con gestión de la información en la nube.


% de forma que funcione en condiciones reales y gestione la información en la nube

% ¿Cómo *llevar *la aplicación "Timed Up and Go" a un TRL 6 en la escala tecnológica, de forma que funcione en condiciones reales y gestione la información en la nube?

% Desarrollar un sistema de monitorización de la prueba Timed Up and Go utilizando un teléfono inteligente.



% El problema debe basarse en la mejora sustancial de la aplicación y en la "captación" de nuevas variables de información tales como la velocidad a la que se levanta una persona, velocidad de giro, si existen anomalías en el caminar de las personas que permita la identificación de otros problemas (o al menos que den un primer indicio para la posterior profundización de estos).


% \href{https://www.overleaf.com/learn}{help library}, or head to our plans page to \href{https://www.overleaf.com/user/subscription/plans}{choose your plan}.

\subsubsection{Objetivos específicos}

\begin{itemize}

	\item Especificar los requerimientos a nivel de TRL 6, para el sistema de monitorización de la prueba \textit{Timed Up and Go}.
    
	\item Desarrollar la aplicación móvil \textit{Timed Up and Go}.
    
	\item Desarrollar la aplicación web con almacenamiento en la nube.
    
	\item Validar la correcta funcionalidad de la aplicación \textit{Timed Up and Go} y probarla en un ambiente cercano al real.

    
\end{itemize}



\subsection{Presentación del documento}




\section{MARCO DE REFERENCIA}

%%% ANTECEDENTES

\subsection{ANTECEDENTES}


En el presente trabajo se pretende mejorar la fiabilidad de los resultados obtenidos en la prueba \textit{Timed Up and Go} utilizando un dispositivo celular para medir los tiempos de cada una de las subfases, además de otros parámetros de interés como el grado de balanceo, la velocidad promedio de la marcha y el largo de la zancada. Para cumplir con este objetivo, primero se debe tener certeza de que un dispositivo celular tendrá el suficiente rango de precisión para medir estos parámetros. \\

En el trabajo \textit{``A Scoping Review of the Validity and Reliability of Smartphone Accelerometers When Collecting Kinematic Gait Data''} \cite{reliabilty-smartphone-kinematic}, liderado por la investigadora Clare Strongman, se consideraron más de 3056 estudios para valorar la efectividad de los teléfonos inteligentes en la recopilación de datos de marcha cinemática. El estudio concluye que el uso de teléfonos inteligentes proporciona una alternativa económica y confiable para recopilar datos cinemáticos. Además, otros estudios han demostrado que no existen diferencias significativas en cuanto a la toma de datos cinemáticos entre teléfonos celulares de gama media y alta y el Vicon MX, una reconocida cámara de captura de movimientos \cite{subtasks}. De esta manera, queda justificado el uso de celulares inteligentes con el ánimo de proporcionar a los profesionales de la salud resultados más fiables mediante la medición de parámetros en la marcha para el test \textit{Timed Up and Go}. \\

En el estudio liderado por la investigadora Clare Strongman, se menciona la importancia de elegir adecuadamente el tiempo de muestreo. Las frecuencias de captura usadas por los \textit{smartphones} varían entre 15 Hz y 100 Hz, sin embargo, según el teorema de Nyquist, se recomienda una frecuencia mínima de 24 Hz con el fin de capturar satisfactoriamente los cambios más rápidos producidos durante la marcha y que pueden ser de interés en los resultados a evaluar. Del mismo modo, se indica que no se debe seleccionar el máximo tiempo de muestreo que el dispositivo celular pueda proporcionar, ya que esto aumenta la probabilidad de que haya ruido en los datos capturados, sino que se deben muestrear los datos a diferentes frecuencias para determinar cuál es la más adecuada. \\



% La elección de la frecuencia de captura es importante para asegurar que los cambios más rápidos sean captados satisfactoriamente con una frecuencia mínima de 24 Hz debido al teorema de Nyquist, sin embargo una alta frecuencia de muestreo aumenta la probabilidad de que haya ruido en los datos capturados.
% - En la mayoría de los estudios el sitio de ubicación de los teléfonos fue en la espina lumbar (espalda baja) dado que está cerca al centro de gravedad de la persona [4,5]
% -
% - El estudio concluye que el uso de teléfonos inteligentes proporcionan una alternativa barata y confiable

\textit{``Instrumented Timed Up and Go Test (iTUG)—More Than Assessing Time to Predict Falls: A Systematic Review''} \cite{iTUG} este estudio, liderado por la investigadora Ortega Bastidas Paulina, es similar al liderado por la investigadora Clare Strongman \cite{reliabilty-smartphone-kinematic}, a diferencia que en este se analizan específicamente 40 estudios relacionados con la prueba \textit{Timed up and Go} utilizando instrumentos de medición para incrementar la predictibilidad de la prueba, a lo que ellos llaman iTUG (\textit{instrumented Timed Up and Go}). Cerca del 75\% de los estudios usaron sensores inerciales como tecnología de medición, adultos mayores sanos y adultos mayores que padecían de enfermedad de parkinson fueron las poblaciones que más se analizaron. \\

% y las poblaciones de participantes que más fueron analizadas fueron adultos mayores sanos y adultos mayores que padecían de enfermedad de parkinson.

En el estudio, la mayoría de las propuestas utilizaron un sólo sensor en la parte baja de la espalda debido a su cercanía con el centro de gravedad de las personas, esto facilita la implementación del iTUG test en el entorno clínico debido a la portabilidad y facilidad de uso que conlleva trabajar con sensores portátiles, específicamente con teléfonos inteligentes \cite{iTUG}. Además, se resalta la efectividad que conlleva segmentar los tiempos en sub fases para la identificación de problemas específicos en personas mayores con deficiencias sensoriales o de equilibrio. Finalmente, el estudio muestra la posibilidad que existe de incrementar el valor de la predictibilidad de la prueba ``\textit{Timed Up and Go}'' a través de la implementación de instrumentación dado que provee un mayor número de características y parámetros relacionados con el desempeño del paciente.\\

``Sistema portable de telemetría haciendo uso de teléfonos inteligentes para la caracterización de la prueba \textit{Timed Up and Go}'' es el título del trabajo del ingeniero Arturo Perez Kuleshova. En él, se lograron diseñar los algoritmos necesarios para la caracterización de las sub-etapas de la prueba \textit{Timed Up and Go}, además, se diseñó la aplicación móvil y de escritorio, implementando una HMI (Human Machine Interface) para poder obtener los datos de los usuarios que realizaron la prueba. Las señales que tomó el celular implementado en este proyecto tuvieron que ser filtradas, dado que presentaban mucho ruído, luego de la implementación de los filtros y algoritmos necesarios, se logró finalmente segmentar cada una de las sub-etapas y los tiempos que al paciente le tomaba realizarlas. El trabajo concluye invitando a desarrollar mejoras en el procesamiento de datos, almacenamiento en la nube e interfaz gráfica y que de esta manera, pueda ser implementado en un laboratorio real \cite{Sistema-portable-TUG-AKuleshov}. \\

Estudios recientes han utilizado teléfonos inteligentes para sensar parámetros de las subfases durante la prueba utilizando estos resultados y analizando patrones de marcha para poder predecir la aparición de enfermedades tempranas como Parkinson y demencia \cite{gait-pattern-PD} \cite{tewnty-step-PD-detenction}. Para el alcance del proyecto no se incurrirá en el diagnóstico de las señales obtenidas, dado que el objetivo radica en ofrecer a los profesionales de la salud una herramienta que les permita brindar predicciones más confiables por medio del análisis de resultados más fiables y precisos utilizando tecnologías de fácil acceso. \\

%% hasta aquí (ok)

\subsection{Niveles de madurez tecnológica (TRL)}

Los niveles de madurez tecnológica, del inglés \textit{Technology Readiness Levels} (TRL), es un concepto creado por la NASA a mediados de los años 70 con el objetivo de saber lo lejos que estaba una tecnología de ser desplegada en el espacio. El concepto sería adoptado más adelante por el departamento de defensa de los estados unidos, agencias gubernamentales, militares y la agencia espacial europea \cite{TRL-EuroFunding}. Hoy en día es utilizada en proyectos tecnológicos para saber en qué etapa se encuentra un proyecto tecnológico y qué necesita para que llegue al mercado.

El TRL consta de nueve niveles que van desde las primeras ideas que se tienen del funcionamiento de un proyecto hasta las pruebas y certificaciones necesarias para que el producto terminado pueda ser lanzado al mercado. Con base a esto se da una breve descripción de lo que se espera de un proyecto por cada nivel: \footnote{La descripción de los niveles de madurez tecnológica se ha tomado de la página euro-funding.com}\\

\noindent \textbf{TRL 1.} Es el nivel de madurez más bajo de un proyecto de innovación. Aquí es donde comienza la idea de investigación científica básica y se inicia la transición a la investigación o idea aplicada.

\noindent \textbf{TRL 2.} La idea o investigación ya se ha aterrizado y los principios científicos están enfocados en áreas específicas de aplicación para definir el concepto.

\noindent \textbf{TRL 3.} En este nivel se realizan las actividades de investigación y desarrollo incluyendo pruebas analíticas, pruebas de concepto a escala de laboratorio, orientadas a demostrar la factibilidad técnica de los proyectos de innovación.

\noindent \textbf{TRL 4.} Los componentes que integran determinado proyecto de innovación han sido identificados y se busca establecer si dichos componentes individuales cuentan con las capacidades para actuar de manera integrada, funcionando conjuntamente en un sistema. En esta etapa se espera un prototipo a nivel banco de laboratorio, en donde se pueda medir con algún grado de seguridad. Dicho prototipo debe ser escalable y cuyas ventajas competitivas y comparativas puedan ser medibles.

\noindent \textbf{TRL 5.} Los elementos básicos de la innovación son integrados de manera que la configuración final es similar a su aplicación final, es decir que está listo para ser usado en la simulación de un entorno real. Se mejoran los modelos tanto técnicos como económicos del diseño inicial, se ha identificado adicionalmente aspectos de seguridad, limitaciones ambiéntales y/o regulatorios entre otros.

\noindent \textbf{TRL 6.} Se cuenta con prototipos piloto capaces de desarrollar todas las funciones necesarias dentro de un sistema determinado, habiendo superado pruebas de factibilidad en condiciones de operación o funcionamiento real. Además, se espera que la tecnología o prototipo pueda ser capaz de funcionar en las condiciones reales en las que se pretende este funcione, por ejemplo, a nivel industrial.

\noindent \textbf{TRL 7.} El sistema se encuentra o está próximo a operar en escala pre-comercial. Es posible llevar a cabo la fase de identificación de aspectos relacionados con la fabricación, la evaluación del ciclo de vida, y la evaluación económica de las tecnologías, contando con la mayor parte de funciones disponibles para pruebas.

\noindent \textbf{TRL 8.} Los sistemas están integrados, han sido probados en su forma final y bajo condiciones supuestas, habiendo alcanzado en muchos casos, el final del desarrollo del sistema.

\noindent \textbf{TRL 9.} En esta fase la innovación está en su fase final y es operable en un sin número de condiciones operativas, está probada y disponible para su comercialización y/o producción disponible para la sociedad. Entrega de producto o tecnología para producción en serie y comercialización.




\subsection{Variables cinéticas}
Dado que el presente trabajo tiene la intención de desarrollar el prototipo final que previamente desarrolló en su trabajo de maestría el ingeniero A. Pérez Kuleshova \cite{Sistema-portable-TUG-AKuleshov}, se toman como referencia las variables cinéticas y cinemáticas: 
\begin{itemize}
	\item El valor de la fuerza de agarre no está relacionado con el riesgo de caídas, mientras
que una disminución de ésta en el tiempo sí lo está.
	\item Menor fuerza de extensión de la rodilla está relacionada con un mayor riesgo.
	\item Una mayor oscilación postural está relacionada con mayor riesgo de caídas.
	\item La dimensión fractal de la marcha permite discriminar personas que se caen de las
que no, siendo la dimensión fractal una medida numérica adimensional que expresa el
grado de irregularidad, en este caso, de la marcha.

%% Fuente: Trabajo del ing. Arturo Pérez
\end{itemize}

\subsection{Variables cinemáticas}
Como se mencionó en el anterior apartado, se toman como referencias las mismas variables cinemáticas implementadas en el marco teórico del trabajo del ingeniero A. Pérez Kuleshova:
\begin{itemize}
	\item Una mayor variabilidad del paso de la marcha está relacionada con mayor riesgo de
caídas.
	\item Un menor número de pasos por paseo está relacionado con mayor riesgo de caídas.
	\item La velocidad de la marcha es similar en personas mayores que se caen y las que no.
	\item En cuanto a la duración de la marcha, una mayor exposición a paseos cortos está
relacionada con mayor riesgo de caídas.
	\item Con la repetición, el margen de estabilidad antero-posterior aumenta por aprendizaje.
	\item Un mayor tiempo de reacción al pisar o pulsar con un dedo está relacionado con un
mayor riesgo de caídas.
	\item Las caídas pueden distinguirse de actividades de la vida diaria mediante el registro de
la señal de aceleración y utilizando un único umbral determinado por los datos
obtenidos de las caídas.

%% -> fuente: Trabajo del ing. Arturo Pérez
\end{itemize}

\subsection{Timed Up and Go}
La prueba del \textit{Timed Up and Go} es ampliamente usada para evaluar la movilidad y el riesgo que tiene una persona de sufrir una caída. Para comenzar la prueba primero se necesita una silla sin reposabrazos en donde va a estar sentado el paciente, cuando se le indique, el paciente deberá levantarse de la silla sin ayudarse de las manos, una vez de pie, debe caminar hacia adelante una distancia de tres metros, donde se puede ubicar un cono que sirve como referencia para indicar el final del recorrido hacia adelante. Una vez llegue al punto de referencia, debe girar sobre su mismo eje y devolverse nuevamente a la silla, girar 180 grados sobre su eje craneocaudal, para finalmente sentarse de nuevo sobre la silla \cite{Convergent-Validity-wearable-sensors}. \\
%% https://www.mdpi.com/1424-8220/17/4/934

Aunque la prueba del \textit{Timed Up and Go} solo consiste en medir el tiempo total de la prueba, no significa que la medición de otros parámetros en las subtareas de la prueba no sean de utilidad y relevancia en el análisis de resultados. La medición de los tiempos de las subtareas puede ayudar a identificar si el tiempo total que tuvo el paciente se debe a una demora en todos las sub fases de la prueba o solamente en algunas en particular \cite{Association-Between-Performance-on-TUG}.
%% https://agsjournals.onlinelibrary.wiley.com/doi/10.1111/jgs.12734
En la figura \ref{subEtapasTug} se muestra el proceso de la prueba \textit{Timed Up and Go}.

\begin{figure}[H]
	\begin {center}    	\includegraphics[width=0.9\textwidth]{Images/Esquema_susFases_TUG.png}
    	\caption{Esquema de las subetapas de la prueba \textit{Timed Up and Go} \cite{Reproducibilidad-TUG}.}
    	\label{subEtapasTug}
	\end {center}
\end{figure}

\subsection{Planos y ejes del cuerpo humano}
Son líneas de referencia que se usan para dividir el cuerpo humano para su estudio. Se pueden identificar 3 planos específicos en los que se pueden clasificar los diferentes movimientos articulares y todos los movimientos anatómicos se realizan alrededor de estos ejes \cite{Planos-ejes-movimiento}.

\subsubsection{Plano sagital}

%% PLANO SAGITAL

\begin{figure}[H]
	\begin {center}	\includegraphics[width=0.4\textwidth]{Images/PlanoSagital.png}
    	\caption{Plano sagital del cuerpo humano.}
    	\label{PlanoSagital}
	\end {center}
\end{figure}

% \begin{figure}[H]
% 	\begin {center}    	\includegraphics[width=0.9\textwidth]{PlanoSagital.png}
%     	\caption{}
%     	\label{PlanoSagital}
% 	\end {center}
% \end{figure}

El plano sagital secciona el cuerpo de adelante hacia atrás. Divide el cuerpo en dos mitades, derecha e izquierda. Como se aprecia en la figura \ref{PlanoSagital}.

\subsubsection{Plano frontal}
%% PLANO FRONTAL
\begin{figure}[H]
	\begin {center}    	\includegraphics[width=0.4\textwidth]{Images/PlanoFrontal.png}
    	\caption{Plano frontal del cuerpo humano.}
    	\label{PlanoFrontal}
	\end {center}
\end{figure}
%% imagenes tomadas de:
%% https://mundoentrenamiento.com/planos-y-ejes-de-movimiento/

El plano frontal secciona el cuerpo lateralmente de lado a lado, dividiéndolo en dos. Una anterior y otra posterior. Como se ve en la figura \ref{PlanoFrontal}.



%% información sobre los ejes tomadas de
%% http://ri.uaemex.mx/bitstream/handle/20.500.11799/103120/secme-34635_1.pdf?sequence=1&isAllowed=y

\subsubsection{Plano transversal}

%% PLANO TRANSVERSAL
\begin{figure}[H]
	\begin {center}    	\includegraphics[width=0.4\textwidth]{Images/PlanoTransversal.png}
    	\caption{Plano transversal del cuerpo humano.}
    	\label{PlanoTransversal}
	\end {center}
\end{figure}
El plano transversal es aquel que divide el cuerpo en dos partes sin simetría: una superior o craneal y otra inferior o caudal. La figura \ref{PlanoTransversal} muestra este plano. \footnote{Ilustraciones de los planos sagital, frontal y transversal tomadas de la página mundoentrenamiento.com}

\subsection{Telemetría y teléfonos inteligentes}
La telemetría es un sistema de comunicación a distancia que permite recoger, procesar y transmitir información de un dispositivo electrónico a otro. Esto es posible gracias a la utilización de sensores y transductores que permiten enviar, recibir y almacenar la información entre distintos dispositivos electrónicos \cite{Telemetría}. \\
%% https://landing.sitrack.com/telemetria-y-sus-aplicaciones#:~:text=La%20telemetr%C3%ADa%20es%20un%20sistema,velocidad%2C%20tiempo%2C%20etc.)

En un sistema de telemetría se pueden encontrar características como acelerómetros, GPS y sensores.

Debido al gran avance de la tecnología y la competencia del mercado, los teléfonos inteligentes son dispositivos que hoy en día cuentan con este tipo de tecnologías equipadas internamente, en donde, además de los ya mencionados, se pueden encontrar giroscopios, magnetómetros, sensores de proximidad o lector de huellas. Además, estudios han demostrado que estas tecnologías embebidas pueden ser utilizadas para medir señales de movimiento que van desde propósitos comerciales hasta clínicos \cite{reliabilty-smartphone-kinematic}.
%% nombre del artículo:
%% A Scoping Review of the Validity and Reliability of Smartphone Accelerometers When Collecting Kinematic Gait Data
%% ProQuest
%% link:
%% https://www.proquest.com/healthcomplete/docview/2882819696/AB504CC06F174D7APQ/8?accountid=174776&sourcetype=Scholarly%20Journals

\subsection{Protocolos de seguridad informática y web}
Los protocolos de seguridad informática o \textit{``cyber security''}, son diseñados para mantener segura la información e integridad informática de una organización, aplicación o proceso en la nube. Para garantizar que una organización mantenga protegidos sus datos, se deben implementar varios protocolos y programas de software que trabajen bien en conjunto. De esta manera existen varias protocolos que se deben seguir para mantener segura la información de una organización o aplicación, tales como la implementación de \textit{firewalls}, encriptamiento de datos y educación en seguridad informática para los administradores de la aplicación con el ánimo de evitar ataques como el \textit{phising} o el \textit{malware} \cite{Cyber-security-protocols}. \\

%% https://www.logsign.com/blog/cyber-security-protocols-that-you-should-know/

\subsection{Arquitectura de software}
La arquitectura de software son un conjunto de artefactos que contiene principios, directrices, políticas, modelos, estándares y procesos, relacionándolos entre sí. La arquitectura de software es la estructura de todas las demás subestructuras de un sistema de información que consta de entidades y sus propiedades visibles externamente, y las relaciones que existen entre ellas \cite{Enterprise_architecture-frameworks}. \\
%% https://cs.nyu.edu/~jcf/classes/g22.3033-007/slides/session2/g22_3033_011_c23.pdf

La arquitectura de software ayuda a los colaboradores y a futuros colaboradores de un proyecto informático a entender mejor el comportamiento estructural del mismo. De esta manera, se tiene un entendimiento general del sistema, lo que permite encontrar vulnerabilidades, posibles mejores procedimentales y volverlo escalable en el largo plazo. Una efectiva arquitectura de software permite una evolución continua del sistema, lo que permite depurar el sistema y realizar mantenimiento de manera eficaz cuando el software lo requiera. Algunas de estas prácticas incluyen la documentación de los elementos de la arquitectura y las relaciones que tienen entre sí, evaluación constante de la arquitectura buscando siempre que encaje con los intereses de la organización y analizar el sistema en busca de posibles errores o implementación de procesos más óptimos. Siguiendo estos lineamientos se aumenta la predictibilidad de la calidad del producto final, se ahorra tiempo y dinero además de garantizar la evolución rentable del sistema \cite{Software-architecture}. \\

%% https://www.sei.cmu.edu/our-work/software-architecture/


\subsection{Interfaces Humano Máquina (HMI)}
HMI, del inglés \textit{Human Machine Interface} se refieren a un panel que permite a un usuario comunicarse con una máquina, software o sistema.
Técnicamente, se puede referir a cualquier pantalla que tenga interacción con una persona, aunque se utiliza normalmente en la industria. De esta manera, son interfaces que permiten al usuario interactuar y controlar a una máquina o programa de software para que realice o procese ciertas señales o datos de interés \cite{HMI}.
%% https://www.copadata.com/es/productos/zenon-software-platform/visualizacion-control/que-significa-hmi-interfaz-humano-maquina-copa-data/#:~:text=HMI%20son%20las%20siglas%20de,para%20las%20de%20entornos%20industriales.


%%%%%%%%%%%%%%%%%%%%%%%%%%%%%%%%%%%%%%%%%%%%%%%%%%%%%
%% DESARROLLO DE LA APLICACIÓN
%%%%%%%%%%%%%%%%%%%%%%%%%%%%%%%%%%%%%%%%%%%%%%%%%%%%%


\section{LEVANTAMIENTO DE REQUERIMIENTOS FUNCIONALES}

\subsection{Requerimientos funcionales a nivel de TRL 6}\label{sec:1.TRL-6-requirements}

Una de las principales causas de fracaso en los proyectos tecnológicos se debe a la falta de planificación o el uso de una correcta metodología, lo que provoca un aumento en los costos previstos, hace que el proyecto sea poco mantenible, escalable y propicia la aparición de fallas y errores en el código que se desarrolla \cite{desarrollo-RUP}. Por esta razón gran parte de los proyectos tecnológicos que se llevan a cabo a gran escala utilizan estrategias de organización tales como las metodologías ágiles o la metodología RUP.

Para asegurar el correcto desarrollo del proyecto se hará uso de la metodología RUP, comenzando por evaluar y definir los requerimientos necesarios para alcanzar el nivel de madurez tecnológica (TRL) 6.

Los niveles de madurez tecnológica, del inglés \textit{Technology Readiness Levels} (TRL), son un concepto creado por la NASA a mediados de los años 70 para determinar qué tan lejos estaba una tecnología de ser desplegada al espacio, poco a poco se fue adoptando en la industria militar y comercial. Hoy en día es utilizada en proyectos tecnológicos para determinar en que etapa se encuentra un proyecto antes de salir al mercado.

Requerimientos para alcanzar un nivel 6 en la escala de madurez tecnológica:

\textbf{TRL 6.} Se cuenta con prototipos piloto capaces de desarrollar todas las funciones necesarias dentro de un sistema determinado, habiendo superado pruebas de factibilidad en condiciones de operación o funcionamiento real. Además, se espera que la tecnología o prototipo pueda ser capaz de funcionar en las condiciones reales en las que se pretende este funcione, por ejemplo, a nivel industrial \cite{TRL-EuroFunding}.

Con el ánimo de abarcar los requerimientos del TRL 6, se desglosará en sus partes fundamentales, dando una breve explicación del trabajo que se llevará a cabo para poder suplir estas condiciones.

En la primera parte de los requerimientos se especifica que se debe contar con ``prototipos piloto capaces de desarrollar las funciones necesarias dentro de un sistema determinado''. De esta manera, se especifican los requerimientos funcionales y no funcionales del proyecto junto con los requerimientos sugeridos por los fisioterapeutas de la Universidad del Valle del laboratorio SERH en la \autoref{subc:objective-population-segmentation}, así como los diagramas de funcionalidad ajustados a los requerimientos de los profesionales de la salud.

Más adelante, se menciona que el prototipo debe haber ``superado pruebas de factibilidad en condiciones de operación o funcionamiento real''. Actualmente, como se menciona en la \autoref{subc:objective-population-segmentation}, se realizan pruebas relacionadas a la marcha en el laboratorio ``SERH'', Universidad del Valle sede San Fernando, en donde se utiliza una IMU comercial de marca BTS, capaz de llevar a cabo la medición de varias pruebas de la marcha entre ellas incluida la prueba del \textit{Timed Up and Go}. Durante todo el desarrollo de este trabajo, se tendrá como referencia la funcionabilidad de dicho sensor inercial comercial, donde las pruebas finales del prototipo desarrollado en el actual trabajo se compararán con los resultados arrojados en pacientes que hagan la misma prueba utilizando dicho sensor comercial. De esta manera se llevarán a cabo las pruebas de factibilidad en condiciones de operación y funcionamiento real, como especifica esta parte del TRL 6.

Finalmente, se pide que el prototipo logre funcionar en las ``condiciones reales en donde se pretende este funcione''. El principal objetivo de desarrollar una aplicación móvil, junto con su plataforma digital para la observación de los resultados tomados por el celular, es que se pueda usar en ambientes remotos y que los resultados puedan ser analizados por expertos en la materia sin la necesidad de transportar todo el equipo electrónico a dichas localidad, lo cual tomaría mucho recursos económicos y temporales. Por esta razón, la aplicación será probada tanto dentro como fuera de las instalaciones del SERH y se determinará la factibilidad del modelo como fuente alternativa de la prueba en localidades remotas, de esta manera se dará por alcanzado el nivel de madurez tecnológica (TRL) 6.


%%%%%%%%%%%%%
%%%%%%%%%%%%%


\subsection{REQUERIMIENTOS NO FUNCIONALES.} \label{subc:objective-population-segmentation}

Con el ánimo de definir la población objetivo y los requerimientos necesarios para desarrollar una aplicación móvil que permita la correcta toma de datos de la prueba \textit{Timed Up and Go}, se procede a entrevistar al grupo de laboratoristas de las instalaciones del ``Servicio de Rehabilitación Humana'' (SERH). 

Las preguntas que se realizaron a los fisioterapeutas, son las siguientes:

\begin{enumerate}
    \item ¿Qué rango de edades son las personas a las que más se les realiza el test \textit{Timed Up and Go}? \\
    - En promedio, el test se le realiza a las personas mayores de 55 años, en su mayoría pacientes mujeres, lo que concuerda con los datos globales para este test.
    
    \item ¿Qué días realizan la prueba del \textit{Timed Up and Go} en las instalaciones del SERH? \\
    - Los dispositivos con los que cuentan las instalaciones del SERH son relativamente nuevos y, desde el 17 de mayo de 2024, se empezó a brindar el servicio de diagnóstico de la marcha de manera semanal los días viernes.
    
    \item ¿Las pruebas relacionadas con la marcha solamente se realizan en las instalaciones del SERH o se hacen en algún otro departamento de la Facultad de Salud de la Universidad del Valle? \\
    - Todas las pruebas, tanto investigativas como de servicio médico, son realizadas en las instalaciones del SERH. El programa adulto mayor realizado en la Universidad del Valle sede Meléndez es el único lugar donde este tipo de pruebas (las relacionadas con la movilidad) se realizan a pacientes fuera de las instalaciones del SERH.
    
    \item ¿La Facultad de Fisioterapia hace algún tipo de salidas de campo para realizar pruebas relacionados con la marcha fuera de las instalaciones de la Universidad del Valle? \\
    - No se realizan salidas de campo.
    
    \item ¿Qué escenarios se consideran propicios para realizar la prueba del \textit{Timed Up and Go}? \\
    - Cualquier escenario que cuente con una superficie plana sin desniveles, al menos cinco metros de largo (para hacer la marcha en la prueba), una silla estable y sin reposabrazos, y las debidas señalizaciones para indicar al paciente hasta dónde debe caminar antes de devolverse a su silla y volver a sentarse.
    
    \item ¿En las instalaciones del SERH se realiza alguna variante de la prueba \textit{Timed Up and Go} para diagnosticar la dependencia de la marcha de la persona? \\
    - La prueba es estándar y no se realizan variantes de la misma para determinar el grado de dependencia de la persona al caminar. Esta prueba no se realiza con pacientes que presenten daño temporal, es decir, que hayan sufrido un accidente y tengan problemas para caminar, ya que estos pacientes, con terapia, irán recuperando su movilidad e independencia al caminar.
\end{enumerate}

De acuerdo con la asesoría brindada por los fisioterapeutas del laboratorio SERH y tomando como referencia las variables que calcula la IMU \textit{BTS G-Sensor}, se enlistan las variables que se deberán trabajar para satisfacer los requerimientos de los fisioterapeutas:

\begin{tabularx}{\textwidth}{X X}
    - Duración de las fases & - Aceleraciones Antero-Posterior \\
    - Aceleración Lateral & - Aceleración Vertical \\
    - Amplitud angular del tronco & - Variación de ángulos registrados por giroscopio\\
\end{tabularx} \\


%%%%%%%%%%%%%%%%%%%%
%%%%%%%%%%%%%%%%%%%%

\subsubsection{Población obejtivo} \label{subc:1.2.objetive-population-definition}

Luego de realizar la visita a las instalaciones del ``SERH'' de la Universidad del Valle y efectuar la debida entrevista al personal de fisioterapia de este laboratorio, se logra determinar la población objetivo y se obtienen varias conclusiones.

\begin{enumerate}
    \item La población objetivo para llevar a cabo la prueba del \textit{Timed Up and Go} son mayores de edad sanos sin riesgos de caída ni reducción de la movilidad temporal, o sea, personas que hayan sufrido un accidente y por medio de terapias puedan volver a recuperar su movilidad. El rango de edades son personas entre los 26 y 48 años. 
    
    \item Es importante reiterar que las instalaciones del SERH cuentan con un sensor inercial marca BTS, el cual es capaz de realizar el diagnóstico de varias pruebas de marcha, entre ellas la del \textit{Timed Up and Go}. El actual proyecto no pretende ser un sustituto de este sensor, sino un complemento donde se pueda utilizar en ubicaciones remotas donde no sea posible transportar todo el set que se requiere para poder utilizar el sensor BTS.
    De igual manera, como se acaba de mencionar, el dispositivo comercial BTS es capaz de realizar \textbf{diagnósticos}, en el caso de la aplicación desarrollada, esto no será posible debido a los alcances del trabajo de grado; sin embargo, brindará todas las herramientas para que un fisioterapeuta cualificado pueda realizar el diagnóstico de una manera más precisa.
    
    \item Para el diseño de la interfaz de la aplicación móvil y web, se tomará como referencia la interfaz gráfica que proporciona el software comercial de la marca BTS en su presentación para aplicaciones médicas (ya que existe uno para aplicaciones deportivas).
\end{enumerate}

De esta manera se define la población objetivo y se determina un número promedio de pruebas que se deben llevar a cabo para poder determinar una correlación con los resultados obtenidos con el sensor \textit{BTS G-Sensor}. Se recomienda entonces realizar pruebas con al menos 5 individuos de cualquier sexo.

%% Hasta aquí OK

% \textbf{Definición de la población objetivo, escenarios de prueba y número de pruebas a realizar con la ayuda de los fisioterapeutas de la Universidad del Valle.}



%%%%%%%%%%%%%%%%%%%%%%%%%%%%%%%%%%%%%%%%%%%%%%%%%%%%%%%%%%%%%%%%%%%%%%
%%%%%%%%%%%%%%%%%%%%%%%%%%%%%%%%%%%%%%%%%%%%%%%%%%%%%%%%%%%%%%%%%%%%%%

\section{DESARROLLO DE LA APLICACIÓN MÓVIL}

Utilizando la información recolectada a lo largo de la \autoref{sec:1.TRL-6-requirements}, se procede a desarrollar la aplicación móvil que permitirá la recolección de datos.

\subsection{REQUISITOS DE SOFTWARE} \label{subc:2.1.Range-movil-devices}

En el trabajo \cite{Sistema-portable-TUG-AKuleshov} el ingeniero Arturo Perez en el apartado de requerimientos de usuario define que la aplicación ``TUG'' desarrollada en su trabajo investigativo, puede ejecutarse en dispositivos con sistema Android 5.0 en adelante y que tengan en funcionamiento el acelerómetro y el giroscopio.\\

Para el año 2022 el porcentaje de usuarios que utilizaban una versión de Android 5.0 o más antiguas era menos del 5\% \cite{Android-versions-2022}, para el año siguiente, eran menos del 2\% de los usuarios que utilizaban estas versiones. Así mismo, cerca del 60\% de los usuarios Android poseían versiones superiores a la decimo primera, mientras que la versión más usada fue la versión Android 13, con el 22.4\% \cite{Android-versions-2023}. \\

Las versiones de Android 5.0 o inferiores se consideran obsoletas por la poca cantidad de usuarios activos que cuentan con ellas. Las versiones más recientes cuentan con características que en la actualidad se consideran básicas, incluyendo mejoras en los sensores inerciales, que serán utilizadas en el presente proyecto. Por esta razón se determina que para que la aplicación pueda correr en la mayoría de los dispositivos activos a la fecha, el sistema operativo mínimo con el que deberán contar los usuarios es Android 10.0, ya que presenta las características mínimas de compatibilidad con las versiones más modernas.



%%%%%%%%%%%%%%%%%%%%%%%%%%%%%
%%%%%%%%%%%%%%%%%%%%%%%%%%%%%

\subsection{DIAGRAMA FUNCIONAL DE LA APLICACIÓN.} \label{subc:2.2.functional-diagram}


La interfaz del aplicativo móvil tiene como objetivo guiar al usuario para la ejecución de la prueba, almacenamiento y envío de datos al servidor. Se procede entonces a realizar los cambios correspondientes en el diagrama funcional ilustrado en la Figura \autoref{fig:diagramaFuncionalMovil}, antes de proceder a aplicar dichos cambios en la aplicación. Entre los cambios registrados se encuentra el envío de datos a la plataforma digital de la marcha humana y la calibración de la aplicación antes de comenzar la prueba.

Para poder realizar de manera adecuada el diagrama funcional de la aplicación se deben primero listar los requerimientos funcionales y no funcionales de la aplicación móvil con el objetivo de visualizar el panorama completo del aplicativo.

\subsubsection{Requerimientos funcionales y no funcionales del aplicativo móvil} \label{subsubc:movil-functional-non-functional-requirements}

\textbf{Requerimientos Funcionales del aplicativo móvil:} \\

\begin{tabularx}{\textwidth}{X X}
    - Procesamiento de datos personales de los usuarios. & - Calibración del dispositivo antes de realizar la prueba.\\
    
    - Medición del tiempo de las sub-fases. & - Medición de las aceleraciones de interés.\\

    - Consulta del estado de envío de datos. & - Inicio y cancelación de la prueba.\\

    - Verificación del estado de los datos. & - Envío de los datos registrados.\\

    - Eliminación de los datos registrados. & - Indicador sonoro de inicio de la prueba\\
    
    - Almacenamiento de los datos en local y remoto. & - Detección automática del final prueba (Después de 5 segundos de inactividad).\\
    
\end{tabularx} \\

\begin{tabularx}{\textwidth}{X}

    - Los usuarios podrán iniciar sesión localmente, ejecutar y guardar en local cuando no se tenga conexión a internet. \\
    
\end{tabularx} \\

    
\textbf{Requerimientos no funcionales del aplicativo móvil:}

\begin{tabularx}{\textwidth}{X}

    - Interfaz amigable con el usuario siguiendo las metodologías de \textit{UI/UX design}.\\
    - Funcionamiento de la aplicación aún sin conexión a internet (Los datos registrados se subirán a la plataforma una vez se restablezca la conexión a internet).\\
    - Protección de datos personales.\\
    - Escalabilidad.\\
    - Mantenibilidad.\\
    - Operabilidad en dispositivos móviles, Android igual o superior a 10.0.\\
    - Aplicación interactiva con el usuario.\\
    - Guía práctica de la usabilidad de la aplicación.\\

\end{tabularx} \\

\begin{figure}[h!]
    \centering
    \includesvg[width=1.0\textwidth]{Images/movil_functional_diagram_V1.svg} % sin la extensión .svg
    \caption{Diagrama funcional de la aplicación móvil.}
    \label{fig:diagramaFuncionalMovil}
\end{figure}

\subsection{FUNCIONAMIENTO DE LA APLICACIÓN}

\subsubsection{Gestión de datos en local} \label{subc:2.3.Movil-interface-development}

Como parte de los requerimientos funcionales de la aplicación se tiene la posibilidad de poder utilizar la aplicación sin tener conexión a internet, de esta manera, la prueba se puede realizar incluso en localidades remotas con mala o nula conexión. Para poder garantizar el almacenamiento segmentado de las pruebas que se registren, se hace necesario desarrollar un sistema de ``inicio de sesión'', de esta manera la aplicación podrá almacenar localmente los datos de las preubas por usuario, así que si el ``usuario1'' ingresa a la aplicación, éste podrá ejecutar la prueba y estas se registrarán con los datos personales de ese usuario, en consecuencia, solamente podrá ver las pruebas que bajo ese usuario se hayan guardado, teniendo así un sistema organizado que facilita el envío y persistencia de los datos en la base de datos del servidor.

Con esta introducción, se procede a explicar el funcionamiento para ambos casos, con y sin conexión a internet.

\subsubsection{Funcionamiento sin conexión a internet}
Al ejecutar la aplicación, se da inicio a la pantalla de presentación (Ver \autoref{fig:SplashAutoTug}), una vez que el paciente presione el botón ``Comenzar'', se hará una rápida validación del estado de los sensores inerciales del teléfono celular, si estos presentan alguna falla se notificará al usuario que hubo un problema con estos y que la aplicación los requiere para su correcto funcionamiento. Una vez se realice el chequeo de los sensores se procede a realizar el chequeo de conexión a internet, en caso de que el celular se encuentre sin conexión a internet se notificará al usuario que se encuentra sin conexión y se procederá a ingresar en modo ``\textit{Offline}''. En este modo es preciso que el usuario vuelva a ingresar todos sus datos personales para garantizar que cada prueba que se ejecute esté vinculada a un usuario real. Una vez ingresados, se ingresará a la pantalla de inicio de la aplicación, donde se encontrarán 3 tarjetas principales con las opciones de ``Comenzar la prueba'', ``Ver tutorial'' y ``Subir pruebas'' (Ver \autoref{fig:HomeAutoTug}), además de un navegador principal con las opciones de navegar a la pantalla de inicio, ver el historial de las pruebas tomadas y enviadas y para acceder a las configuraciones de la aplicación (Ver \autoref{fig:HistoryAutoTug} y \autoref{fig:SettingsAutoTug}).


\begin{figure}[h!]
    \centering
    \begin{subfigure}{0.45\textwidth}
        \centering
        \includegraphics[width=0.75\linewidth]{Images/SplashAutoTug.png}
        \caption{Pantalla de bienvenida}
        \label{fig:SplashAutoTug}
    \end{subfigure}
    \hspace{1cm}
    \begin{subfigure}{0.45\textwidth}
        \centering
        \includegraphics[width=0.75\linewidth]{Images/HomeViewAutoTug.png}
        \caption{Pantalla principal autoTug}
        \label{fig:HomeAutoTug}
    \end{subfigure}
    \caption{Pantalla bienvenida y Principal}
    \label{fig:Splash&HomeAutoTug}
\end{figure}


\begin{figure}[h!]
    \centering
    \begin{subfigure}{0.45\textwidth}
        \centering
        \includegraphics[width=0.8\linewidth]{Images/HistorialPruebasAutoTug.png}
        \caption{Historial de las pruebas tomadas}
        \label{fig:HistoryAutoTug}
    \end{subfigure}
    \hspace{1cm}
    \begin{subfigure}{0.45\textwidth}
        \centering
        \includegraphics[width=0.8\linewidth]{Images/SettingsAutoTug.png}
        \caption{Menu de configuraciones}
        \label{fig:SettingsAutoTug}
    \end{subfigure}
    \caption{Pantalla de historial de pruebas y configuración}
    \label{fig:History&SettingsAutoTug}
\end{figure}

\subsubsection{Funcionamiento con conexión a internet}\label{subc:Online-functionality}

El funcionamiento con conexión a internet es muy similar al de sin conexión, lo único que cambia son las validaciones que se deben hacer con el servidor, es decir, cuando el usuario quiera enviar las pruebas tomadas al servidor, la aplicación verificará si el dispositivo está conectado a internet, de ser así, verificará si el usuario tiene una sesión activa con el servidor, esto es necesario dado que para poder enviar datos al servidor, este solicitará un token de seguridad único entregado el usuario cuando éste inicie sesión. Si el usuario cuenta con este token la aplicación hará una petición POST al servidor enviando los datos para que este haga la persistencia en la base de datos general de la plataforma de la marcha, si la transacción es satisfactoria se le mostrará un resumen al usuario de la prueba, mostrando los tiempos principales de las fases, en caso de que haya ocurrido un error se le notificará al usuario que hubo un error ya sea porque el servidor no se encuentra en funcionamiento o porque éste envió un error de procesamiento, esto se da cuando hubo un error al ejecutar la prueba, por lo tanto el usuario deberá repetir la prueba.

(PEGAR AQUÍ IMÁGENES DE RESPUESTA SATISFACTORIA CON EL RESUMEN DE LA PRUEBA Y LAS RESPUESTAS DE EXCEPCIÓN, DEBEN HABER POR LO MENOS DOS RESPUESTAS DE EXCEPCIÓN, CUANDO OCURRE UN TIMEOUT {SERVIDOR CAÍDO}, O CUANDO EL SERVIDOR NO PUDO HACER EL CALCULO CORRECTO DE TODAS LAS VARIABLES {PRUEBA TOMADA INCORRECTAMENTE})

\subsubsection{Ejecución de la prueba}

Con el ánimo de que la prueba pueda ser ejecutada por cualquier persona incluso si es la primera vez que éste está usando la app, se le mostrará un pequeño tutorial cuando el usuario oprima por primera vez la opción ``Comenzar prueba'' en desde la pantalla de inicio (Ver \autoref{fig:HomeAutoTug}), en dicho tutorial se le explicará el protocola a seguir para poder ejecutar correctamente la prueba, comenzando primero con las recomendaciones para poder ejecutar la prueba, las cuales son contar con una segunda persona para poder oprimir el botón ``Calibrar'' cuando el paciente ya tenga ubicado el celular en la espalda baja. Luego de las recomendaciones el protocolo es el siguiente: \\

\noindent \textbf{Ubicación del celular.} Para poder ejecutar la prueba de manera correcta se debe tener un sujetador horizontal que permita sujetar el celular en la espalda baja del paciente, en el mercado se pueden conseguir riñoneras deportivas como se muestra en la \autoref{fig:SujetadorCelularComercial}. Debido a que se tuvo inconvenientes para adquirir este producto, para este proyecto se decidió confeccionar uno a partir de los sujetadores de brazo convencionales (Ver \autoref{fig:SujetadorConfeccionado}), y se reutilizaron los orificios para pasar por ahí una correa. El celular deberá colocarse en la espalda baja del paciente, en la vertebra L2, ya que esta es la ubicación que se recomienda colocar el sensor en el manual de usuario del BTS GSensor, y que se corroboró con las pruebas realizadas con la aplicación es el mejor lugar para obtener resultados más claros en cuanto a la detección de las sub-fases de la prueba.


\begin{figure}[h!]
    \centering
    \begin{subfigure}{0.45\textwidth}
        \centering
        \includegraphics[width=0.75\linewidth]{Images/sujetador_celular_horizontal.jpg}
        \caption{Riñonera deportiva horizontal.}
        \label{fig:SujetadorCelularComercial}
    \end{subfigure}
    \hspace{1cm}
    \begin{subfigure}{0.45\textwidth}
        \centering
        \includegraphics[width=0.75\linewidth]{Images/HomeViewAutoTug.png}
        \caption{Sujetador celular confeccionado.}
        \label{fig:SujetadorConfeccionado}
    \end{subfigure}
    \caption{Sujetadores para celular, comercial y diseñado.}
    \label{fig:Sujetador celular comercial y confeccionado.}
\end{figure}

\noindent \textbf{Calibración.} Una vez ubicado el celular, el asistente o supervisor de la prueba deberá oprimir el botón calibrar que aparecerá en pantalla, acto seguido el paciente deberá sentarse en la silla con la espalda recta y recostado en el espaldar de la silla, la aplicación le dará 10 segundos al paciente para ubicarse una vez oprimido el botón de calibrar, acto seguido comenzará la calibración del dispositivo el cual consta de 5 segundos con el paciente completamente quieto, durante este tiempo la aplicación recopilará la información de los sensores para poder determinar cual es el punto de referencia cuando el paciente se encuentre quieto, además de determinar si el dispotivo se encuentra alineado o si éste se encuentra muy torcido y se deba corregir la ubicación. \\


\noindent \textbf{Ejecución de la prueba.} Una vez terminada la calibración, sonará un beep lo que le indicará al paciente que puede comenzar a realizar la prueba. El paciente procederá a levantarse de la silla, recorrer una distancia de 3 metros, dar un giro de 180°, devolverse a la silla, girar nuevamente para volverse a sentar, donde deberá permanecer quieto hasta que la aplicación vuelva a sonar el beep, indicando que la prueba a finalizado. Esta detección automática del final de la prueba se determinó a los 5 segundos de inactividad en los giroscopios tanto del eje X como del eje Y. \\


\subsubsection{Inicio de sesión modo Online}

Como se mencionó en la subsección \autoref{subc:Online-functionality}, cuando el usuario disponga de conexión a internet este podrá autenticarse desde un principio con el servidor, la aplicación detectará automáticamente la conexión a internet y le mostrará la pantalla de Inicio de sesión (Ver \autoref{fig:LogInAutoTug}). Para poder iniciar sesión se deberá contar con un usuario y una contraseña, el usuario será la cédula de ciudadanía del paciente o su correo electrónico, mientras que la clave será proporcionada por los administradores de la plataforma. \\

Una ingresados los datos, se mandará una petición al servidor para la autenticación de las credenciales, de ser correctas, se devolvera una respuesta http con código 200, y dentro del body del JSON de respuesta se enviará un token de seguridad que se solicitará a la aplicación al momento de enviar los datos de las pruebas tomadas. \\

Si la autenticación salió de manera correcta, se hará una última validación interna antes de mostrar la pantalla de inicio, se verificará si el usuario que acaba de iniciar sesión existe dentro de la base de datos local del dispositivo, si no se encuentra información del usuario se le solicitará al paciente rellenar la información por medio de un formulario con los datos necesarios para realizar la prueba, esto es requerido dado que al momento de cread el usuario en la base de datos, este solamente se crea con número de usuario, correo, contraseña y rol del usuario, sin embargo, para las pruebas clínicas se necesita más información acerca del paciente, por este motivo, se solicita esta información al paciente incluso si logra iniciar sesión correctamente, de esta manera, el usuario se crea dentro de una base de datos interna del celular con SQLite, y dicha información es la que se enviará adjunta con los datos raw de las señales capturadas al momento de ejecutar la pruba. \\

En el caso de que el usuario no tenga conexión a internet, o sí tenga conexión pero haya un problema con el servidor, el paciente podrá iniciar sesión en modo Offline, pero para esto tendrá que volver a ingresar sus datos personales nuevamente en el formulario (Ver \autoref{fig:FormAutoTug}). Una vez ingresados podrá ejecutar la prueba y almacenarlas en local ligadas a ese usuario, y una vez se recupere la conexión con el servidor se podrán enviar para ser almacenadas.

\begin{figure}[h!]
    \begin{subfigure}{0.45\textwidth}
        \centering
        \includegraphics[width=1\linewidth]{Images/InicioSesionAutoTug.png}
        \caption{Pantalla de inicio de sesión}
        \label{fig:LogInAutoTug}
    \end{subfigure}
    \hspace{1cm}
    \begin{subfigure}{0.45\textwidth}
        \centering
        \includegraphics[width=1\linewidth]{Images/InicioSesionOfflineAutoTug.png}
        \caption{Formulario registro del paciente en local}
        \label{fig:FormAutoTug}
    \end{subfigure}
    \caption{Pantallas de inicio de sesión Online y Offline}
    \label{fig:OnlineOfflineLogInAutoTug}
\end{figure}

\section{MIGRACIÓN A LA NUBE}

En el trabajo de maestría del ingeniero Arturo Pérez Kuleshova, se evidenció que la mejor forma para detectar el inicio de cada fase de la prueba era analizando las señales de los giroscopios del dispositivo celular, y la manera en cómo se ubicó el celular dado que los celulares con más largos en el eje horizontal que en el vertical, esto podría causar problemas en las fases de sentado-parado y parado-sentado, ya que los extremos podrían chocar con la propia espalda o con los glúteos del paciente, por esta razón, se decidió colocar el celular de manera horizontal para la ejecución de la prueba.\\

Esta posición conlleva a un ajuste en los ejes naturales del celular, al rotar 90 grados, los ejes X y Y son intercambiados de la siguiente manera:

\begin{equation*}
    Smartphone(X,Y) = Smartphone'(X,Y) = (Y,-X)
\end{equation*}

Con lo anterior, el eje craneocaudal corresponde al eje X tanto en el acelerómetro como en el giroscopio, el eje laterolateral corresponde al eje Y y el eje anteroposterior corresponde al eje Z.\\

El acelerómetro mide la aceleración (m/s2) del dispositivo en la dirección correspondiente y el giróscopo mide la velocidad angular (º/s) del dispositivo, tomando de referencia los otros ejes respecto al eje a medir \cite{Sistema-portable-TUG-AKuleshov}.\\

Por medio de anális se determinó que las señales de los giroscópis de los ejes X y Y eran las mejores para determinar el comienzo y final de cada fase, siendo las señales en el eje X quien mejor determinaban el inicio de de la prueba y de la primer marcha, por medio de la integración de la señal en Y se lograba obtener el desplazamiento de esta variable, el susuario al dar media vuelta para volver de regreso a la silla, est supone un desplazamiento de $3\pi$ radianes, y para la segunda vuelta antes de sentarse, obtenemos $6\pi$ radianes, siendo el desplazamiento de la señal la mejor opción para determinar el comienzo y fin de cada giro.\\

Al final del informe se destaca que los resultados fueron satisfactorios, sin embargo, para la continuación en este trabajo de grado se evidenció que los algoritmos requerían ajustes, dado que bajo ciertas condiciones, los tiempos de las fases no podían ser calculadas automáticamente.


% \subsection{AJUSTE DE LOS ALGORITMOS PARA LA DETECCIÓN DE LAS FASES}

% Como se mencionó en el apartado anterior, el procedimiento que se llevó acabo para determinar el comienzo de cada fase de la prueba fue por medio de las señales X y Y de los giroscopios del celular. Después de analizar las señales tomadas ejecutando la prueba, se determinó que la señal en el eje Y (Eje X del celular rotada 90°), era la que mejor permitía detectar los giros, ya que al integrar la señal se obtenía el desplazamiento angular del paciente. Por otro lado, la señal en el eje X (Eje Y del celular rotada 90°), era la que mejor permitía detectar el inicio de la prueba y la primer marcha, ya que en estas dos fases se presentaba un pico evidente al momento de que el paciente extendiera su tronco hacia adelante y para levantarse y luego lo flexionara nuevamente cuando ya estuviera de pie.\\

% Para poder ajustar los algoritmos, se realizaron varias pruebas con sujetos de prueba sanos, en donde se tomaron las señales de los giroscopios y acelerómetros del celular, y se analizaron las señales para determinar los puntos de inicio y fin de cada fase. A continuación se muestran dos ejemplos de las señales tomadas en las pruebas realizadas.\\

% Como se puede ver en la figura \autoref{fig:TestConCono}, en la señal "DesplazamientoX"la señal presenta un pico negativo antes de comenzar a ascender, esto se debe a que el paciente camina en línea recta hacia donde está el cono, luego gira ligeramente a la izquierda (o derecha) para bordear el cono, acto seguido vuelve a girar en la dirección contraria para volver a la silla, esto genera dos picos en la señal de desplazamiento. El algoritmo ya desarrollado tomaba en cuenta estos picos para determinar el comienzo y fin de las fases de giro, sin embargo, la prueba `Timed Up and Go' no necesariamente necesita tener un cono al final de los tres metros ya que esta podría realizarse sin esta, y sin el cono el paciente ya no daría el ligero giro hacia un lado lo que no haría que no se presentara el pequeño pico antes de que la señal comience a crecer.\\





% Por este motivo, se decidió ajustar el algoritmo para que la detección de los giros no dependiera de los picos que se presentaban en la señal de desplazamiento, por esta razón, se aumentó el umbral para determinar por dónde el paciente iba a realizar el giro, ya sea por la izquierda o por la derecha, para esto se tomó el umbral de 0.9, ya que cuando el paciente alcanzara este valor en el `DesplazamientoX' se puede en la mayoría de los casos afirmar que el paciente decidió hacer el giro en el sentido de las manecillas del reloj, al alcanzar el 30\% del valor total del giro de 180°. Ahora, este umbral nos sirve para para detectar por qué lado realizó el giro el paciente, sin embargo, este es no es momento en el que el paciente comenzó a girar, para determinar esto, se determina una bandera para cuando el paciente haya alcanzado el umbral de 0.9 radianes, y se implementa una búsqueda en los valores entre el comienzo de la marcha 1 y la bandera que hemos colocado, y cuando el valor del `Desplazamiento' esté entre los $0.6$ o $-0.6$, sabremos que es ahí donde comenzó el giro, se determinaron estos valores luego de analizar varios resultados, determinando que en ciertas ocasiones, los pacientes podían presentar pequeños picos en el eje laterolateral mientras caminaban hacia la demarcación de los tres metros, por ende, para evitar que estos picos interfirieran en la demarcación de la fase de giro, se decidió que el valor de 0.6 radianes era el mejor para determinar el comienzo del giro, ya que en todos los casos, el paciente no alcanzaba este valor al caminar hacia la demarcación de los tres metros antes de comenzar con el giro.\\


% \subsubsection{Fase Marcha 2}

% La forma como se definió el inicio de la marcha 2 fue cuando el `DesplazamientoX' alcanzara el valor de $3.2$ o $-3.2$ radianes, si era positivo definía el giro de izquierda a derecha y visceversa para cuando el valor fuera negativo, esto indicaba que el paciente ha dado un giro completo de 180° y ha comenzado a caminar de regreso a la silla. Esto en teoría es correcto, sin embargo, mediante la observación de las señales tomadas en las pruebas realizadas, se evidenció que en ciertos casos, el paciente podía dar un pequeño giro antes de llegar a la demarcación de los tres metros, y luego corregir su trayectoria para volver a la silla, esto generaba que el valor del `DesplazamientoX' superara los $3.2$ radianes, y por ende, el algoritmo detectaba erróneamente el comienzo de la marcha 2 antes de que el paciente realmente comenzara a caminar de regreso a la silla.\\

% En otros casos, la señal de `DesplazamientoX' no alcanzaba el valor de $3.2$ radianes, ya que el paciente podía dar un giro muy cerrado para volver a la silla, y en este caso, el valor máximo alcanzado por la señal era cercano a los $2.8$ radianes o incluso menor, lo que hacía que el algoritmo no detectara el comienzo de la marcha 2 y por ende no pudiera calcular los tiempos de las fases siguientes.\\

% Para solucionar este inconveniente, se decidió implementar un umbral dinámico para detectar el comienzo de la marcha 2, este umbral se calcula tomando el valor máximo alcanzado por la señal `DesplazamientoX' durante toda la prueba, y se multiplica por un factor de $0.9$, de esta manera, si el paciente da un giro muy cerrado, el umbral se ajustará a este valor, y si el paciente da un giro más amplio, el umbral también se ajustará a este nuevo valor. De esta manera, se logra detectar de manera más precisa el comienzo de la marcha 2 sin importar la forma en que el paciente decida girar para volver a la silla.\\


\section{Principales cambios en la lógica de segmentación temporal}

A continuación, se describen con mayor detalle los cambios realizados en la lógica de segmentación del algoritmo, comparando la versión original con la versión actualizada del proceso. El objetivo de esta descripción es resaltar las diferencias conceptuales más relevantes, haciendo énfasis en la modificación de umbrales numéricos, en la redefinición de las condiciones lógicas que determinan el comienzo y finalización de cada fase, y en la incorporación de nuevos criterios relacionados con la velocidad angular para detectar los giros con mayor fiabilidad.

\begin{figure}[h!]
    \centering
    Images/SignalsXYTest1
        \includesvg[width=1\textwidth]{Images/SignalsXYTest1}
        \caption{Señal prueba sin cono}
        \label{fig:TestConCono}
\end{figure}


\begin{figure}[h!]
    \centering
        \includesvg[width=1\linewidth]{Images/SignalsXYTest2}
        \caption{Señal giro con cono}
        \label{fig:TestSinCono}
\end{figure}



\begin{figure}[H]
\centering
\begin{tikzpicture}[scale=0.85, every node/.style={scale=0.85}]

\node[startstop] (start) {Inicio del recorrido de datos};

\node[decision, below=of start] (d1)
  {¿Aceleración AP $> 0{,}1$\\
   y desplazamiento AP $> 0{,}1$\\
   y aún no hay inicio sentado--parado?};

\node[block, right=of d1] (sp)
  {Marcar inicio\\ sentado--parado\\ $t_{\text{SP}}$};

\node[decision, below=of d1] (d2)
  {¿Aceleración AP $< -0{,}1$\\
   después del inicio\\ sentado--parado?};

\node[block, right=of d2] (neg)
  {Registrar cruce\\ negativo\\ (cuerpo sube)};

\node[decision, below=of d2] (d3)
  {¿Se ha detectado\\ cruce positivo y\\ negativo, y aún no\\ hay inicio de marcha 1?};

\node[block, right=of d3] (g1)
  {Marcar inicio\\ de marcha 1\\ $t_{\text{G1}}$};

\node[startstop, below=of d3] (end)
  {Continuar con\\ detección de giro 1};

\draw[line] (start) -- (d1);

\draw[line] (d1) -- node[above]{Sí} (sp);
\draw[line] (d1) -- node[right]{No} (d2);

\draw[line] (d2) -- node[above]{Sí} (neg);
\draw[line] (d2) -- node[right]{No} (d3);

\draw[line] (neg) |- (d3);

\draw[line] (d3) -- node[above]{Sí} (g1);
\draw[line] (d3) -- node[right]{No} (end);

\draw[line] (g1) |- (end);

\end{tikzpicture}%
\caption{Detección del inicio sentado--parado y del comienzo de la primera fase de marcha.}
\end{figure}


\begin{figure}[H]
\centering
\begin{tikzpicture}[scale=0.85, every node/.style={scale=0.5}]

\node[startstop] (start) {Desde marcha 1\\ (tiempos $t \ge t_{\text{G1}}$)};

\node[decision, below=of start] (d1)
  {¿Desplazamiento ML $< -0{,}9$\\
   (candidato a giro hacia la izquierda)?};

\node[block, right=of d1] (candL)
  {Recorrer tramo\\ entre inicio de marcha 1\\ y el instante actual};

\node[decision, below=of d1] (d2)
  {¿Desplazamiento ML $> 0{,}9$\\
   (candidato a giro hacia la derecha)?};

\node[block, right=of d2] (candR)
  {Recorrer tramo\\ entre inicio de marcha 1\\ y el instante actual};

\node[block, below=of d2] (markTurn1)
  {Marcar instante en el que\\ el desplazamiento lateral\\ se separa de la zona central\\ como inicio de giro 1\\ $t_{\text{T1}}$};

\node[decision, below=of markTurn1] (d3)
  {¿Se ha detectado máximo\\ giro 1 y se ha registrado\\ el sentido (izquierda/derecha)?};

\node[block, right=of d3] (dir)
  {Guardar sentido\\ del giro 1\\ (izquierda / derecha)};

\node[decision, below=of d3] (d4)
  {¿Se ha detectado en el\\ giróscopo un tramo\\ cercano a cero tras el giro 1\\ y aún no hay inicio de marcha 2?};

\node[block, right=of d4] (g2)
  {Marcar inicio de marcha 2\\ $t_{\text{G2}}$ y guardar\\ índice para el giro 2};

\node[startstop, below=of d4] (end)
  {Continuar con\\ detección de giro 2};

\draw[line] (start) -- (d1);

\draw[line] (d1) -- node[above]{Sí} (candL);
\draw[line] (d1) -- node[right]{No} (d2);

\draw[line] (d2) -- node[above]{Sí} (candR);
\draw[line] (d2) -- node[right]{No} (markTurn1);

\draw[line] (candL) |- (markTurn1);
\draw[line] (candR) |- (markTurn1);

\draw[line] (markTurn1) -- (d3);

\draw[line] (d3) -- node[above]{Sí} (dir);
\draw[line] (d3) -- node[right]{No} (d4);

\draw[line] (dir) |- (d4);

\draw[line] (d4) -- node[above]{Sí} (g2);
\draw[line] (d4) -- node[right]{No} (end);

\draw[line] (g2) |- (end);

\end{tikzpicture}%
\caption{Detección del giro 1 y del comienzo de la segunda fase de marcha.}
\end{figure}


\begin{figure}[H]
\centering
\begin{tikzpicture}[scale=0.85, every node/.style={scale=0.5}]

\node[startstop] (start) {Desde marcha 2\\ (tiempos $t \ge t_{\text{G2}}$)};

\node[decision, below=of start] (d1)
  {¿La trayectoria lateral\\ ha alcanzado el tramo\\ de giro máximo en ML\\ según el sentido del primer giro?};

\node[block, right=of d1] (deriv1)
  {Calcular derivada\\ primera del desplazamiento ML\\ entre inicio de marcha 2\\ e instante actual};

\node[decision, below=of d1] (d2)
  {¿Cambio de pendiente\\ superior al umbral\\ (inicio de giro 2)?};

\node[block, right=of d2] (t2)
  {Marcar inicio\\ del giro 2\\ $t_{\text{T2}}$};

\node[decision, below=of d2] (d3)
  {¿Módulo del giróscopo ML\\ por debajo de $0{,}05$\\ durante un intervalo\\ continuo?};

\node[block, right=of d3] (endTurn2)
  {Marcar fin de giro 2\\ $t_{\text{FT2}}$};

\node[decision, below=of d3] (d4)
  {¿Aceleración AP $> 0{,}01$\\ después del fin del giro 2\\ y aún no hay inicio parado--sentado?};

\node[block, right=of d4] (sitStart)
  {Marcar inicio\\ parado--sentado\\ $t_{\text{PS}}$};

\node[decision, below=of d4] (d5)
  {¿Aceleración AP $< -0{,}05$\\ seguida de un retorno\\ a valores cercanos a cero?};

\node[block, right=of d5] (testEnd)
  {Marcar fin de la prueba\\ $t_{\text{fin}}$};

\node[startstop, below=of d5] (end)
  {Fin del análisis de fases};

\draw[line] (start) -- (d1);

\draw[line] (d1) -- node[above]{Sí} (deriv1);
\draw[line] (d1) -- node[right]{No} (d3);

\draw[line] (deriv1) |- (d2);

\draw[line] (d2) -- node[above]{Sí} (t2);
\draw[line] (d2) -- node[right]{No} (d3);

\draw[line] (t2) |- (d3);

\draw[line] (d3) -- node[above]{Sí} (endTurn2);
\draw[line] (d3) -- node[right]{No} (d4);

\draw[line] (endTurn2) |- (d4);

\draw[line] (d4) -- node[above]{Sí} (sitStart);
\draw[line] (d4) -- node[right]{No} (d5);

\draw[line] (sitStart) |- (d5);

\draw[line] (d5) -- node[above]{Sí} (testEnd);
\draw[line] (d5) -- node[right]{No} (end);

\draw[line] (testEnd) |- (end);

\end{tikzpicture}%
\caption{Detección del giro 2, de la transición parado--sentado y del fin de la prueba.}
\end{figure}












\subsection{Cambios generales de preprocesamiento}

Un primer cambio importante consiste en la introducción de una comprobación automática de la orientación de los ejes del sensor. En la versión original era necesario asumir que el teléfono estaba correctamente orientado, mientras que en la versión actual se verifica la magnitud relativa de los desplazamientos integrados en los dos ejes horizontales. Si el desplazamiento lateral resulta ser significativamente menor que el desplazamiento antero–posterior, se interpreta que los ejes han sido invertidos, lo cual puede ocurrir si el dispositivo es colocado al revés o con rotación. En ese caso se realiza un intercambio automático de componentes y su posterior reintegración, garantizando así que el procesamiento posterior trabaje siempre con las mismas referencias físicas. Este cambio aumenta considerablemente la robustez del algoritmo y elimina fallos que antes dependían exclusivamente de la correcta posición del dispositivo.

Otro cambio general consiste en separar explícitamente el instante en el cual termina el segundo giro. En la versión previa, tanto el giro como el movimiento de sentado formaban una única transición final, de manera que la duración del segundo giro incluía parte del proceso posterior a la rotación. En contraste, la versión actual define un final del segundo giro basado en el acercamiento de la velocidad angular hacia valores inferiores a un pequeño umbral alrededor de cero, y sólo a partir de ese punto comienza la fase pie–a–sentado. Esta separación temporal permite medir de forma más precisa ambas fases.

\subsubsection{Inicio de la fase sentado--parado}

La fase sentado–parado estaba definida originalmente por un umbral fijo aplicado directamente sobre la aceleración, de manera que el inicio se detectaba cuando la señal superaba aproximadamente un valor de $0{,}05$. En la versión actual dicho valor se incrementó aproximadamente al doble, de modo que sólo se detecten movimientos suficientemente marcados. Asimismo, la lógica actual requiere simultáneamente que el desplazamiento integrado en el eje de avance sea mayor que un pequeño valor umbral. Esto implica que ya no es suficiente simplemente con que exista un cambio aceleracional aislado, sino que debe verificarse un movimiento real del cuerpo. El cambio en los umbrales no sólo reduce falsos comienzos, sino que además elimina la sensibilidad a oscilaciones de alta frecuencia presentes durante el arranque de la señal.

En la lógica antigua, el final de la fase sentado–parado y el comienzo de la marcha se deducían a partir de una secuencia de cruces por umbrales positivos y negativos. Esta secuencia se basaba en detectar primero un cambio negativo, después un cambio positivo y finalmente una estabilización positiva. Ahora, la versión actual emplea solamente un cruce negativo seguido de un cruce por valores próximos a cero. Esto disminuye la dependencia de un nuevo pico positivo, que podía no registrarse con claridad dependiendo del tipo de teléfono y de la forma de levantarse de cada sujeto. De este modo, la transición queda determinada por un patrón más robusto y menos dependiente de oscilaciones específicas de la señal.

\subsubsection{Detección de la marcha inicial}

En comparación con la versión original, la detección de la marcha inicial en la versión nueva deja de depender de un último ascenso de la aceleración vertical. Toda la lógica ahora considera suficiente que el patrón completo de cruce negativo y estabilización se haya producido, incluso si la señal no vuelve a niveles positivos significativos. Con esta modificación se evita que ciertos registros, en los cuales el usuario inicia la marcha sin impulso vertical evidente, generen errores de detección.

\subsubsection{Inicio del primer giro}

Un cambio relevante aparece también en la detección del primer giro. En la versión previa se usaban umbrales relativamente altos de desplazamiento lateral, lo cual obligaba a que el usuario hubiera recorrido una parte considerable de la trayectoria antes de que el algoritmo reconociera el giro. Además, la lógica anterior dependía de una búsqueda hacia atrás sobre la señal integrada para determinar con exactitud cuándo había ocurrido el cambio de dirección. En la versión actual, los umbrales son notablemente menores, por lo que el giro puede identificarse antes. También se reemplaza la búsqueda invertida por una búsqueda limitada a la ventana comprendida entre el inicio de la marcha y el instante actual, lo que simplifica el procesamiento y evita tener que procesar secciones extensas del registro tras cada condición de giro.

Además, desaparece la necesidad de emplear banderas explícitas para distinguir entre giro hacia la izquierda o hacia la derecha en esta primera fase. En su lugar, la dirección se deduce directamente del signo del desplazamiento, lo que hace la lógica más compacta y menos dependiente de múltiples estados intermedios.

\subsubsection{Detección de la marcha de retorno}

A diferencia de la lógica anterior, donde el inicio de la marcha de retorno dependía únicamente del desplazamiento lateral una vez terminado el primer giro, la nueva versión emplea información adicional sobre la velocidad angular y exige una serie de muestras consecutivas en torno al valor cero antes de determinar que la rotación ha terminado. Esto implica que la transición entre giro y marcha deja de estar gobernada exclusivamente por un cruce de desplazamiento y pasa a estar controlada por un comportamiento dinámico más realista: un giro termina cuando el cuerpo deja de rotar, no únicamente cuando atraviesa un valor fijo. Por tanto, esta fase incorpora una lógica dinámica que combina comportamiento del desplazamiento con comportamiento de su velocidad.

\subsubsection{Inicio del segundo giro}

La detección del segundo giro es probablemente donde se introdujeron los ajustes más profundos. En la versión original se empleaba una lógica basada en bandas de desplazamiento, donde cada banda activaba una bandera distinta y estas banderas se combinaban para determinar el instante del giro. En contraste, la nueva versión calcula la derivada del desplazamiento y emplea esta derivada para identificar cuándo el movimiento lateral cambia de forma significativa. La transición hacia el giro se reconoce cuando esta derivada supera un umbral fijo, indicando que el usuario ha comenzado a modificar su trayectoria. Además, la dirección tomada durante el primer giro condiciona los valores umbrales empleados para el segundo giro, haciendo que esta fase sea coherente con la lectura real del movimiento. De esta forma, la detección depende de la secuencia de giros y no únicamente de umbrales estáticos. Asimismo, desaparecen prácticamente todas las banderas internas asociadas a “zonas” del giro, que en la versión original complicaban la lógica.

\subsubsection{Transición final pie--sentado y final de la prueba}

Finalmente, la lógica que detecta el inicio de la última fase también se simplifica considerablemente. Mientras que en la versión inicial se realizaba una búsqueda sobre el historial reciente, ahora la nueva versión identifica el inicio del movimiento cuando la aceleración supera un umbral positivo relativamente pequeño. En la práctica, esto evita búsquedas hacia atrás y depende solamente del instante local. El final de la prueba continúa definiéndose mediante cambios de signo, pero con banderas nuevas dedicadas exclusivamente a esta fase, sin reutilizar estados internos previos. En conjunto, la separación entre el final del segundo giro y la entrada en la posición sentada permite que cada fase disponga de criterios independientes, evitando que los movimientos finales queden mezclados con la rotación.

\subsubsection{Resumen conceptual}

Podría resumirse que la versión actual reestructura la lógica no solamente cambiando valores numéricos, sino modificando profundamente la manera en que los movimientos se interpretan. El algoritmo deja de depender sólo del desplazamiento integrado y de secuencias de cruces por umbrales altos, y pasa a apoyarse en una combinación de desplazamiento, derivada del movimiento y velocidad angular. Los umbrales se ajustan a valores más realistas, eliminando dependencia de picos poco robustos y evitando que pequeñas oscilaciones sean interpretadas como movimientos significativos. Por tanto, la nueva lógica distingue mejor cada fase, reduce el número de banderas intermedias y mejora la capacidad del sistema para segmentar la prueba con mayor fiabilidad clínica.



\subsubsection{Limitaciones de la plataforma}
Desde un principio se consideró la opción de otorgar a los pacientes la libertad de crear su propio usuario directamente desde la aplicación, sin embargo, debido a que para el desarrollo de la plataforma no se tenía total libertad de codificación, no se pudo llegar a un acuerdo para la creación de los usuarios directamente de la aplicación, debido a las limitaciones de la plataforma y a la preocupación de saturación de usuarios por parte de la aplicación, por ende, los pacientes solamente podrán ser creados desde la plataforma de la marcha por los administradores, y serán estos los que proporcionarán usuario y contraseña.




% \begin{figure}[h!]
%     \begin{subfigure}{0.45\textwidth}
%         \centering
%         \includegraphics[width=1\linewidth]{Images/start-test-old-version.png}
%         \caption{Vista original de la aplicación TUG para arrancar el test.}
%         \label{fig:old-view-start-test}
%     \end{subfigure}
%     \hspace{1cm}
%     \begin{subfigure}{0.45\textwidth}
%         \centering
%         \includegraphics[width=1\linewidth]{Images/Start-test-view-new-version.png}
%         \caption{Propuesta de nueva vista aplicación TUG para arrancar el test.}
%         \label{fig:new-view-start-test}

%     \end{subfigure}
%     \caption{Comparación de vistas para arrancar la prueba del \textit{Timed Up and Go}.}
%     \label{fig:start-test-comparison-old-new-version}
% \end{figure}

La vista original para arrancar la prueba del \textit{Timed Up and Go} como se muestra en la \autoref{fig:old-view-start-test}, está llena de información y botones que en primera instancia no tendrían utilidad sin antes haber realizado la prueba. Por esta razón se propone una vista minimalista, intuitiva y dejando la única opción ejecutable después de haber ingresado los datos de usuario, como se aprecia en la \autoref{fig:new-view-start-test}. Así mismo, antes de iniciar la prueba, habrá una animación en el centro de la pantalla ilustrando la ejecución de la prueba. De esta manera se siguen las metodologías de \textit{UX/UI design} revisadas anteriormente y se toma como referencia el inicio de la prueba de la IMU \textit{BTS G-Sensor}, el cual muestra una animación de la prueba.

\subsection{Actividad 2.4. Análisis de los algoritmos diseñados por el ingeniero Arturo Perez para la prueba Timed Up and Go.}

Los algoritmos realizados en el trabajo \cite{Sistema-portable-TUG-AKuleshov} fueron analizados para poder replicar ejecutar la aplicación móvil, dicho análisis servirá para poder realizar los cambios correspondientes en los algoritmos que permitirán la toma de datos de los nuevos parámetros de aceleración discutidos en la \autoref{subc:objective-population-segmentation}.

Las actividades 2.5, 2.6 y 2.7 se realizarán en el próximo semestre académico del presente año debido a la complejidad tema.



%%%%%%%%%%%%%%%%%%%%%%%%%%%%%%%%%%%%%%%%%%%%%%%%%%%%%%%%%%%%%%%%%%%%
%%%%%%%%%%%%%%%%%%%%%%%%%%%%%%%%%%%%%%%%%%%%%%%%%%%%%%%%%%%%%%%%%%%%

\section{Desarrollar la aplicación web con almacenamiento en la nube.} \label{sec:3.web-application-development}

\subsection{Actividad 3.1. Desarrollo de la arquitectura de software del aplicativo web incluyendo la elección del servidor o proveedor de servicios donde se almacenarán los datos de los usuarios.} \label{subc:3.1.Web-software-architecture}

Uno de los propósitos del presente trabajo de grado es lograr la automatización del almacenamiento de los datos tomados por la aplicación móvil para que estos puedan ser analizados de manera remota por un especialista, es por esto que se propone la arquitectura de la figura \ref{fig:software-architecture-data-automation}, la cual permitiría el flujo de datos unidireccional para los usuarios de la aplicación (pacientes que realizan la prueba) y bidireccional para los fisioterapeutas que analizarán los datos los cuales tendrán privilegios de administrador y podrán visualizar todos los datos recopilados por la aplicación.

Los datos podrán ser visualizados desde la plataforma virtual de análisis de la marcha humana la cual será desarrollada por el grupo de investigación GICI y con los aportes del presente trabajo de grado. De esta manera, la arquitectura que se propone deberá ser evaluada por todo el equipo de investigación y estará sujeta a los cambios que el grupo tome en conjunto. \\

\textbf{Requisitos no funcionales:}

\begin{tabularx}{\textwidth}{X X X}
    - Verificación de usuario & - Base datos central & - Interfaz amigable \\
    - Escalabilidad h/v & - Mantenibilidad & - Administración de permisos \\
    - Uso de API's & - Protección datos personales
\end{tabularx} \\ \\


\textbf{Actores de la arquitectura:} 

\begin{tabularx}{\textwidth}{X X X}
    - Base datos General & - Usuario & - Autenticación usuario \\
    - Dispositivos (nodos) & 

\end{tabularx} \\



\subsection{Actividad 3.2. Desarrollo de la interfaz web aplicando metodologías de \textit{UI/UX design} junto con técnicas de optimización de páginas web.} \label{subc:3.2.Web-interface-development}

El desarrollo de la aplicación web debe verse respaldado con un diagrama funcional que permita visualizar todo el alcance del aplicativo, sin embargo, dado que este aplicativo estará alojado en la plataforma digital desarrollado por el grupo de investigación de control industrial ``GICI'', no se podrán desarrollar a completitud hasta haber definido la metodología que se llevará cabo para el desarrollo de dicha plataforma. Pese a esto, se adelantaron los trabajos del lado del \textit{Frontend} de algunas visualizaciones que dispondrá la plataforma una vez se concluya con qué tipo de servicios trabajará la plataforma en su totalidad, entre otras palabras, una vez se concluya cómo se trabajará el \textit{Backend} en la plataforma.

% Se recomienda utilizar un patrón de arquitectura como el de cliente-servidor o el modelo MVC ya que son los que mejor se adaptan a las necesidades del proyecto.



% \begin{figure}[h!]
%     \begin{subfigure}{0.45\textwidth}
%         \centering
%         \includegraphics[width=1\linewidth]{Images/Login_gait_platform_View.pdf}
%         \caption{Login aplicación web.}
%         \label{fig:login-gait-platform-view}
%     \end{subfigure}
%     \hspace{1cm}
%     \begin{subfigure}{0.45\textwidth}
%         \centering
%         \includegraphics[width=1\linewidth]{Images/Home_Admin View_Gait_Platform.pdf}
%         \caption{Vista principal aplicación web.}
%         \label{fig:home-admin-view-gait-platform}
%     \end{subfigure}
%     \caption{Interfaces propuestas para el \textit{Login} y la vista principal de la aplicación web plataforma de la marcha humana.}
%     \label{fig:gait-platform-home-login-view}
% \end{figure}

En la \autoref{fig:login-gait-platform-view} se propone un Login sencillo para el acceso a usuario registrados o la solicitud de creación de usuario que será evaluada por los administradores del sitio. Por otra parte en la \autoref{fig:home-admin-view-gait-platform} se desarrolló una propuesta para la visualización de las diferentes opciones que tendrá un usuario con permisos de administrador el cual deberá tener acceso a los diferentes proyectos que estén utilizando la plataforma digital para alojar los datos que estos recolecten, de igual manera tendrá acceso a la administración de memoria y a la gestión de permisos de nuevos usuarios. 

De momento estas vistas son solo una propuesta que tendrá que ser discutida con el grupo de investigación y determinar si son implementas.

\subsection{Actividad 3.3. Diseño de la base de datos donde se almacenarán los datos de los usuarios implementando protocolos de seguridad informática.}

El diseño de la base de datos a implementar en el proyecto se ve influida en el tipo de arquitectura para la base de datos que se desarrollará en la plataforma digital de la marcha. Sin embargo, un diseño preliminar de la base de datos se ve en la \autoref{fig:relational-entity-database}, donde se pueden distinguir 3 tablas principales con el fin de seguir las reglas de una base de datos en su tercera formal normal. La tabla ``Datos usuario'' presenta una relación de uno a muchos con la tabla "Pruebas registradas", ya que un mismo usuario puede tener registradas una más pruebas en la base de datos. Así mismo, la tabla "Pruebas registradas" presenta una relación de uno a uno con la tabla "Metadatos", donde cada prueba registrada tiene una fecha y hora específicas.

\section{Conclusiones} \label{sec:conclusions}

% 


$\theta_Y(k) = \theta_Y(k-1) + \omega_Y(k),\Delta t$


% sumando la velocidad angular instantánea $\omega_Y$ por el intervalo de muestreo. Esta integración actúa como un filtro de ruido, suavizando las variaciones rápidas y acumulando solo el cambio neto de orientación. El algoritmo impone condiciones sobre $\theta_Y$ para determinar el inicio del giro: cuando el ángulo integrado supera un umbral (por ejemplo, $|\theta_Y| > 2,5$), significa que el usuario comenzó a girar su cuerpo. En ese instante (t₂) se marca el comienzo del primer giro. Dado que el giro puede ser hacia la izquierda o derecha, se evalúa el signo de $\omega_Y$/$\theta_Y$ para distinguir la dirección y establecer banderas (Giro_a_izquierda o Giro_a_derecha). Para refinar el tiempo exacto t₂, el algoritmo realiza un barrido temporal inverso: una vez detectado que $|\theta_Y|$ excedió 2,5, se recorre la señal hacia atrás para encontrar el punto donde $\theta_Y$ comenzó a desviarse (por ejemplo, donde $\theta_Y$ cruzó ±0,5), y ese punto se toma como el inicio efectivo del giro. Durante el desarrollo del giro alrededor del cono, la señal integrada $\theta_Y$ presenta típicamente una secuencia de variaciones positiva, luego negativa y nuevamente positiva (o viceversa) mientras el sujeto rodea el cono y se realinea para el camino de regreso. El fin del primer giro se identifica cuando el sujeto termina de rodear el cono y vuelve a orientarse hacia la dirección de regreso. Algorítmicamente, esto se refleja en $\theta_Y$ estabilizándose tras cambiar de signo por última vez. Por ejemplo, para un giro hacia la izquierda, $\theta_Y$ alcanzará un valor negativo extremo cuando el sujeto esté girando alrededor del cono, y al terminar el giro $\theta_Y$ volverá a aumentar (>-3,2 rad) indicando que el sujeto se enderezó hacia la trayectoria de vuelta. Ese instante se marca como t₃, fin del giro 1 e inicio de la segunda marcha.


% Marcha de regreso (segunda fase de caminata): Tras completar el primer giro, el sujeto camina de regreso hacia la silla. El inicio de la marcha de regreso t₃ se definió al terminar el giro previo; no obstante, si el sujeto hace una pequeña pausa tras el giro (como se observó en algunos casos), el algoritmo puede tomar el reinicio claro de la señal de giróscopo Y o acelerómetro como referencia. Durante la marcha de regreso, nuevamente se observan patrones de paso en acelerómetro (picos periódicos) y el giróscopo Y permanece relativamente estable, dado que el sujeto camina en línea recta. El final de esta fase ocurrirá cuando el sujeto inicie el giro final para sentarse, detectado a continuación.


% Giro final para sentarse (segundo giro 180°): Al llegar de vuelta cerca de la silla, el sujeto debe girar ~180° sobre su propio eje para orientarse con la silla y sentarse. Esta fase de giro 2 suele ser más rápida y de menor desplazamiento que el giro en el cono, apareciendo como una única variación prominente en la señal del giróscopo Y. El algoritmo, nuevamente analizando la integral $\theta_Y$, determina el inicio del segundo giro t₄ cuando detecta dicha variación final. Dado que la dirección de este giro final depende de cómo realizó el giro inicial (podría girar en el mismo sentido que el primer giro o en el opuesto), se consideraron cuatro casos posibles combinando las orientaciones (izquierda-izquierda, izquierda-derecha, derecha-izquierda, derecha-derecha). En la etapa de desarrollo se definieron rangos de $\theta_Y$ integrados para clasificar el caso: por ejemplo, si al terminar la marcha de regreso la integral $\theta_Y$ se encuentra en un rango alto positivo (bandera Bandera_arriba_2), implicaría que ambos giros previos fueron hacia la izquierda, entonces se espera que el segundo giro inicie cuando $\theta_Y$ supere cierto valor positivo (ej. >6); si $\theta_Y$ está cerca de cero con una bandera intermedia (Bandera_medio_2), indicaría giros opuestos en ida y vuelta, etc., ajustando el umbral condicional (ej. >0 o >0,25) según el caso. De manera análoga, un valor alto negativo de $\theta_Y$ (bandera Bandera_abajo_2) indicaría giros previos a la derecha, esperando el segundo giro cuando $\theta_Y < -6`. Estas condiciones lógicas (implementadas con if/else anidados) garantizan detectar robustamente el comienzo del giro final t₄ en cualquiera de los cuatro escenarios posibles. En resumen, t₄ marca el momento en que el sujeto empieza a pivotar para sentarse. //


% Sentarse en la silla (transición bípedo-sedente) y fin de la prueba: Finalmente, se delimita la fase de sentarse, desde que el sujeto inicia la maniobra de descenso hasta que se estabiliza sentado. Para detectar el inicio de la fase de sentado (t₅), se vuelve a analizar el giróscopo X. Al igual que en la etapa inicial, este eje capta el balanceo antero-posterior del tronco. Cuando el sujeto comienza a sentarse, suele inclinar el torso hacia adelante y luego hacia atrás al apoyarse en la silla. El algoritmo busca entonces una variación significativa de $\omega_X$: por ejemplo, cuando $\omega_X$ excede +1 (umbral que indica un movimiento angular brusco hacia adelante) se toma ese momento –o su antecedente inmediato determinado por barrido inverso cuando $\omega_X$ cruza de valores negativos a 0– como t₅. A partir de t₅, el sistema considera que el sujeto está en proceso de sentarse. El fin de la fase de sentado y fin de la prueba se determina cuando el giróscopo X indica que el movimiento ha cesado completamente. En la práctica, se espera la secuencia: inclinación adelante (señal $X$ positiva), seguida de inclinación atrás al recostarse (señal $X$ negativa) y finalmente estabilización. Así, se emplean dos umbrales similares a los de la etapa inicial: al detectar que $\omega_X$ cae por debajo de un pequeño valor negativo (ej. < –0,05), se activa una bandera (Bandera_arriba_1_fin) indicando que el tronco comenzó a recostarse; luego, cuando $\omega_X$ vuelve a subir por encima de –0,05 (es decir, la oscilación termina y la señal retorna cerca de cero), se activa Bandera_abajo_1_fin que marca el fin de la transición de sentado. El tiempo correspondiente se registra como t₆, finalizando oficialmente el cronómetro del TUG. En escenarios reales, se añade la condición de que si tras t₆ no se detecta actividad durante un corto intervalo (señales inerciales quietas), se confirma el fin de la prueba. Esto previene contar tiempo o ruido residual después de que el sujeto ya se sentó.


Cálculo de tiempos de subfases y variables derivadas: Con los instantes $t_0, t_1, ..., t_6$ determinados por el algoritmo de segmentación, se calculan los tiempos de cada subfase del TUG mediante simples diferencias. En particular:
Tiempo total del TUG: $T_{\text{total}} = t_{6} - t_{0}$. Es el indicador global de desempeño (menos tiempo implica mejor movilidad funcional).


Tiempo de levantarse (sedente-bípedo): $T_{\text{levantar}} = t_{1} - t_{0}$, correspondiente a la transición de estar sentado a estar de pie.


Tiempo de marcha inicial (ida 3 m): $T_{\text{marcha1}} = t_{2} - t_{1}$, cubre desde que el sujeto empieza a caminar hasta justo antes de iniciar el giro en el cono.


Tiempo de giro en el cono: $T_{\text{giro1}} = t_{3} - t_{2}$, duración para completar el primer giro de 180° alrededor del cono.


Tiempo de marcha de regreso: $T_{\text{marcha2}} = t_{4} - t_{3}$, desde que termina el primer giro hasta antes de iniciar el giro final para sentarse.


Tiempo de giro final (pre-sentado): $T_{\text{giro2}} = t_{5} - t_{4}$, tiempo empleado en girar 180° para orientarse con la silla.


Tiempo de sentarse (bípedo-sedente): $T_{\text{sentarse}} = t_{6} - t_{5}$, corresponde a la maniobra de descenso y acomodación en la silla.


Cada una de estas variables temporales ofrece información sobre el desempeño en esa sub-tarea específica. La suma de todos los tiempos parciales equivale al tiempo total del TUG ($T_{\text{total}} \approx \sum_{i} T_{\text{subfase}_i}$). Un análisis cuantitativo de estos parámetros parciales permite identificar si un desempeño pobre en la prueba se debe a lentitud en alguna fase particular (por ejemplo, marcha lenta vs. dificultad en giros o en levantarse/sentarse). En el contexto de evaluación de riesgo de caídas, estas variables biomecánicas segmentadas brindan un perfil más completo: por ejemplo, un tiempo de giro prolongado podría indicar problemas de equilibrio o coordinación, mientras que un tiempo de transición sentado-parado elevado puede sugerir disminución de fuerza en miembros inferiores.\\

% Además de los tiempos, del procesamiento de las señales inerciales se pueden derivar otras variables biomecánicas de interés. Por ejemplo, la aceleración lineal medida en cada eje puede integrarse (una vez corregida la componente de gravedad) para estimar velocidades lineales o incluso desplazamientos durante la marcha. En este proyecto, dado que la distancia de marcha es fija (3 m ida y 3 m vuelta), no fue imprescindible calcular la posición, pero sí podrían obtenerse velocidades promedio de marcha a partir de $T_{\text{marcha1}}$ y $T_{\text{marcha2}}$ (velocidad ≈ distancia/tiempo). Igualmente, la señal de velocidad angular (giróscopo) integrada proporciona ángulos de giro efectivos del sujeto (por ejemplo, la integral del eje Y al final del giro en el cono debería acercarse a ~$\pi$ radianes, es decir 180°). Durante el procesamiento se observó que la integración de $\omega_Y$ no solo facilitó la detección de eventos, sino que también redujo el ruido presente en $\omega_Y$ cruda, actuando como un filtro acumulativo. Las señales de aceleración pueden filtrarse adicionalmente para separar movimientos de baja frecuencia (asociados a desplazamientos del cuerpo) de componentes de alta frecuencia (vibraciones o impacto de pasos). Para futuras mejoras, podrían aplicarse filtros pasaaltos para aislar los picos de aceleración al contacto de cada paso, calculando cadencia o número de pasos, o bien filtros pasabajos e integración doble para estimar el desplazamiento de manera más precisa.



% 








En el presente informe de los avances registrados para el trabajo de Grado 1 se lograron investigar y documentar los conceptos teóricos necesarios para el desarrollo del proyecto, así como definir la población objetivo, la gama de equipos donde correrá la aplicación y los nuevos parámetros de interés para la recolección de datos.

Aunque no se logró concretar el objetivo específico 2, se pueden destacar avances importantes en cuanto la funcionalidad del aplicativo móvil. De igual manera, se realizaron avances en la estructura y desarrollo de la aplicación móvil y web, así como una vista preliminar de la arquitectura y la base de datos a implementar.


































%%%%%%%%%%%%%%%%%%%%%%%%%%%%%%%%%%%%%%%%%%%%%%%%%%%%%
%% CONCLUSIONES Y RECOMENDACIONES
%%%%%%%%%%%%%%%%%%%%%%%%%%%%%%%%%%%%%%%%%%%%%%%%%%%%%

\clearpage
\section{CONCLUSIONES}

% Conclusiones aquí.

\clearpage
\section{RECOMENDACIONES}

% Recomendaciones aquí.

\end{justify}

%%%%%%%%%%%%%%%%%%%%%%%%%%%%%%%%%%%%%%%%%%%%%%%%%%%%%
%% BIBLIOGRAFÍA PRINCIPAL
%%%%%%%%%%%%%%%%%%%%%%%%%%%%%%%%%%%%%%%%%%%%%%%%%%%%%

\addcontentsline{toc}{section}{REFERENCIAS}
\bibliographystyle{plain}
\bibliography{bibliography}

%%%%%%%%%%%%%%%%%%%%%%%%%%%%%%%%%%%%%%%%%%%%%%%%%%%%%
%% BIBLIOGRAFÍA COMPLEMENTARIA (ejemplo)
%%%%%%%%%%%%%%%%%%%%%%%%%%%%%%%%%%%%%%%%%%%%%%%%%%%%%

\clearpage
\begin{justify}
\section*{BIBLIOGRAFÍA COMPLEMENTARIA}

    
\addcontentsline{toc}{section}{BIBLIOGRAFÍA COMPLEMENTARIA}

Strongman, C. (2020). \textit{Modern approaches to gait analysis using wearable sensors}. Journal of Biomechanics, 54(2), 110–125. \\

Muñoz, H. (2019). \textit{Tecnologías móviles aplicadas a la salud}. Editorial Alfaomega. \\

López, M. \& García, F. (2021). \textit{Sistemas portátiles para la evaluación clínica del movimiento humano}. IEEE Latin America Transactions, 19(8), 1402–1410. \\

\end{justify}

%%%%%%%%%%%%%%%%%%%%%%%%%%%%%%%%%%%%%%%%%%%%%%%%%%%%%
%% ÍNDICE (analítico)
%%%%%%%%%%%%%%%%%%%%%%%%%%%%%%%%%%%%%%%%%%%%%%%%%%%%%

%\clearpage
%\addcontentsline{toc}{section}{ÍNDICE}
%\printindex

%%%%%%%%%%%%%%%%%%%%%%%%%%%%%%%%%%%%%%%%%%%%%%%%%%%%%
%% ANEXOS
%%%%%%%%%%%%%%%%%%%%%%%%%%%%%%%%%%%%%%%%%%%%%%%%%%%%%

\clearpage
\section*{ANEXOS}
\addcontentsline{toc}{section}{ANEXOS}

%% PLANO TRANSVERSAL


%%% ANEXO A %%%%%%%%%%%%%%%%%%%%%%%%%%%%%%%%%%%%%%%%%%
\subsection*{ANEXO A. Encuesta aplicada a los participantes}

\noindent A continuación se presenta la encuesta utilizada para recopilar información demográfica previa a la prueba.

\begin{itemize}
    \item Edad del participante  
    \item Género  
    \item Nivel de actividad física  
    \item Historial de caídas durante el último año  
\end{itemize}

\noindent\textit{Fuente: elaboración propia.}

\vspace{1cm}

%%% ANEXO B %%%%%%%%%%%%%%%%%%%%%%%%%%%%%%%%%%%%%%%%%%
\subsection*{ANEXO B. Fotografías del montaje experimental}

% \begin{figure}[H]
%     \centering
%     \includegraphics[width=0.7\textwidth]{ejemplo_foto_montaje.jpg}
%     \caption{Montaje utilizado durante las pruebas experimentales.}
%     \label{anexoB_foto1}
% \end{figure}

\noindent\textit{Fuente: archivo del investigador.}

\vspace{1cm}

%%% ANEXO C %%%%%%%%%%%%%%%%%%%%%%%%%%%%%%%%%%%%%%%%%%
\subsection*{ANEXO C. Código fuente ejemplo}

\noindent Fragmento de un archivo de configuración utilizado durante el desarrollo:

\begin{verbatim}
{
    "sample_rate": 50,
    "filter_type": "butterworth",
    "cloud_sync": true
}
\end{verbatim}

\noindent\textit{Fuente: elaboración propia.}

\end{document}
